\section{Einführung}
Die Forschungsrichtung \emph{Intelligente und Interaktive Systeme} behandelt Themen verschiedener, naturwissenschafltlicher Studienrichtungen, dazu zählen hauptsächlich Informatik, Mathematik und Physik. Hauptaufgabe ist es Roboter zu entwickeln, die durch verschiedene Sensoren oder Kameras die Umwelt wahrnehmen und durch selbständig erlerntes Wissen diese Daten interpretieren, um Arbeiten durchzuführen.\footnote{https://iis.uibk.ac.at/, 22.05.15}\ Das Wort \emph{Interaktiv} stellt die zweite, wichtige Eigenschaft der Roboter dar. Sie sollen mit der Umwelt interaktiv agieren können, zum Beispiel mittels Gesten eines Menschen.
\section{Überblick der Forschungsrichtung} \label{überblick}
Dieses Kapitel soll einen kleinen Überblick über die 3 wichtigen Teilgebiete von \emph{Intelligente und Interaktive Systeme} geben, diese wären Computer Vision, Maschinelles Lernen und Robotik.
	\subsection{Computer Vision} \label{vis}
	\enquote{Researchers  in  computer  vision  have  been  developing, in		parallel,  mathematical  techniques for recovering the three-dimensional shape and appearance of objects in imagery.}\cite{Szeliski:2010:CVA:1941882} Das erzeugte Bild soll sich möglichst nahe am menschlich visuellen System orientieren. Durch Methoden der Bildverarbeitung (Filter) sollen dann Eigenschaften von Objekten im Bild extrahiert werden. Diese Eigenschaften können dann zum Beispiel beim \emph{maschinellen Lernen} dazu verwendet werden, um das künstlich neuronale Netzwerk weiterlernen zu lassen.
	\subsection{Maschinelles Lernen} \label{masch}
	\enquote{Machine learning is the study of computational methods for improving performance by mechanizing the acquisition of knowledge from experience.}\cite{Langley:1995:AML:219717.219768} Solche Erfahrungen dienen dann zum besseren Lösen von zukünftigen Problemen. \emph{Maschinelles Lernen} ist sehr verwandt mit dem Bereich der \emph{Künstlichen Intelligenz}, dort wird genauer erforscht, wie man Erfahrungen verwendet, um daraus Schlüsse zu ziehen. Verwendete Methoden dafür sind beispielweise neuronale Netzwerke oder genetische Algorithmen.\footnote{http://www.dbai.tuwien.ac.at/education/AIKonzepte/Folien, 22.05.15}
	\subsection{Robotik}
	\enquote{Das Themengebiet der Robotik [...], befasst sich mit dem Versuch, das Konzept der Interaktion mit der physischen Welt auf Prinzipien der Informationstechnik, sowie auf eine technisch machbare Kinetik zu reduzieren.}\footnote{http://de.wikipedia.org/w/index.php?title=Robotik\&oldid=142318841} Das stellt nun die Verbindung zwischen den vorigen Kapiteln \ref{vis} \& \ref{masch}\ her. Das Gesehene wird dann in Kombination mit dem Erlernten auf die Bewegung des Roboters umgesetzt.
\section{Meilensteine}
Für die die schon erreichten Erfolge im Gebiet der \emph{Intelligente und Interaktive Systeme} sind besonders die folgenden Errungenschaften verantwortlich. Sie sind essentiell für die Programmierung eines Roboters.
	\subsection{Match Moving}
	\enquote{Matchmoving is a technique that allows computer graphics to be inserted into live-action footage with correct position, scale, orientation, and motion.}\cite{dobbert2012matchmoving} Diese Technik macht es möglich, mit dem Computer erzeugte 3D-Objekte in einen bewegten Film einzufügen. Das ist eine der wichtigsten Erfindungen für die heutige Filmindustrie, welche ermöglicht, großartige Szenen mit sehr vielen Objekten zu erzeugen, ohne es in Realität aufgenommen zu haben, wie man beispielhaft im Film \emph{Herr der Ringe} sieht.
	\subsection{Deep Belief Network}
	\enquote{Ein Deep Belief Network ist ein künstliches neuronales Netz mit mehreren Schichten, bei dem jede Schicht aus einer Restricted Boltzmann Maschine besteht.}\footnote{http://ganymed.imib.rwth-aachen.de/lehmann/seminare/bv\_2008-06.pdf, 22.05.15} Dem Netzwerk werden Bilder als Trainingsdaten gegeben, um daraus komplexe Regelmäßigkeiten zu erkennen. Nach genügend Bildern ist es dann in der Lage, die erlernten Zusammenhänge in neuen Bildern zu erkennen. Zum Beispiel, wenn man dem Netzwerk genügend Bilder mit Katzenmotiven zum Lernen gibt, ist es dann in der Lage automatisch Katzen aus ganz unterschiedlichen Perspektiven in neuen Bildern zu erkennen und zu extrahieren.
	\subsection{Inpainting}
	Unter diesem Begriff versteht man die Rekonstruktion von Zerstörten oder nicht vorhandenen Bereichen in einem Bild. Ein Roboter benötigt diese Technik, um unwichtige Objekte im Vordergrund des Bildes zu entfernen, um dann diese entfernten Bereiche mit \emph{Inpainting} soweit wiederherzustellen, sodass das Bild zur Weiterverarbeitung verwendet werden kann.
\section{Wichtige Personen}
Die nachfolgenden Kapitel stellen kurz 2 sehr wichtige Persönlichkeiten in den oben vorgestellten Forschungsrichtungen im Kapitel \ref{überblick} vor.
	\subsection{Leslie Valiant}
	Leslie Valiant ist ein britischer Wissenschaftler, der an der Universität Warwick in Informatik promovierte. Er wurde in Ungarn geboren und lehrt nun an der Universität in Harvard. Er erhielt 2010 den Turing Award für ausgezeichnete Forschungsergebnisse in den Themen Künstliche Intelligenz, Komplexitätstheorie und \emph{Maschinelles Lernen}. Von ihm stammt das PAC-Modell (Probably Approximately Correct Learning), das ein Framework für das maschinelle Lernen ist.\footnote{http://amturing.acm.org/award\_winners/valiant\_2612174.cfm}
	\subsection{Geoffrey Hinton}
	Geoffrey Hinton ist ebenfalls ein britischer Wissenschaftler, der die Studien Informatik und Psychologie an der Universität in Cambridge abgeschlossen hat. Er beschäftigt sich hauptsächlich mit neuronalen Netzwerken, insbesondere mit \emph{Deep Belief Networks}. In diesem Thema ist er Vorreiter und hat mehrere wichtige Trainingsalgorithmen für solche Netzwerke entwickelt. Heute arbeitet er Teilzeit als Professor an der Universität in Toronto, als auch bei Google in Kalifornien.\footnote{http://www.cs.toronto.edu/~hinton/}
\section{Schlusswort}
Diese Seminararbeit gibt einen kurzen Überblick über das Forschungsgebiet und erklärt die wichtigen Personen und zugehörigen Meilensteine. Da \emph{Intelligente und Interaktive Systeme} noch eine relative junge Disziplin und dieser Bereich auch sehr nachgefragt in der Industrie ist, wird man in Zukunft sehr schnell Fortschritte erreichen.