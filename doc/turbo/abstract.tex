Diese Bachelorarbeit implementiert das Turbo-Kode-Verfahren, verpackt in einem R-Paket. Dabei stellt diese Kodierung ein Teilgebiet der Kanalkodierung dar, welche bei heutigen Kommunikationskanälen nicht mehr wegzudenken ist. Da auftretende Fehler während der Übertragung der Nachricht, dadurch wieder rekonstruierbar sind. Hauptanwendungsgebiete sind dabei die Mobil- und Satellitenkommunikation, da hierbei wichtig ist, dass eine Nachricht nicht mehrmals übertragen werden muss, bis sie korrekt ankommt. Um für zukünftige Studierende das Erlernen dieser Kommunikationstechnik zu erleichtern, sollten geeignete Visualisierungen geschaffen werden. Diese Arbeit steht in enger Verbindung mit den beiden Bachelorarbeiten, die Blockkodierung und Faltungskodierung behandeln. Alle drei Kodierungsverfahren sollten in das selbe R-Paket inkludiert werden.
