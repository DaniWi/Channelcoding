Das Ziel der Bachelorarbeit war die Erstellung eines R-Paketes zur Umsetzung des  Turbo-Kode-Verfahrens, welches eine Kanalkodierung darstellt. Dabei stand der didaktische Nutzen für zukünftige Studierende im Mittelpunkt. Deswegen bietet das Paket lehrreiche Visualisierungen der Kodierung und Dekodierung an. Dadurch soll das Verständnis der Verfahren erleichtert werden. Darüber hinaus werden Berichte bei den Simulationen erstellt, die eine einfache Analyse der Leistungsfähigkeit der verschiedenen Kanalkodierungsverfahren zulässt.

Die beiden alternativen Verfahren, Blockkodes und Faltungskodes, wurden von den Kollegen Nocker \cite{nocker} und Wimmer \cite{wimmer} in ihren Bachelorarbeiten umgesetzt. Alle drei Kodierungen wurden in ein gemeinsames Paket verpackt. Somit ist ein kompaktes R-Paket entstanden, das alle Kanalkodierungsverfahren beherrscht und dadurch vielseitig einsetzbar ist.  

Als Erweiterung der vorhandenen Funktionalität würde sich eine Ausweitung der Visualisierungen auf längere Nachrichten anbieten. Dadurch könnte die Darstellung auf mehr als 18 Bits erweitert werden. Das Turbo-Kode-Verfahren könnte um die Verwendung von Blockkodes ergänzt werden. Zusätzlich zur parallelen wäre die iterative Verkettung interessant, da dies der erste Ansatz der Kodeverkettung war. Als alternativer Algorithmus, bei der Dekodierung, könnte der aufwändigere BCJR-Algorithmus implementiert werden, der bei längeren Nachrichten leistungsfähiger ist, als der Viterbi-Algorithmus \cite[233-236]{schoenfeld2012informations}.