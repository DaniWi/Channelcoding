\begin{longtable}{|p{\textwidth}|}
\hline
\rowcolor{lightblue}TurboGetPunctuationMatrix\\
\hline
\\
\texttt{TurboGetPunctuationMatrix(punctuation.vector, visualize)}\\
\\
Erzeugt eine Punktierungsmatrix von einem mitgegebenen Vektor für das Turbo-Kode-Verfahren (drei Zeilen). Es dürfen keine Null-Spalten entstehen!\\
\\
\textbf{Argumente:}\\
\texttt{punctuation.vector} - Vektor der in eine Punktierungsmatrix transformiert wird. Eine 1 behaltet das Bit, eine 0 verwirft das Bit.\\
\texttt{visualize} - Wenn \texttt{TRUE} wird Punktierungsmatrix dargestellt. Standard: \texttt{FALSE}\\
\\
\textbf{Rückgabewert:}\\
Punktierungsmatrix, die für das Turbo-Kode-Verfahren geeignet ist.\\
\\
\hline
\caption[TurboGetPunctuationMatrix]{TurboGetPunctuationMatrix - Funktionserklärung}
\end{longtable}