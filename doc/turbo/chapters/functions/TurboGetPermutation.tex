\begin{longtable}{|p{\textwidth}|}
\hline
\rowcolor{lightblue}TurboGetPermutation\\
\hline
\\
\texttt{TurboGetPermutation(message.length, coder.info, type, args, visualize)}\\
\\
Erzeugt einen Permutationsvektor für die Interleaver beim Turbo-Kode-Verfahren. Dabei können verschiedene Typen von Permutationen ausgewählt werden. (RANDOM, PRIMITIVE, CYCLIC, BLOCK, HELICAL, DIAGONAL) Die genauere Erklärung des jeweiligen Typs findet sich im Kapitel \ref{cha:implementation}. Den jeweiligen Typen müssen spezielle Argumente in einer Liste mitgegebenen werden:
\begin{itemize}
\item RANDOM: Benötigt keine Argumente, wird komplett zufällig erstellt.
\item PRIMITVE: Erzeugt einen Vektor (1,2,3...) und schiebt diesen um den \emph{root}-Wert.
\item CYCLIC: Benötigt die Argumente \emph{cols} und \emph{rows} für die Erzeugung der Matrix und \emph{distance} um den Permutationsvektor zu erzeugen.
\item BLOCK: Benötigt die Argumente \emph{cols} und \emph{rows} für die Erzeugung der Matrix. Matrix wird von links nach rechts befüllt und von oben nach unten ausgelesen.
\item HELICAL: Benötigt die Argumente \emph{cols} und \emph{rows} für die Erzeugung der Matrix. Matrix wird von links nach rechts befüllt und von links oben nach rechts unten ausgelesen.
\item DIAGONAL: Benötigt die Argumente \emph{cols} und \emph{rows} für die Erzeugung der Matrix. Matrix wird von links nach rechts befüllt und werden diagonal ausgelesen.
\end{itemize} \\
\\
\textbf{Argumente:}\\
\texttt{message.length} - Länge der Nachricht die kodiert werden möchte.\\
\texttt{coder.info} - Faltungskodierer der mit den Funktionen \emph{ConvGenerateEncoder, ConvGenerateRscEncoder} erzeugt werden kann. Standard: \emph{ConvGenerateRscEncoder(2,2,c(5,7))}\\
\texttt{type} - Typ des erzeugten Permutationsvektors. Standard: RANDOM\\
\texttt{args} - Argumente für den jeweiligen Typ in einer Liste. Standard: NULL\\
\texttt{visualize} - Wenn TRUE wird die Permutationsmatrix dargestellt. Standard: FALSE\\
\\
\textbf{Rückgabewert:}\\
Permutationsvektor mit der richtigen Länge für die mitgegebene Nachricht und Kodierer.\\
\\
\hline
\caption{TurboGetPermutation - Funktionserklärung}
\end{longtable}