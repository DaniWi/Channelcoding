\begin{longtable}{|p{\textwidth}|}
\hline
\rowcolor{lightblue}ChannelcodingSimulation\\
\hline
\\
\texttt{ChannelcodingSimulation(msg.length, min.db, max.db, db.interval, iterations.per.db, turbo.decode.iterations, visualize)}\\
\\
Simulation von Block-, Faltungs- und Turbo-Kodes und Vergleich der Ergebnisse von verschiedenen Signal/Rausch-Verhältnissen. Dabei wird die jeweilige Simulation der einzelnen Kodierungsverfahren aufgerufen.\\
\\
\textbf{Argumente:}\\
\texttt{msg.length} - Länge der Nachricht. Standard: 100\\
\texttt{min.db} - Untergrenze des Signal/Rausch-Verhältnisses. Standard: 0.1\\
\texttt{max.db} - Obergrenze des Signal/Rausch-Verhältnisses. Standard: 2.0\\
\texttt{db.interval} - Schrittweite pro Erhöhung des Signal/Rausch-Verhältnisses. Standard: 0.1\\
\texttt{iterations.per.db} - Iterationen pro Signal/Rausch-Verhältnisses. Standard: 100\\
\texttt{turbo.decode.iterations} - Anzahl von Turbo-Dekodierung-Iterationen. Standard: 5\\
\texttt{visualize} - Wenn \texttt{TRUE} wird PDF-Bericht erstellt. Standard: \texttt{FALSE}\\
\\
\textbf{Rückgabewert:}\\
Dataframe, welches alle drei Simulationsergebnisse der Kodierungsverfahren beinhaltet.\\
\\
\hline
\caption[ChannelcodingSimulation]{ChannelcodingSimulation - Funktionserklärung}
\end{longtable}