\begin{longtable}{|p{\textwidth}|}
\hline
\rowcolor{lightblue}TurboSimulation\\
\hline
\\
\texttt{TurboSimulation(coder, permutation.type, permutation.args, decode.iterations, msg.length, min.db, max.db, db.interval, iterations.per.db, punctuation.matrix, visualize)}\\
\\
Automatische Simulation eines Kodierungs- und Dekodierungsverfahrens von Turbo-Kodes. Nach dem Kodieren wird der resultierende Kode verrauscht und im Anschluss dekodiert. Für das jeweilige SNR wird dieses Verfahren mehrmals (iterations.per.db) wiederholt und am Ende ein Durchschnitt der Bitfehlerrate berechnet.\\
\\
\textbf{Argumente:}\\
\texttt{coder} - Kodierer der für Kodierung und Dekodierung verwendet wird. Kann mittels den Funktionen \emph{ConvGenerateEncoder, ConvGenerateRscEncoder} erzeugt werden. Standard: \emph{ConvGenerateRscEncoder(2,2,c(5,7))}\\
\texttt{permutation.type} - Typ des Permutationsvektors. Standard: PRIMITIVE\\
\texttt{permutation.args} - Argumente für die Erzeugung des Permutationsvektors. Standard: \emph{list(root=0)}\\
\texttt{decode.iterations} - Anzahl von Iterationen bei der Dekodierung. Standard: 5\\
\texttt{msg.length} - Länge der Nachricht. Standard: 100\\
\texttt{min.db} - Untergrenze des SNR. Standard: 0.1\\
\texttt{max.db} - Obergrenze des SNR. Standard: 2.0\\
\texttt{db.interval} - Schrittweite pro Erhöhung des SNR. Standard: 0.1\\
\texttt{iterations.per.db} - Iterationen pro SNR zur Durchschnittsbildung. Standard: 100\\
\texttt{punctuation.matrix} - Verwendete Punktierungsmatrix. Standard: wird nicht punktiert\\
\texttt{visualize} - Wenn TRUE wird PDF-Bericht erstellt. Standard: FALSE\\
\\
\textbf{Rückgabewert:}\\
Dataframe das für jedes SNR eine Bitfehlerrate beinhaltet.\\
\\
\hline
\caption{TurboSimulation - Funktionserklärung}
\end{longtable}