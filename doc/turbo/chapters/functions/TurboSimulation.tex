\begin{longtable}{|p{\textwidth}|}
\hline
\rowcolor{lightblue}TurboSimulation\\
\hline
\\
\texttt{TurboSimulation(coder, permutation.type, permutation.args, decode.iterations, msg.length, min.db, max.db, db.interval, iterations.per.db, punctuation.matrix, visualize)}\\
\\
Automatische Simulation eines Kodierungs- und Dekodierungsverfahrens von Turbo-Kodes. Nach dem Kodieren wird der resultierende Kode verrauscht und im Anschluss dekodiert. Für das jeweilige Signal/Rausch-Verhältnis wird dieses Verfahren (\texttt{iterations.per.db}) mehrmals wiederholt und am Ende ein Durchschnitt der Bitfehlerrate berechnet.\\
\\
\textbf{Argumente:}\\
\texttt{coder} - Kodierer der für Kodierung und Dekodierung verwendet wird. Standard: \texttt{ConvGenerateRscEncoder(2,2,c(5,7))}\\
\texttt{permutation.type} - Typ des Permutationsvektors. Standard: \texttt{PRIMITIVE}\\
\texttt{permutation.args} - Argumente für die Erzeugung des Permutationsvektors. Standard: \texttt{list(root=0)}\\
\texttt{decode.iterations} - Anzahl der Dekodierungsiterationen. Standard: 5\\
\texttt{msg.length} - Länge der Nachricht. Standard: 100\\
\texttt{min.db} - Untergrenze des Signal/Rausch-Verhältnisses. Standard: 0.1\\
\texttt{max.db} - Obergrenze des Signal/Rausch-Verhältnisses. Standard: 2.0\\
\texttt{db.interval} - Schrittweite pro Erhöhung des Signal/Rausch-Verhältnisses. Standard: 0.1\\
\texttt{iterations.per.db} - Iterationen pro Signal/Rausch-Verhältnis. Standard: 100\\
\texttt{punctuation.matrix} - Verwendete Punktierungsmatrix. Standard: wird nicht punktiert\\
\texttt{visualize} - Wenn \texttt{TRUE} wird PDF-Bericht erstellt. Standard: \texttt{FALSE}\\
\\
\textbf{Rückgabewert:}\\
Dataframe das die Bitfehlerrate je Signal/Rausch-Verhältnis beinhaltet.\\
\\
\hline
\caption[TurboSimulation]{TurboSimulation - Funktionserklärung}
\end{longtable}