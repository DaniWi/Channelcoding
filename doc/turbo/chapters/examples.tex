\section{Erzeugen von Kodierer, Permutationsvektor und Punktierungsmatrix}
\label{sec:example_createHelpers}

\begin{lstlisting}[caption=blabla]
input <- c(1,0,1,1,0)

message.encoded <- TurboEncode(input)
result <- TurboDecode(message.encoded, 5)
print(result)

coder <- ConvGenerateRscEncoder(2, 2, c(5, 7))
perm <- TurboGetpermutation(length(input), coder, "RANDOM")
\end{lstlisting}

\begin{lstlisting}
input <- c(1,0,1,1,0)

message.encoded <- TurboEncode(input)
result <- TurboDecode(message.encoded, 5)
print(result)

coder <- ConvGenerateRscEncoder(2, 2, c(5, 7))
perm <- TurboGetpermutation(length(input), coder, "RANDOM")
\end{lstlisting}

\section{Kodieren und Dekodieren ohne Punktierung}
\label{sec:example_withoutPunctuation}

\section{Kodieren und Dekodieren mit Punktierung}
\label{sec:example_withPunctuation}

\section{Simulationen}
\label{sec:example_simulations}

\subsection{Turbo-Kode-Simulation}

\subsection{Kanalkodierungs-Simulation}

\subsection{Vergleich mehrerer Simulationen}