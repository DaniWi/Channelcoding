Aufgrund der steigenden Anzahl von mobilen Anwendungsgeräte, ist die Mobilkommunikation ein wichtiger Teilbereich der Nachrichtentechnik. Zur Datenübertragung werden elektromagnetische Funkwellen benutzt, die aufgrund der äußeren Umwelteinflüsse , störanfälliger als kabelgebundene Übertragungskanäle sind. Deswegen benötigt es eine Technik, welche diese Störungen während der Übertragung beheben kann. Dabei kommen die Kanalkodierungsverfahren ins Spiel, welche es ermöglichen vor der Versendung der Daten zusätzlichen Information der Nachricht hinzuzufügen, um beim Empfang die entstandenen Fehler zu korrigieren.

Eine fortgeschrittenes Verfahren bei der Kanalkodierung sind die Turbo-Kodes. Diese kombinieren die bereits bekannten Kodierungen, wodurch ein Gesamtkode entsteht, dessen Leistungsfähigkeit nahe der theoretischen Grenze liegt. Aufgrund der langen Signallaufzeiten in der Mobil- und Satellitenkommunikation werden Turbo-Kodes eingesetzt. Da bei einer fehlerhaften Übertragung, ein erneutes Senden nicht praktikabel wäre. Deswegen kann aus den Informationen, die vor dem Versand hinzugefügt wurden, die Originalnachricht trotz Störungen wiederhergestellt werden. Schon seit den 80er Jahren wird diese Art der Kodierung in Bereichen der Speichersystem, wie CD oder DVD, eingesetzt. Für den heutigen Erfolg der Mobilfunkstandards LTE und UMTS ist dieses Verfahren ausschlaggebend, um die Übertragungsgeschwindigkeiten zu erreichen. Auch die bekannte ESA Raumsonde Rosetta, die einen Kometen seit August 2014 umkreist, benutzt diese Technik für die Kommunikation mit dem Kontrollzentrum auf der Erde.~\cite[S.~242~f.]{schoenfeld2012informations} 

Ziel dieser Arbeit war es das Prinzip des Turbo-Kode-Verfahrens, mit der Programmiersprache R, in ein wiederverwendbares R-Paket zu implementieren. Für zukünftige Studierende sollte eine visuelle Darstellung geschaffen werden, die den prinzipiellen Ablauf der Kodierung und Dekodierung darstellen. Durch die praktische Anwendung und grafische Visualisierung, soll das Verständnis von Turbo-Kodes erleichtert werden. Somit ist das R-Paket eine optimale Ergänzung zum theoretischen Unterricht an der Universität.

Die beiden weiteren Bachelorarbeiten \enquote{Kanalkodierung mit Blockkodes} von Benedikt Wimmer \cite{wimmer} und \enquote{Kanalkodierung mit Faltungskodes} von Martin Nocker \cite{nocker} vervollständigen das R-Paket, um die gesamte Funktionalität der Kanalkodierung zu integrieren. Dadurch erhalten die Studierenden ein kompaktes und einfach verwendbares R-Paket, das alle wichtigen Bestandteile des Kanalkodierungsverfahren beinhaltet und eine hilfreiche visuelle Darstellung anbietet.   