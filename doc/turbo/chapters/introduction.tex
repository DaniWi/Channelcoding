Aufgrund der steigenden Anzahl von mobilen Anwendungsgeräten ist die Mobilkommunikation ein wichtiger Teilbereich der Nachrichtentechnik. Zur Datenübertragung werden elektromagnetische Funkwellen benutzt, die aufgrund der äußeren Umwelteinflüsse störanfälliger als kabelgebundene Übertragungskanäle sind. Deswegen benötigt es eine Technik, welche es dem Empfänger ermöglicht, die Fehler die aus den Störungen resultieren, beheben zu können. Dabei kommen die Verfahren der Kanalkodierung zum Einsatz. Diese ermöglichen es vor dem Absenden der Daten, zusätzliche Informationen zur Nachricht hinzuzufügen, um beim Empfang die entstandenen Fehler zu korrigieren.

Ein fortgeschrittenes Verfahren bei der Kanalkodierung sind die Turbo-Kodes. Diese kombinieren die bereits bekannten Kodierungen, wodurch ein Gesamtkode entsteht, dessen Leistungsfähigkeit nahe der maximalen theoretischen Übertragungsgrenze liegt. Aufgrund der langen Signallaufzeiten in der Mobil- und Satellitenkommunikation werden Turbo-Kodes eingesetzt, da bei einer fehlerhaften Übertragung ein erneutes Senden nicht praktikabel wäre. Deswegen kann aus den Informationen, die vor dem Versand hinzugefügt wurden, die Originalnachricht trotz Störungen wiederhergestellt werden. Schon seit den 1980er Jahren wird diese Art der Kodierung in Bereichen der Speichersysteme, wie CD oder DVD, eingesetzt. Für den heutigen Erfolg der Mobilfunkstandards LTE und UMTS ist dieses Verfahren ausschlaggebend, um die Übertragungsgeschwindigkeiten zu erreichen. Auch die bekannte ESA Raumsonde Rosetta, die seit August 2014 einen Kometen umkreist, benutzt diese Technik für die Kommunikation mit dem Kontrollzentrum auf der Erde.~\cite[S.~242~f.]{schoenfeld2012informations} 

Ziel dieser Arbeit war es, das Prinzip des Turbo-Kode-Verfahrens, mit der Programmiersprache R, in ein wiederverwendbares R-Paket zu implementieren. Für zukünftige Studierende sollte eine visuelle Darstellung geschaffen werden, die den prinzipiellen Ablauf der Kodierung und Dekodierung darstellt. Durch die praktische Anwendung und grafische Visualisierung soll das Verständnis von Turbo-Kodes erleichtert werden. Somit ist das R-Paket eine optimale Ergänzung zum theoretischen Unterricht an den Universitäten.

Die beiden weiteren Bachelorarbeiten \enquote{Kanalkodierung mit Blockkodes} von Benedikt Wimmer \cite{wimmer} und \enquote{Kanalkodierung mit Faltungskodes} von Martin Nocker \cite{nocker} vervollständigen das R-Paket, um die gesamte Funktionalität der Kanalkodierung zu integrieren. Dadurch erhalten die Studierenden ein kompaktes und einfach verwendbares R-Paket, welches alle wichtigen Bestandteile des Kanalkodierungsverfahren beinhaltet und eine hilfreiche visuelle Darstellung anbietet.   

Diese Bachelorarbeit startet in Kapitel~\ref{cha:basics} mit den Grundlagen der Kanalkodierung. Diese werden benötigt, um anschließende Verfahren besser verstehen zu können. Danach werden in Kapitel~\ref{cha:technologies} kurz die verwendeten Technologien und deren Vorteile erläutert. Die genaue Erklärung der Implementation des Paketes, wird in Kapitel~\ref{cha:implementation} veranschaulicht. Zur Verwendung der Funktionen finden sich in Kapitel~\ref{cha:interface} die Schnittstelle des Paketes. Um eine Hilfe bei dem Verständnis der Visualisierungen zu erhalten, werden in Kapitel~\ref{cha:visualization} alle wichtigen Darstellungen im Detail erklärt. Der Einsteig zur Verwendung des R-Paketes soll mit den Beispielen von Kapitel~\ref{cha:examples} erleichtert werden. Einige mögliche Erweiterungen werden zum Abschluss noch in Kapitel~\ref{cha:result} vorgebracht.