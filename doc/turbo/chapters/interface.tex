\definecolor{lightblue}{RGB}{150,180,255}

Dieses Kapitel soll als zusätzliche Hilfe bei der Verwendung der R-Funktionen dienen, da hier die genauen Schnittstellen der Funktionen erklärt sind und dadurch die Verwendung erleichtert werden sollte. Nachdem das Paket installiert ist, steht die Hilfe auch im RStudio zur Verfügung, diese ist allerdings auf Englisch und nicht so umfangreich. Falls aus diesem Kapitel die Verwendung noch nicht komplett klar ist, kann im Kapitel \ref{cha:implementation} die genaue Implementierung nachgelesen werden und daraus die weitere Verwendung geschlossen werden.
\section{Turbo-Kode Funktionen}

\subsection{Kodieren}
\label{sec:interface_encode}
\begin{frame}{Kodierer Information}

\begin{itemize}
\tightlist
\item
  Nicht-Rekursiver Kodierer
\item
  Anzahl von Ausgängen : \[N=2\]
\item
  Anzahl von Registern : \[M=2\]
\item
  Generatoren : \[(1,7)_8 = \begin{pmatrix}001 \\ 111 \\ \end{pmatrix}\]
\end{itemize}

\end{frame}

\begin{frame}{Kodierer Matrix : Nächster Zustand}

\begin{longtable}[c]{@{}lcc@{}}
\toprule
& Bit 0 & Bit 1\tabularnewline
\midrule
\endhead
Zustand 1 & 0 & 2\tabularnewline
Zustand 2 & 2 & 0\tabularnewline
Zustand 3 & 3 & 1\tabularnewline
Zustand 4 & 1 & 3\tabularnewline
\bottomrule
\end{longtable}

\end{frame}

\begin{frame}{Kodierer Matrix : Voriger Zustand}

\begin{longtable}[c]{@{}lccc@{}}
\toprule
& Bit 0 & Bit 1 & 2.Möglichkeit\tabularnewline
\midrule
\endhead
Zustand 1 & 0 & 1 & -1\tabularnewline
Zustand 2 & 3 & 2 & -1\tabularnewline
Zustand 3 & 1 & 0 & -1\tabularnewline
Zustand 4 & 2 & 3 & -1\tabularnewline
\bottomrule
\end{longtable}

\end{frame}

\begin{frame}{Kodierer Matrix : Ausgangsbits}

\begin{longtable}[c]{@{}lcc@{}}
\toprule
& Bit 0 & Bit 1\tabularnewline
\midrule
\endhead
Zustand 1 & 00 & 11\tabularnewline
Zustand 2 & 00 & 11\tabularnewline
Zustand 3 & 01 & 10\tabularnewline
Zustand 4 & 01 & 10\tabularnewline
\bottomrule
\end{longtable}

\end{frame}

\begin{frame}{Turbo-Kodierer Information}

\begin{itemize}
\tightlist
\item
  Interleaver : (1, 2, 3, 4, 5, 6, 7)
\item
  Kode-Rate : \[\frac{1}{3}\]
\end{itemize}

\end{frame}

\begin{frame}{Turbo-Kodierer}

\begin{itemize}
\tightlist
\item
  Input : 1, 1, 1, 0, 0, 0, 1, 1, 1, 1, 1, 1, 1, 1, 1, 1, 1, 1, 1, 1,
  \ldots{}
\end{itemize}

\begin{center}
\begin{tikzpicture}[node distance=8mm, inner sep=2mm, outer sep=0, >=stealth, font=\tiny]

\tikzstyle{encoderStyle} = [rectangle, draw, minimum height=height("Inp")+4mm]
\tikzstyle{interleaverStyle} = [encoderStyle]
\tikzstyle{startEndStyle} = [encoderStyle]
\tikzstyle{multiplexerStyle} = [encoderStyle]
\tikzstyle{showStyle} = [rectangle, draw, thick, draw=red, text=red, inner sep=1.5mm]
\tikzstyle{arrowStyle} = [->]
\tikzstyle{showDrawStyle} = [->, red, thick]

\node [interleaverStyle] (interleaver) {Interleaver};
\node [encoderStyle] (encoder1) [above right=of interleaver] {Kodierer 1};
\node [encoderStyle] (encoder2) [below right=of interleaver] {Kodierer 2};
\node [] (multiplexerTemp) [right=of encoder1] {};
\node [multiplexerStyle, rotate=90] (multiplexer) at (multiplexerTemp.east) {Multiplexer};
\node [startEndStyle] (start) [above left=of interleaver] {Input};
\node [startEndStyle] (end) [right=of multiplexerTemp] {Output};

\draw [arrowStyle] (interleaver.south) |- node (afterInterleaver) {}(encoder2.west);
\draw [arrowStyle] (encoder1.east) -- node (afterParity1) {} (multiplexer.north);
\draw [arrowStyle] (start.east)  -| node[midway] (afterInput) {} (interleaver.north);
\draw [arrowStyle] (afterInput.center) |- (encoder1.west);
\draw [arrowStyle] (encoder2.east) -| ++(0.2,0.7) node (afterParity2) {} |- ($(multiplexer.north)+(0,-.5)$);
\draw [arrowStyle] (afterInput.center) |- ($(multiplexer.north)+(0,.5)$);
\draw [arrowStyle] (multiplexer.south) -- node (afterMultiplexer) {} (end.west);

\begin{scope}[on background layer]
\node [draw, fit=(encoder1) (interleaver) (encoder2) (multiplexer)] (background) {};
\end{scope}

\visible<2>{
  \node [showStyle] (showAbove) [above=4mm of background] {\textbf{1, 1, 1, 0, 0, 0, 1, 1, 1, 1, 1, 1, 1, 1, 1, 1, 1, 1, 1, 1, ...}};
  \draw [showDrawStyle] (showAbove) -- (afterInput.center);
  \node [showStyle] (showBelow) [below=4mm of background] {\textbf{3, 1, 1, 0, 0, 0, 1, 1, 1, 1, 1, 1, 1, 1, 1, 1, 1, 1, 1, 1, ...}};
  \draw [showDrawStyle] (showBelow) -- (afterParity1.center);
  }
\visible<3>{
  \node [showStyle] (showAbove) [above=4mm of background] {\textbf{1, 1, 1, 0, 0, 0, 1, 1, 1, 1, 1, 1, 1, 1, 1, 1, 1, 1, 1, 1, ...}};
  \draw [showDrawStyle] (showAbove) -- (afterInput.center);
  \node [showStyle] (showBelow) [below=4mm of background] {\textbf{2, 1, 1, 0, 0, 0, 1, 1, 1, 1, 1, 1, 1, 1, 1, 1, 1, 1, 1, 1, ...}};
  \draw [showDrawStyle] (showBelow) -- (afterInterleaver.center);
  }
\visible<4>{
  \node [showStyle] (showAbove) [above=4mm of background] {\textbf{2, 1, 1, 0, 0, 0, 1, 1, 1, 1, 1, 1, 1, 1, 1, 1, 1, 1, 1, 1, ...}};
  \draw [showDrawStyle] (showAbove) -- (afterInterleaver);
  \node [showStyle] (showBelow) [below=4mm of background] {\textbf{4, 1, 1, 0, 0, 0, 1, 1, 1, 1, 1, 1, 1, 1, 1, 1, 1, 1, 1, 1, ...}};
  \draw [showDrawStyle] (showBelow) -- (afterParity2.center);
  }

\end{tikzpicture}
\end{center}

\begin{itemize}
\tightlist
\item
  Output : 5, 1, 1, 0, 0, 0, 1, 1, 1, 1, 1, 1, 1, 1, 1, 1, 1, 1, 1, 1,
  1, 1, 1, 0, 0, 0, 1, 1, 1, 1, 1, 1, 1, 1, 1, 1, 1, 1, 1, 1, 1, 1, 1,
  0, 0, \ldots{}
\end{itemize}

\end{frame}


\subsection{Dekodieren}
\label{sec:interface_decode}
\begin{longtable}{|p{\textwidth}|}
\hline
\rowcolor{lightblue}TurboDecode\\
\hline
\\
\texttt{TurboDecode(code, perumtation.vector, iterations, coder.info, parity.index, punctuation.matrix, visualize)}\\
\\
Dekodiert eine Nachricht mittels dem Turbo-Kode-Verfahren. Dabei werden zuerst die Punktierungsbits wieder eingefügt, falls Punktierung verwendet wurde. Danach erfolgt die Dekodierung, wobei die Ergebnisse der Dekodierer als Eingang des Nächsten verwendet werden. Dieser Vorgang wird mehrmals iterativ ausgeführt, je nach Anzahl der gewählten Iterationen.\\
\\
\textbf{Argumente:}\\
\texttt{code} - Nachricht die dekodiert wird. Der genaue Signalpegel wird verwendet.\\
\texttt{permutation.vector} - Permutationsvektor der für den Interleaver verwendet wird. Standard: \texttt{PRIMITIVE (root=0)}\\
\texttt{iterations} - Anzahl der Dekodierungsiterationen. Standard: 1\\
\texttt{coder.info} - Faltungskodierer der bei der Kodierung bereits verwendet wurde. Standard: \texttt{ConvGenerateRscEncoder(2,2,c(5,7))}\\
\texttt{parity.index} - Gibt an welcher Ausgang vom mitgegebenen Kodierer verwendet wird. Standard: letzter Ausgang\\
\texttt{punctuation.matrix} - Die mitgegebene Punktierungsmatrix wird am Anfang des Dekodiervorgangs,  für das Einfügen der jeweiligen Bits verwendet. Standard: keine Punktierung\\
\texttt{visualize} - Wenn \texttt{TRUE} wird PDF-Bericht erstellt. Standard: \texttt{FALSE}\\
\\
\textbf{Rückgabewert:}\\
Liste welche die dekodierte Nachricht und die genau berechneten Soft-Werte enthält.\\
\\
\hline
\caption[TurboDecode]{TurboDecode - Funktionserklärung}
\end{longtable}

\subsection{Simulation}
\label{sec:interface_simulation}
\begin{longtable}{|p{\textwidth}|}
\hline
\rowcolor{lightblue}TurboSimulation\\
\hline
\\
\texttt{TurboSimulation(coder, permutation.type, permutation.args, decode.iterations, msg.length, min.db, max.db, db.interval, iterations.per.db, punctuation.matrix, visualize)}\\
\\
Automatische Simulation eines Kodierungs- und Dekodierungsverfahrens von Turbo-Kodes. Nach dem Kodieren wird der resultierende Kode verrauscht und im Anschluss dekodiert. Für das jeweilige Signal/Rausch-Verhältnis wird dieses Verfahren mehrmals (\texttt{iterations.per.db}) wiederholt und am Ende ein Durchschnitt der Bitfehlerrate berechnet.\\
\\
\textbf{Argumente:}\\
\texttt{coder} - Kodierer der für Kodierung und Dekodierung verwendet wird. Kann mittels den Funktionen \texttt{ConvGenerateEncoder, ConvGenerateRscEncoder} erzeugt werden. Standard: \texttt{ConvGenerateRscEncoder(2,2,c(5,7))}\\
\texttt{permutation.type} - Typ des Permutationsvektors. Standard: \texttt{PRIMITIVE}\\
\texttt{permutation.args} - Argumente für die Erzeugung des Permutationsvektors. Standard: \texttt{list(root=0)}\\
\texttt{decode.iterations} - Anzahl von Iterationen bei der Dekodierung. Standard: 5\\
\texttt{msg.length} - Länge der Nachricht. Standard: 100\\
\texttt{min.db} - Untergrenze des Signal/Rausch-Verhältnisses. Standard: 0.1\\
\texttt{max.db} - Obergrenze des Signal/Rausch-Verhältnisses. Standard: 2.0\\
\texttt{db.interval} - Schrittweite pro Erhöhung des Signal/Rausch-Verhältnisses. Standard: 0.1\\
\texttt{iterations.per.db} - Iterationen pro Signal/Rausch-Verhältnis zur Durchschnittsbildung. Standard: 100\\
\texttt{punctuation.matrix} - Verwendete Punktierungsmatrix. Standard: wird nicht punktiert\\
\texttt{visualize} - Wenn \texttt{TRUE} wird PDF-Bericht erstellt. Standard: \texttt{FALSE}\\
\\
\textbf{Rückgabewert:}\\
Dataframe das für jedes Signal/Rausch-Verhältnis eine Bitfehlerrate beinhaltet.\\
\\
\hline
\caption[TurboSimulation]{TurboSimulation - Funktionserklärung}
\end{longtable}

\section{Hilfsfunktionen}

\subsection{Permutationvektor erzeugen}
\label{sec:interface_permutation}
\begin{longtable}{|p{\textwidth}|}
\hline
\rowcolor{lightblue}TurboGetPermutation\\
\hline
\\
\texttt{TurboGetPermutation(message.length, coder.info, type, args, visualize)}\\
\\
Erzeugt einen Permutationsvektor für die Interleaver beim Turbo-Kode-Verfahren. Dabei können verschiedene Typen von Permutationen ausgewählt werden (RANDOM, PRIMITIVE, CYCLIC, BLOCK, HELICAL, DIAGONAL). Die genauere Erklärung des jeweiligen Typs findet sich in Kapitel \ref{cha:implementation}. Den jeweiligen Typen müssen spezielle Argumente in einer Liste mitgegebenen werden:
\begin{itemize}
\item RANDOM: Benötigt keine Argumente, wird komplett zufällig erstellt.
\item PRIMITVE: Erzeugt einen Vektor (1,2,3,...) und schiebt diesen um den \texttt{root}-Wert.
\item CYCLIC: Benötigt die Argumente \texttt{cols} und \texttt{rows} für die Erzeugung der Matrix und \texttt{distance} um den Permutationsvektor zu erzeugen.
\item BLOCK: Benötigt die Argumente \texttt{cols} und \texttt{rows} für die Erzeugung der Matrix. Matrix wird von links nach rechts befüllt und von oben nach unten ausgelesen.
\item HELICAL: Benötigt die Argumente \texttt{cols} und \texttt{rows} für die Erzeugung der Matrix. Matrix wird von links nach rechts befüllt und von links oben nach rechts unten ausgelesen.
\item DIAGONAL: Benötigt die Argumente \texttt{cols} und \texttt{rows} für die Erzeugung der Matrix. Matrix wird von links nach rechts befüllt und dann diagonal ausgelesen.
\end{itemize} \\
\\
\textbf{Argumente:}\\
\texttt{message.length} - Länge der Nachricht die kodiert werden möchte.\\
\texttt{coder.info} - Faltungskodierer der mit den Funktionen \texttt{ConvGenerateEncoder, ConvGenerateRscEncoder} erzeugt werden kann. Standard: \texttt{ConvGenerateRscEncoder(2,2,c(5,7))}\\
\texttt{type} - Typ des erzeugten Permutationsvektors. Standard: \texttt{RANDOM}\\
\texttt{args} - Argumente für den jeweiligen Typ in einer Liste. Standard: \texttt{NULL}\\
\texttt{visualize} - Wenn \texttt{TRUE} wird die Permutationsmatrix dargestellt. Standard: \texttt{FALSE}\\
\\
\textbf{Rückgabewert:}\\
Permutationsvektor mit der richtigen Länge für die mitgegebene Nachricht und Kodierer.\\
\\
\hline
\caption[TurboGetPermutation]{TurboGetPermutation - Funktionserklärung}
\end{longtable}

\subsection{Punktierungsmatrix erzeugen}
\label{sec:interface_punctuation}
\begin{longtable}{|p{\textwidth}|}
\hline
\rowcolor{lightblue}TurboGetPunctuationMatrix\\
\hline
\\
\texttt{TurboGetPunctuationMatrix(punctuation.vector, visualize)}\\
\\
Erzeugt eine Punktierungsmatrix von einem mitgegebenen Vektor für das Turbo-Code-Verfahren (drei Zeilen). Es dürfen keine Null-Spalten entstehen!\\
\\
\textbf{Argumente:}\\
\texttt{punctuation.vector} - Vektor der in eine Punktierumatrix transformiert wird. Eine 1 behaltet das Bit, eine 0 verwirft das Bit.\\
\texttt{visualize} - Wenn TRUE wird Punktierungsmatrix dargestellt. Standard: FALSE\\
\\
\textbf{Rückgabewert:}\\
Punktierungsmatrix die für das Turbo-Kode-Verfahren geeignet ist.\\
\\
\hline
\caption{TurboGetPunctuationMatrix - Funktionserklärung}
\end{longtable}

\subsection{Erzeugte Visualisierungen öffnen}
\label{sec:interface_openPDF}
\begin{longtable}{|p{\textwidth}|}
\hline
\rowcolor{lightblue}TurboOpenPDF\\
\hline
\\
\texttt{TurboOpenPDF(encode, punctured, simulation)}\\
\\
Öffnet die mit \emph{TurboEncode, TurboDecode, TurboSimulation} bereits erzeugten PDF-Berichte. Die Dateien liegen im Paket-Ordner im Installationsverzeichnis von R.\\
\\
\textbf{Argumente:}\\
\texttt{encode} - Wenn TRUE werden PDFs vom Kodierungs-, bei FALSE vom Dekodierungsverfahren geöffnet. Standard: TRUE\\
\texttt{punctured} - Wenn TRUE werden PDFs mit Punktierungsverfahren geöffnet. Standard: FALSE\\
\texttt{simulation} - Wenn TRUE öffnet sich der Bericht der Simulation. Standard: FALSE\\
\\
\hline
\caption{TurboOpenPDF - Funktionserklärung}
\end{longtable}

\section{Kanalkodierungsfunktionen im Paket}

\subsection{Rauschen hinzufügen}
\label{sec:interface_applyNoise}
\begin{longtable}{|p{\textwidth}|}
\hline
\rowcolor{lightblue}
ApplyNoise\\
\hline
\\
\texttt{ApplyNoise(msg, SNR.db, binary)}\\
\\
Verrauscht ein Signals basierend auf das AWGN (additive white gaussian noise) Modell. Das ist das Standardmodell für die Simulation eines Übertragungskanals.\\
\\
\textbf{Argumente:}\\
\texttt{msg} - Nachricht die verrauscht wird.\\
\texttt{SNR.db} - Signal/Rausch-Verhältnis des Übertragungskanals. Standard: 3.0\\
\texttt{binary} - Blockkode-Parameter. Nicht zu verwenden! Standard: FALSE\\
\\
\textbf{Rückgabewert:}\\
Verrauschtes Signal.\\
\\
\hline
\caption{ApplyNoise - Funktionserklärung}
\end{longtable}

\subsection{Simulation und Vergleich von Block-, Faltungs- und Turbo-Kodes}
\label{sec:interface_channelcodingSimulation}
% ChannelcodingSimulation.tex
\begin{longtable}{|p{\textwidth}|}
\hline
\rowcolor{lightblue}
ChannelcodingSimulation
\\
\hline
\\
\texttt{ChannelcodingSimulation(msg.length, min.db, max.db, db.interval, iterations.per.db, turbo.decode.iterations, visualize)}\\
\\
Simulation von Block-, Faltungs- und Turbo-Kodes und Vergleich ihrer Bitfehlerraten bei unterschiedlichen SNR.\\
\\
\textbf{Argumente:}\\
\texttt{msg.length} - Nachrichtenlänge der zufällig generierten Nachrichten. Standard: 100\\
\texttt{min.db} - Untergrenze der getesteten SNR. Standard: 0.1\\
\texttt{max.db} - Obergrenze der getesteten SNR. Standard: 2.0\\
\texttt{db.interval} - Schrittweite zwischen zwei getesteten SNR. Standard: 0.1\\
\texttt{iterations.per.db} - Anzahl der Iterationen (Kodieren und Dekodieren) je SNR. Standard: 100\\
\texttt{turbo.decode.iterations} - Anzahl der Iterationen bei der Turbo-Dekodierung. Standard: 5\\
\texttt{visualize} - Wenn \texttt{TRUE} wird ein PDF-Bericht erstellt. Standard: \texttt{FALSE}\\
\\
\textbf{Rückgabewert:}\\
Dataframe das alle Simulationsergebnisse der 3 Kodierungsverfahren beinhaltet.\\
\\
\hline
\caption{ChannelcodingSimulation Funktion}
\end{longtable}

\subsection{Darstellen verschiedener Simulationen}
\label{sec:interface_plotSimulationData}
% PlotSimulationData.tex
\begin{longtable}{|p{\textwidth}|}
\hline
\rowcolor{lightblue}PlotSimulationData\\
\hline
\\
\texttt{PlotSimulationData(\dots)}\\
\\
Stellt mehrere mitgegebene Dataframes in einem Diagramm dar. Damit kann man verschiedene Kanalkodierungsverfahren miteinander vergleichen.\\
\\
\textbf{Argumente:}\\
\texttt{\dots} - Dataframes die mit den Simulationsfunktionen erzeugt wurden.\\	
\\
\hline
\caption{PlotSimulationData - Funktionserklärung}
\end{longtable}