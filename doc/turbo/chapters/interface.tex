\definecolor{lightblue}{RGB}{150,180,255}
Dieses Kapitel soll als zusätzliche Hilfe (Nachschlagewerk) bei der Verwendung der R-Funktionen dienen, da hier die genauen Schnittstellen der Funktionen erklärt sind. Nachdem das Paket installiert ist, steht die Hilfe auch im RStudio zur Verfügung, diese ist allerdings auf Englisch und nicht so umfangreich. Falls aus diesem Kapitel die Verwendung noch nicht komplett klar ist, kann man sich auch die Beispiele in Kapitel~\ref{cha:examples} anschauen. Für eine genaue Erklärung der Funktionen kann in Kapitel~\ref{cha:implementation} die genaue Implementierung nachgelesen werden und daraus die weitere Verwendung geschlossen werden.

In Kapitel~\ref{sec:interface_turboFunctions} sind alle Funktionen gelistet, die für das Kodieren und Dekodieren notwendig sind. Ein paar Hilfsfunktionen sind in Kapitel~\ref{sec:interface_helperFunctions} nachzuschlagen und die allgemeinen Funktionen, die für alle drei Verfahren notwendig sind, können im Kapitel~\ref{sec:interface_channelFunctions} nachgelesen werden. 
\section{Turbo-Kode Funktionen}
\label{sec:interface_turboFunctions}

\subsection{Kodieren}
\label{sec:interface_encode}
\begin{longtable}{|p{\textwidth}|}
\hline
\rowcolor{lightblue}TurboEncode\\
\hline
\\
\texttt{TurboEncode(message, permutation.vector, coder.info, parity.index, punctuation.matrix, visualize)}\\
\\
Kodiert eine Nachricht mittels dem Turbo-Kode-Verfahren. Dabei wird die Nachricht in zwei systematische Kodierer geschickt und am Ausgang wird das Ergebnis der beiden Kodierer mit der Originalnachricht zusammengesetzt. Falls eine Punktierungsmatrix mitgegeben wird, erfolgt am Ende die Punktierung und somit die Erhöhung der Koderate.\\
\\
\textbf{Argumente:}\\
\texttt{message} - Nachricht die kodiert wird. (0er und 1er)\\
\texttt{permutation.vector} - Permutationsvektor der für den Interleaver verwendet wird. Standard: \texttt{PRIMITIVE (root=0)}\\
\texttt{coder.info} - Faltungskodierer der mit den Funktionen \texttt{ConvGenerateEncoder, ConvGenerateRscEncoder} erzeugt werden kann. Muss ein systematischer Kodierer sein, also ein RSC oder Ausgang 1 muss durchgeschalten sein. Standard: \texttt{ConvGenerateRscEncoder(2,2,c(5,7))}\\
\texttt{parity.index} - Gibt an welcher Ausgang vom mitgegebenen Kodierer verwendet wird. Standard: letzter Ausgang\\
\texttt{punctuation.matrix} - Die Matrix wird am Ende des Kodiervorgangs zur Punktierung verwendet. Standard: keine Punktierung\\
\texttt{visualize} - Wenn \texttt{TRUE} wird PDF-Bericht erstellt. Standard: \texttt{FALSE}\\
\\
\textbf{Rückgabewert:}\\
Die kodierte Nachricht mit den Signalwerten -1 und 1, welche die Bits 1 und 0 darstellen. Falls punktiert wurde, wird eine Liste mit dem Originalkode und dem punktierten Kode zurückgegeben.\\
\\
\hline
\caption[TurboEncode]{TurboEncode - Funktionserklärung}
\end{longtable}

\subsection{Dekodieren}
\label{sec:interface_decode}
\begin{longtable}{|p{\textwidth}|}
\hline
\rowcolor{lightblue}TurboDecode\\
\hline
\\
\texttt{TurboDecode(code, permutation.vector, iterations, coder.info, parity.index, punctuation.matrix, visualize)}\\
\\
Dekodiert eine Nachricht mittels dem Turbo-Kode-Verfahren. Dabei werden zuerst die Punktierungsbits wieder eingefügt, falls Punktierung verwendet wurde. Danach erfolgt die Dekodierung, wobei die Ergebnisse der Dekodierer als Eingang des Nächsten verwendet werden. Dieser Vorgang wird mehrmals iterativ ausgeführt, je nach Anzahl der gewählten Iterationen.\\
\\
\textbf{Argumente:}\\
\texttt{code} - Nachricht die dekodiert wird. Der genaue Signalpegel wird verwendet.\\
\texttt{permutation.vector} - Permutationsvektor der für den Interleaver verwendet wird. Standard: \texttt{PRIMITIVE (root=0)}\\
\texttt{iterations} - Anzahl der Dekodierungsiterationen. Standard: 1\\
\texttt{coder.info} - Faltungskodierer der bei der Kodierung bereits verwendet wurde. Standard: \texttt{ConvGenerateRscEncoder(2,2,c(5,7))}\\
\texttt{parity.index} - Gibt an welcher Ausgang vom mitgegebenen Kodierer verwendet wird. Standard: letzter Ausgang\\
\texttt{punctuation.matrix} - Die mitgegebene Punktierungsmatrix wird am Anfang des Dekodiervorgangs,  für das Einfügen der jeweiligen Bits verwendet. Standard: keine Punktierung\\
\texttt{visualize} - Wenn \texttt{TRUE} wird PDF-Bericht erstellt. Standard: \texttt{FALSE}\\
\\
\textbf{Rückgabewert:}\\
Liste welche die dekodierte Nachricht und die genau berechneten Soft-Werte enthält.\\
\\
\hline
\caption[TurboDecode]{TurboDecode - Funktionserklärung}
\end{longtable}

\subsection{Simulation}
\label{sec:interface_simulation}
\begin{longtable}{|p{\textwidth}|}
\hline
\rowcolor{lightblue}TurboSimulation\\
\hline
\\
\texttt{TurboSimulation(coder, permutation.type, permutation.args, decode.iterations, msg.length, min.db, max.db, db.interval, iterations.per.db, punctuation.matrix, visualize)}\\
\\
Automatische Simulation eines Kodierungs- und Dekodierungsverfahrens von Turbo-Kodes. Nach dem Kodieren wird der resultierende Kode verrauscht und im Anschluss dekodiert. Für das jeweilige Signal/Rausch-Verhältnis wird dieses Verfahren mehrmals (\texttt{iterations.per.db}) wiederholt und am Ende ein Durchschnitt der Bitfehlerrate berechnet.\\
\\
\textbf{Argumente:}\\
\texttt{coder} - Kodierer der für Kodierung und Dekodierung verwendet wird. Kann mittels den Funktionen \texttt{ConvGenerateEncoder, ConvGenerateRscEncoder} erzeugt werden. Standard: \texttt{ConvGenerateRscEncoder(2,2,c(5,7))}\\
\texttt{permutation.type} - Typ des Permutationsvektors. Standard: \texttt{PRIMITIVE}\\
\texttt{permutation.args} - Argumente für die Erzeugung des Permutationsvektors. Standard: \texttt{list(root=0)}\\
\texttt{decode.iterations} - Anzahl von Iterationen bei der Dekodierung. Standard: 5\\
\texttt{msg.length} - Länge der Nachricht. Standard: 100\\
\texttt{min.db} - Untergrenze des Signal/Rausch-Verhältnisses. Standard: 0.1\\
\texttt{max.db} - Obergrenze des Signal/Rausch-Verhältnisses. Standard: 2.0\\
\texttt{db.interval} - Schrittweite pro Erhöhung des Signal/Rausch-Verhältnisses. Standard: 0.1\\
\texttt{iterations.per.db} - Iterationen pro Signal/Rausch-Verhältnis zur Durchschnittsbildung. Standard: 100\\
\texttt{punctuation.matrix} - Verwendete Punktierungsmatrix. Standard: wird nicht punktiert\\
\texttt{visualize} - Wenn \texttt{TRUE} wird PDF-Bericht erstellt. Standard: \texttt{FALSE}\\
\\
\textbf{Rückgabewert:}\\
Dataframe das für jedes Signal/Rausch-Verhältnis eine Bitfehlerrate beinhaltet.\\
\\
\hline
\caption[TurboSimulation]{TurboSimulation - Funktionserklärung}
\end{longtable}

\section{Hilfsfunktionen}
\label{sec:interface_helperFunctions}

\subsection{Permutationvektor erzeugen}
\label{sec:interface_permutation}
\begin{longtable}{|p{\textwidth}|}
\hline
\rowcolor{lightblue}TurboGetPermutation\\
\hline
\\
\texttt{TurboGetPermutation(message.length, coder.info, type, args, visualize)}\\
\\
Erzeugt einen Permutationsvektor für die Interleaver beim Turbo-Kode-Verfahren. Dabei können verschiedene Typen von Permutationen ausgewählt werden (RANDOM, PRIMITIVE, CYCLIC, BLOCK, HELICAL, DIAGONAL). Die genauere Erklärung des jeweiligen Typs findet sich im Kapitel \ref{cha:implementation}. Den jeweiligen Typen müssen spezielle Argumente in einer Liste mitgegebenen werden:
\begin{itemize}
\item RANDOM: Benötigt keine Argumente, wird komplett zufällig erstellt.
\item PRIMITVE: Erzeugt einen Vektor (1,2,3...) und schiebt diesen um den \emph{root}-Wert.
\item CYCLIC: Benötigt die Argumente \emph{cols} und \emph{rows} für die Erzeugung der Matrix und \emph{distance} um den Permutationsvektor zu erzeugen.
\item BLOCK: Benötigt die Argumente \emph{cols} und \emph{rows} für die Erzeugung der Matrix. Matrix wird von links nach rechts befüllt und von oben nach unten ausgelesen.
\item HELICAL: Benötigt die Argumente \emph{cols} und \emph{rows} für die Erzeugung der Matrix. Matrix wird von links nach rechts befüllt und von links oben nach rechts unten ausgelesen.
\item DIAGONAL: Benötigt die Argumente \emph{cols} und \emph{rows} für die Erzeugung der Matrix. Matrix wird von links nach rechts befüllt und dann diagonal ausgelesen.
\end{itemize} \\
\\
\textbf{Argumente:}\\
\texttt{message.length} - Länge der Nachricht die kodiert werden möchte.\\
\texttt{coder.info} - Faltungskodierer der mit den Funktionen \emph{ConvGenerateEncoder, ConvGenerateRscEncoder} erzeugt werden kann. Standard: \emph{ConvGenerateRscEncoder(2,2,c(5,7))}\\
\texttt{type} - Typ des erzeugten Permutationsvektors. Standard: RANDOM\\
\texttt{args} - Argumente für den jeweiligen Typ in einer Liste. Standard: NULL\\
\texttt{visualize} - Wenn TRUE wird die Permutationsmatrix dargestellt. Standard: FALSE\\
\\
\textbf{Rückgabewert:}\\
Permutationsvektor mit der richtigen Länge für die mitgegebene Nachricht und Kodierer.\\
\\
\hline
\caption{TurboGetPermutation - Funktionserklärung}
\end{longtable}

\subsection{Punktierungsmatrix erzeugen}
\label{sec:interface_punctuation}
\begin{longtable}{|p{\textwidth}|}
\hline
\rowcolor{lightblue}TurboGetPunctuationMatrix\\
\hline
\\
\texttt{TurboGetPunctuationMatrix(punctuation.vector, visualize)}\\
\\
Erzeugt eine Punktierungsmatrix von einem mitgegebenen Vektor für das Turbo-Kode-Verfahren (drei Zeilen). Es dürfen keine Null-Spalten entstehen!\\
\\
\textbf{Argumente:}\\
\texttt{punctuation.vector} - Vektor der in eine Punktierumatrix transformiert wird. Eine 1 behaltet das Bit, eine 0 verwirft das Bit.\\
\texttt{visualize} - Wenn \texttt{TRUE} wird Punktierungsmatrix dargestellt. Standard: \texttt{FALSE}\\
\\
\textbf{Rückgabewert:}\\
Punktierungsmatrix die für das Turbo-Kode-Verfahren geeignet ist.\\
\\
\hline
\caption[TurboGetPunctuationMatrix]{TurboGetPunctuationMatrix - Funktionserklärung}
\end{longtable}

\subsection{Erzeugte Visualisierungen öffnen}
\label{sec:interface_openPDF}
\begin{longtable}{|p{\textwidth}|}
\hline
\rowcolor{lightblue}TurboOpenPDF\\
\hline
\\
\texttt{TurboOpenPDF(encode, punctured, simulation)}\\
\\
Öffnet die mit \emph{TurboEncode, TurboDecode, TurboSimulation} bereits erzeugten PDF-Berichte. Die Dateien liegen im Paket-Ordner im Installationsverzeichnis von R.\\
\\
\textbf{Argumente:}\\
\texttt{encode} - Wenn TRUE werden PDFs vom Kodierungs-, bei FALSE vom Dekodierungsverfahren geöffnet. Standard: TRUE\\
\texttt{punctured} - Wenn TRUE werden PDFs mit Punktierungsverfahren geöffnet. Standard: FALSE\\
\texttt{simulation} - Wenn TRUE öffnet sich der Bericht der Simulation. Standard: FALSE\\
\\
\hline
\caption{TurboOpenPDF - Funktionserklärung}
\end{longtable}

\section{Kanalkodierungsfunktionen im Paket}
\label{sec:interface_channelFunctions}

\subsection{Rauschen hinzufügen}
\label{sec:interface_applyNoise}
\begin{longtable}{|p{\textwidth}|}
\hline
\rowcolor{lightblue}
ApplyNoise\\
\hline
\\
\texttt{ApplyNoise(msg, SNR.db, binary)}\\
\\
Verrauscht ein Signal basierend auf das AWGN (additive white gaussian noise) Modell. Das ist das Standardmodell für die Simulation eines Übertragungskanals.\\
\\
\textbf{Argumente:}\\
\texttt{msg} - Nachricht die verrauscht wird.\\
\texttt{SNR.db} - Signal/Rausch-Verhältnis des Übertragungskanals. Standard: 3.0\\
\texttt{binary} - Blockkode-Parameter. Nicht zu verwenden! Standard: FALSE\\
\\
\textbf{Rückgabewert:}\\
Verrauschtes Signal.\\
\\
\hline
\caption{ApplyNoise - Funktionserklärung}
\label{func:applynoise}
\end{longtable}

\subsection{Simulation und Vergleich von Block-, Faltungs- und Turbo-Kodes}
\label{sec:interface_channelcodingSimulation}
% ChannelcodingSimulation.tex
\begin{longtable}{|p{\textwidth}|}
\hline
\rowcolor{lightblue}ChannelcodingSimulation\\
\hline
\\
\texttt{ChannelcodingSimulation(msg.length, min.db, max.db, db.interval, iterations.per.db, turbo.decode.iterations, visualize)}\\
\\
Simulation of channelcoding techniques (blockcodes, convolutional codes and turbo codes) and comparison of their bit-error-rates.\\
\\
\textbf{Arguments:}\\
\texttt{msg.length} - Message length of the randomly created messages to be encoded. Default: 100\\
\texttt{min.db} - Minimum SNR to be tested. Default: 0.1\\
\texttt{max.db} - Maximum SNR to be tested. Default: 2.0\\
\texttt{db.interval} - Step between two SNRs tested. Default: 0.1\\
\texttt{iterations.per.db} - Number of encode and decode processes per SNR. Default: 100\\
\texttt{turbo.decode.iterations} - Number of decoding iterations inside the turbo decoder. Default: 5\\
\texttt{visualize} - If true a PDF report is generated. Default: FALSE\\
\\
\textbf{Returns:}\\
Distorted message containing noise.\\
\\
\hline
\end{longtable}

\subsection{Darstellen verschiedener Simulationen}
\label{sec:interface_plotSimulationData}
% PlotSimulationData.tex
\begin{longtable}{|p{\textwidth}|}
\hline
\rowcolor{lightblue}PlotSimulationData\\
\hline
\\
\texttt{PlotSimulationData(\dots)}\\
\\
Stellt mehrere mitgegebene Dataframes in einem Diagramm dar. Damit kann man verschiedene Kanalkodierungsverfahren miteinander vergleichen.\\
\\
\textbf{Argumente:}\\
\texttt{\dots} - Dataframes die mit den Simulationsfunktionen erzeugt wurden.\\	
\\
\hline
\caption[PlotSimulationData]{PlotSimulationData - Funktionserklärung}
\end{longtable}