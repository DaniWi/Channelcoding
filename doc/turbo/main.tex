\documentclass[germanthesis]{thesis-style}
% options:
% [germanthesis] - Thesis is written in German
% [plainunnumbered] - Don't print numbers on plain pages
% [earlydraft] - Settings for quick draft printouts
% [watermark] - Print current time/date at bottom of each page
% [phdthesis] - switch to PhD thesis style
% [twoside] - double sided
% [cutmargins] - text body fills complete page

\bibliography{references}

% additional packages
\usepackage{color, colortbl} 
\usepackage{longtable}
\usepackage{placeins}
\usepackage{caption}

\newtheorem{t_def}{Definition}[chapter]
\newtheorem{e_exa}{Beispiel}[chapter]

% change name of caption by longtables
\captionsetup[longtable]{name=Funktion}

% set listing language to R
\lstset{language=R, prebreak={}}

\captionsetup{width=0.8\textwidth}

% makro for resizing pictures
\makeatletter
\def\ScaleIfNeeded{%
\ifdim\Gin@nat@width>\linewidth
\linewidth
\else
\Gin@nat@width
\fi
}
\makeatother

\author{Daniel~Witsch}
\title{R-Paket für Kanalkodierung mit Turbo-Kodes}
%\title{R~package for channel~coding with convolutional~codes}
\birthday{28. Dezember 1992}
\birthplace{Zams}
%\thesisstart{1. Januar 2009}
\thesistype{Bachelor's thesis}
\thesistypegerman{Bachelorarbeit}
\thesiscite{Bachelor's thesis}
\advisors{Univ.-Prof.~Dr.~Rainer~Böhme,Dr.~Pascal~Schöttle}
\date{\today}

\begin{document}

% Titelseiten und Eidesstattliche Erklärung
\maketitle
% Deutsche Zusammenfassung
\abstract{Diese Bachelorarbeit implementiert das Turbo-Kode-Verfahren, verpackt in einem R-Paket. Diese Kodierung ist ein Teilgebiet der Kanalkodierung, die bei heutigen Kommunikationskanälen nicht mehr wegzudenken ist, da auftretende Fehler während der Übertragung der Nachricht, dadurch wieder rekonstruierbar sind. Hauptanwendungsgebiet sind dabei die Mobil- und Satellitenkommunikation, da hierbei wichtig ist, dass eine Nachricht nicht mehrmals übertragen werden muss, bis sie korrekt ankommt. Um für zukünftige Studierende das Erlernen dieser Kommunikationstechnik zu erleichtern, sollten geeignete Visualisierungen geschaffen werden. Diese Arbeit steht in enger Verbindung mit den beiden Bachelorarbeiten, die Blockkodierung und Faltungskodierung behandeln. Alle drei Kodierungsverfahren sollten in das selbe R-Paket inkludiert werden.
}
% Literaturverzeichnis
\tableofcontents
\clearpage
\pagenumbering{arabic}

\chapter{Einleitung und Motivation}
\label{cha:introduction}
Aufgrund der steigenden Anzahl von mobilen Anwendungsgeräten ist die Mobilkommunikation ein wichtiger Teilbereich der Nachrichtentechnik. Zur Datenübertragung werden elektromagnetische Funkwellen benutzt, die aufgrund der äußeren Umwelteinflüsse störanfälliger als kabelgebundene Übertragungskanäle sind. Deswegen benötigt es eine Technik, welche es dem Empfänger ermöglicht, die Fehler die aus den Störungen resultieren, beheben zu können. Dabei kommen die Verfahren der Kanalkodierung zum Einsatz. Diese ermöglichen es vor dem Absenden der Daten, zusätzliche Informationen zur Nachricht hinzuzufügen, um beim Empfang die entstandenen Fehler zu korrigieren.

Ein fortgeschrittenes Verfahren bei der Kanalkodierung sind die Turbo-Kodes. Diese kombinieren die bereits bekannten Kodierungen, wodurch ein Gesamtkode entsteht, dessen Leistungsfähigkeit nahe der maximalen theoretischen Übertragungsgrenze liegt. Aufgrund der langen Signallaufzeiten in der Mobil- und Satellitenkommunikation werden Turbo-Kodes eingesetzt, da bei einer fehlerhaften Übertragung ein erneutes Senden nicht praktikabel wäre. Deswegen kann aus den Informationen, die vor dem Versand hinzugefügt wurden, die Originalnachricht trotz Störungen wiederhergestellt werden. Schon seit den 1980er Jahren wird diese Art der Kodierung in Bereichen der Speichersysteme, wie CD oder DVD, eingesetzt. Für den heutigen Erfolg der Mobilfunkstandards LTE und UMTS ist dieses Verfahren ausschlaggebend, um die Übertragungsgeschwindigkeiten zu erreichen. Auch die bekannte ESA Raumsonde Rosetta, die seit August 2014 einen Kometen umkreist, benutzt diese Technik für die Kommunikation mit dem Kontrollzentrum auf der Erde.~\cite[S.~242~f.]{schoenfeld2012informations} 

Ziel dieser Arbeit war es, das Prinzip des Turbo-Kode-Verfahrens, mit der Programmiersprache R, in ein wiederverwendbares R-Paket zu implementieren. Für zukünftige Studierende sollte eine visuelle Darstellung geschaffen werden, die den prinzipiellen Ablauf der Kodierung und Dekodierung darstellt. Durch die praktische Anwendung und grafische Visualisierung soll das Verständnis von Turbo-Kodes erleichtert werden. Somit ist das R-Paket eine optimale Ergänzung zum theoretischen Unterricht an den Universitäten.

Die beiden weiteren Bachelorarbeiten \enquote{Kanalkodierung mit Blockkodes} von Benedikt Wimmer \cite{wimmer} und \enquote{Kanalkodierung mit Faltungskodes} von Martin Nocker \cite{nocker} vervollständigen das R-Paket, um die gesamte Funktionalität der Kanalkodierung zu integrieren. Dadurch erhalten die Studierenden ein kompaktes und einfach verwendbares R-Paket, welches alle wichtigen Bestandteile des Kanalkodierungsverfahren beinhaltet und eine hilfreiche visuelle Darstellung anbietet.   

Diese Bachelorarbeit startet in Kapitel~\ref{cha:basics} mit den Grundlagen der Kanalkodierung. Diese werden benötigt, um anschließende Verfahren besser verstehen zu können. Danach werden in Kapitel~\ref{cha:technologies} kurz die verwendeten Technologien und deren Vorteile erläutert. Die genaue Erklärung der Implementation des Paketes, wird in Kapitel~\ref{cha:implementation} veranschaulicht. Zur Verwendung der Funktionen finden sich in Kapitel~\ref{cha:interface} die Schnittstelle des Paketes. Um eine Hilfe bei dem Verständnis der Visualisierungen zu erhalten, werden in Kapitel~\ref{cha:visualization} alle wichtigen Darstellungen im Detail erklärt. Der Einsteig zur Verwendung des R-Paketes soll mit den Beispielen von Kapitel~\ref{cha:examples} erleichtert werden. Einige mögliche Erweiterungen werden zum Abschluss noch in Kapitel~\ref{cha:result} vorgebracht.

\chapter{Grundlagen}
\label{cha:basics}
In diesm Kapitel werden alle wichtigen theoretischen Konzepte von Kanalkodierung und im speziellen zu Turbo-Kodes erklärt. Damit ist es möglich, die Implementation in C- beziehungsweise R-Code, welche in Kapitel~\ref{cha:implementation} erläutert wird, zu verstehen. Die Notation und der Inhalt orientieren sich an dem Buch von Schönfeld \cite{schoenfeld2012informations}.

Zuerst werden in Kapitel~\ref{sec:channelcoding} alle Grundlagen zur Kanalkodierung erläutert, welche dann bei der genauen Erklärung der Turbo-Kodes in Kapitel~\ref{sec:turboCodes} gebraucht werden.

\section{Kanalkodierung}
\label{sec:channelcoding}

\begin{figure}[th]
\centering
\includegraphics[width=\ScaleIfNeeded]{pictures/Channelmodel}
\caption{Modell der Nachrichtenübertragung,~Quelle:~\cite[10]{schoenfeld2012informations}}
\label{pic:channelmodel}
\end{figure}

In der Abbildung~\ref{pic:channelmodel} ist ein generelles Modell der Nachrichtenübertragung dargestellt, welches den Weg einer Nachricht zeigt. Nachdem der Sender (Quelle) die Nachricht durch den Quellenkodierer und Kanalkodierer geschickt hat, gelangt das Kanalwort auf den Übertragungskanal. Dort stören äußere Einflüsse das Signale und verändern es. Beim Empfänger (Senke) wird versucht, die Nachricht wieder zu dekodieren, um die Originalnachricht zu erhalten.

Das Ziel der Kanalkodierung dabei ist, den Quellenkodewörtern Redundanz hinzuzufügen, um nach einer fehlerhaften Übertragung über das Übertragungsmedium die Ausgangsnachricht wieder rekonstruieren zu können. Dabei können Fehler erkannt und im bestem Fall sogar komplett korrigiert werden.

Kanalkodierungen lassen sich in drei Kategorien unterteilen:

\begin{itemize}
\item Blockkodes
\item Faltungskodes
\item Turbo-Kodes
\end{itemize} 

Bei der ersten genannten Kategorie wird immer eine Nachricht einer bestimmten Länge kodiert. Bei Faltungskodes wird hingegen der ganzen Nachricht, unabhängig von der Länge, kontinuierlich Redundanz hinzugefügt. Turbo-Kodes andererseits verwenden bereits existierende Kodierer, wenden diese aber mehrfach an und vereinen das Ergebnis der Einzelnen.

\subsection{Zweites SHANNONsches Kodierungstheorem}
\label{sec:shannonTheorem}
Auf dem Übertragungskanal wird die zu übertragende Information zu einer bestimmen Wahrscheinlichkeit verfälscht. Durch die Kanalkodierung können Fehler bis zu einem bestimmten Grad korrigiert werden, jedoch bleiben Fehler mit einer bestimmten Restwahrscheinlichkeit.

SHANNON\footnote{Wurde nach Claude Elwood Shannon und Ralph Hartley benannt.} hat mit seinem zweitem Theorem bewiesen, dass es theoretisch möglich ist, eine Kanalkodierung zu finden, welche die Restfehlerwahrscheinlichkeit beliebig klein halten kann. Allerdings nur unter der Bedingung, dass der Sender weniger Kodewörter erzeugt, als der Kanaldekodierer beim Empfänger verarbeiten kann. Bereits 1948 erkannte SHANNON eine Grenze, die sogenannte SHANNON-Grenze, welche eine Grenze des Signal/Rausch-Verhältnisses, im Bezug auf die hinzugefügte Redundanz und der erreichbaren Restfehlerwahrscheinlichkeit darstellt. Somit wurde das Ziel gesteckt, mit einer geeigneten Kodierung, dieser Grenze möglichst nahe zu kommen. \cite[S.~125~f.]{schoenfeld2012informations}

Ein großer Schritt in Richtung dieser SHANNON-Grenze war die Vorstellung der Turbo-Kodes im Jahre 1993. Mit diesen war es erstmals möglich, dieser Grenze sehr nahe zu kommen. Vorgestellt wurde diese Art der Dekodierung in einem \emph{IEEE}-Artikel von Claude Berrou \cite{berrou1996near}.

\subsection{Prinzipien der Fehlerkorrektur}
\label{sec:principlesMistakesCorrection}
Das Anfügen von Redundanz kann auf ganz unterschiedliche Weisen erfolgen. Grundsätzlich gibt es 2 Möglichkeiten diese dem Quellenkodewort hinzuzufügen:

\begin{itemize}
\item Wiederholung der Nachricht
\item Rekonstruktion der Nachricht
\end{itemize}

Bei der ersten Möglichkeit fordert der Empfänger die Nachricht erneut an, wenn er bemerkt, dass die Nachricht nicht korrekt übertragen wurde. Er kann den Fehler durch Kontrollstellen in der Nachricht, zum Beispiel Paritätsbits, erkennen. Bei der zweiten Möglichkeit dient die hinzugefügte Redundanz nicht nur der Erkennung der Fehler, sondern auch als Hilfe zur Rekonstruktion der Originalnachricht. Natürlich hier muss die Redundanz größer sein, als bei der ersten Methode, die nur Fehler erkennt.

Es wird vorausgesetzt, dass die Wahrscheinlichkeit von Einzelfehlern größer ist, als das Auftreten von Bündelfehlern, also mehrere Fehler hintereinander. Jede Kodierung hat seine Grenzen, deswegen kann auch der beste Dekodierer nicht unendlich viele Fehler korrigieren. Zur Rekonstruktion des Kodewortes wird die \emph{Maximum-Likelihood-Methode} verwendet. Dabei wird das Kodewort gesucht, das einem existierenden, realistischen Wort am nähesten liegt. Dadurch liefert der Dekodierer immer ein Kodewort, welches die geringste Distanz zum gesendeten Kodewort aufweist. Das bedeutet jedoch nicht, dass dies unbedingt die richtige Nachricht sein muss. Jedoch steigt der Berechnungsaufwand dieser Methode mit der Länger der Nachricht enorm an.~\cite[126-129]{schoenfeld2012informations} 

\subsection{Verwendete Notation}
\label{sec:notation}
In den folgenden Kapiteln werden immer wieder Variablen verwendet, welche hier eingeführt und anschließend immer benützt werden, ohne sie bei der konkreten Formel nochmals anzuführen. Bei diesen Definition werden immer Kodewörter mit dem Alphabet $\{0,1\}$ verwendet, da dieser Fall am meisten Bedeutung hat.

Nachrichten werden zuerst mit dem Quellenkodierer in eine möglichst redundanzfreie Form gebracht, um die verwendeten Bits möglichst gut auszunutzen. Diese Kodewörter werden als Quellenkodewörter bezeichnet und sind folgenderweise definiert:

\begin{t_def}
Ein Wort $a^* \in \{0,1\}^l$ wird als Quellenkodewort der Länge $l$ bezeichnet.
\end{t_def}

Im Anschluss wird durch den Kanalkodierer Redundanz den Quellenkodewörtern hinzugefügt. Das resultierende Kodewort sieht folgendermaßen aus:

\begin{t_def}
Ein Wort $a \in \{0,1\}^n$ wird als Kanalkodewort der Länge $n$ bezeichnet.
\end{t_def} 

Durch das Hinzufügen von Redundanz ergeben sich $k = n - l$ redundante Stellen. Diese werden zur Fehlererkennung und Fehlerkorrektur bei der Dekodierung verwendet.

\begin{e_exa}
Gegeben sei eine Nachricht vom Quellenkodierer der Länge $l=2$, $a^*_{1}=(00),a^*_{2}=(01),a^*_{3}=(10),a^*_{4}=(11)$. Danach fügt der Kanalkodierer eine redundante Stelle $k=1$ hinzu. Somit haben die Nachrichten dann eine Länge von $n=3$. Sie könnten folgenderweise aussehen, $a_{1}=(001),a_{2}=(010),a_{3}=(100),a_{4}=(110)$.
\end{e_exa}

\subsection{Kenngrößen von Kanalkodes}
\label{sec:channelParameters}
Für die Betrachtung von verschiedenen Kanalkodierungen ist es wichtig, Unterscheidungsmerkmale zu finden. Dazu wird als erstes in Kapitel~\ref{sec:hammingDistance} die Distanz zwischen zwei Kodewörtern erklärt, um im Anschluss in Kapitel~\ref{sec:hammingWeight} das Gewicht eines Wortes berechnen zu können. Zuletzt wird in Kapitel~\ref{sec:codeRate} wird noch der Begriff der Koderate eingeführt.

\subsubsection{Hamming-Distanz}
\label{sec:hammingDistance}
Damit durch einzelne Fehler auf dem Übertragungskanal nicht eine andere existierende Nachricht produziert wird, sollen Kodewörter angestrebt werden, die sich möglichst weit voneinander unterscheiden. Eine wichtige Kenngröße dabei ist der Abstand zwischen zwei Kodewörtern.

\begin{t_def}
Die Anzahl der Stellen, in denen sich 2 Kodewörter $a_i$ und $a_j$ unterscheiden, bezeichnet man als \emph{Hamming-Distanz} zwischen den beiden Wörtern.
\end{t_def} 
 
\begin{equation}
d(a_i,a_j) = \sum^{n}_{g=1} (a_{ig} \oplus a_{jg})
\label{eq:hammingDistance}
\end{equation}

Diese lässt sich bei Binärkodes mit der Formel~\ref{eq:hammingDistance} berechnen. Dabei wird einfach gezählt, wie viele Stellen sich unterscheiden. Interessant ist der kleinste Unterschied zwischen zwei Kodewörtern, diese Zahl wird \emph{minimale Hamming-Distanz} $d_{min}$ genannt.

Ein Kode der alle Verfälschungen $\leq f_e$ erkennen kann, muss eine \emph{minimale Hamming-Distanz} von 

\begin{equation}
d_{min} = f_e + 1
\end{equation}

besitzen. Mit $f_e$ lässt sich der Grad der Fehlererkennung beschreiben. Die Anzahl der korrigierbaren Fehler $f_k$ lässt sich folgendermaßen berechnen:

\begin{equation}
f_k = \frac{d_{min}-1}{2}
\end{equation}

Damit nun Fehler erkannt und korrigiert werden können, muss die \emph{minimale Hamming-Distanz} mindestens folgende Gleichung erfüllen:~\cite[S.~132~f.]{schoenfeld2012informations}

\begin{equation}
d_{min} = f_e + f_k + 1
\end{equation} 

\subsubsection{Hamming-Gewicht}
\label{sec:hammingWeight}
Ein weiterer Zusammenhang zwischen der \emph{Hamming-Distanz} und den zu korrigierende Fehler, beschreibt das \emph{Hamming-Gewicht}.

\begin{t_def}
Anzahl der 1er in einem binären Kodewort $a=(a_1,a_2,\dotsc)$.
\begin{equation}
w(a) = \sum^n_{j=1} a_j
\label{eq:hammingWeight}
\end{equation} 
\end{t_def} 

Zur Berechnung dieses Gewichtes kann die Formel~\ref{eq:hammingWeight} verwendet werden. Solange 

\begin{equation}
d(a_i,b)=w(a_i \oplus b) = w(e) \leq f_k
\end{equation} 

kann das Kanalkodewort korrigiert werden. Dabei entspricht $b$ einem existierenden Kanalkodewort und $a_i$ der zu korrigierenden Nachricht.~\cite[134]{schoenfeld2012informations}
\subsubsection{Koderate}
\label{sec:codeRate}
Um ein Maß zu bekommen, wieviel Redundanz der Nachricht bei einem Kodierer hinzugefügt wird, wurde die Koderate eingeführt.

\begin{t_def}
Das Verhältnis der Längen zwischen Quellenkodewort $a^*=(a^*_1,a^*_2,\dotsc,a^*_l)$ und Kanalkodewort $a=(a_1,a_2,\dotsc,a_n)$ wird Koderate genannt.
\begin{equation}
R = \frac{l}{n} \leq 1
\end{equation} 
\end{t_def} 

Eine hohe Koderate zeigt, dass wenig Redundanz hinzugefügt wurde und somit weniger Fehler korrigiert werden können. Dagegen hat eine niedrige Koderate den Vorteil, dass sehr viele Fehler bei der Übertragung wiederherstellbar sind.~\cite[136]{schoenfeld2012informations} 

\subsection{Modell eines Übertragungskanal}
\label{sec:channels}
Um Nachrichten an den gewünschten Empfänger zu transportieren, können verschiedene Übertragungsmedien, wie zum Beispiel Kupferkabel oder Funkwellen, verwendet werden. Jedes dieser hat spezielle Eigenschaften und Charakteristiken. Damit diese Übertragungsmedien mit einem Computerprogramm simuliert werden können, muss ein Modell geschaffen werden, welches einen solchen Übertragungskanal bestmöglich abbildet. Dieses Modell bildet alle Störungen und Dämpfungen ab, die während der Übertragung auftreten können. Als ein gut geeignetes Modell stellte sich das Überlagern des Signals mit dem \emph{Additiv Weißes Gaußsches Rauschen} (AWGR) heraus \cite[81]{schoenfeld2012informations}. Dieses Rauschen wird wie folgt berechnet:

\begin{equation}
E_s = \frac{1}{L} \displaystyle\sum_{i=0}^{L-1} |x[i]|^2; \quad L = length(x)
\label{eq:power}
\end{equation}

In der Formel~\ref{eq:power} wird zuerst die Länge der Nachricht berechnet, um im Anschluss über alle Bits zu iterieren und die Summe der Quadrate des Signals zu berechnen. Diese Summe wird anschließend noch durch die Länge dividiert, um die Leistung je Bit zu erhalten.

\begin{equation}
noise = \sqrt{\frac{E_s}{10^{dB/10}}} * randn(1,L)
\label{eq:noise}
\end{equation}

Das Signal/Rausch-Verhältnis muss zuerst vom Dezibel-Bereich in den Linearen umgerechnet werden, wie im Nenner der Wurzel in der Gleichung~\ref{eq:noise} zu sehen ist.
Zur Berechnung des Rauschens werden $L$ Zufallszahlen zwischen 0 und 1 erzeugt und diese mit dem Wurzelergebnis multipliziert. 

\begin{equation}
y = x + noise
\label{eq:noisySignal}
\end{equation}

Danach wird einfach das Rauschen mit dem Originalsignal überlagert, um das verrauschte Signal zu erhalten. Dieses stellt nun den simulierten Übertragungskanal dar. Die Stärke der Störung kann mittels der $dB$-Variable beeinflusst werden.~\cite{AWGN}

\section{Turbo-Kodes}
\label{sec:turboCodes}
Mit der Einführung von Mobilfunk und Satellitenkommunikation wurden neue Kodes benötigt, die auch zufällig verteilte Einzelfehler und lange Bündelfehler korrigieren können. Eine Lösung dieses Problems sah man in der Verkettung mehrerer bereits existierender Kodes. Dabei werden die Vorteile der einzelnen Kodes genutzt, um einen noch leistungsfähigeren Gesamtkode zu erhalten, der mit Einbeziehung der Soft-Werte (also den realen Signalpegel beim Eingang des Dekodierers) einen Durchbruch in Richtung der SHANNON-Grenze darstellt. Bei der Dekodierung wird die Nachricht iterativ, also mehrmals durch die selben Dekodierer geschickt, um so eine Verbesserung der Ergebnisse zu erzielen.~\cite[S.~242~f.]{schoenfeld2012informations}

Die ersten Versuche wurden mit der seriellen Verkettung versucht, welche auch heute noch in der Anwendung sind. Beim klassischen Turbo-Kode werden jedoch parallel verkettete Faltungskodes verwendet, die iterativ dekodiert werden. Diese Art der Turbo-Kodes wurden auch in dieser Bachelorarbeit verwendet. Deshalb wird in Kapitel Kapitel~\ref{sec:parallelConvCodes} auch nur diese Art genauer erklärt.

\subsection{Parallel verkettete Faltungskodes} 
\label{sec:parallelConvCodes}
Bei der parallelen Verkettung von Faltungskodes werden meistens systematische (ein Ausgang wird durchgeschalten) rekursive Faltungskodierer verwendet, die nur wenige Register besitzen.

\begin{figure}[th]
\centering
\includegraphics[width=\ScaleIfNeeded]{pictures/TurboModel}
\caption{Schema der parallelen Kodeverkettung,~Quelle:~\cite[251]{schoenfeld2012informations}}
\label{pic:turbomodel}
\end{figure}

Bei der Kodierung wird, wie in Abbildung~\ref{pic:turbomodel} zu sehen, das Quellenkodewort $a^*$ in den Kodierer $K_1$ und permutiert in den Kodierer $K_2$ geschickt. Das resultierende Kanalkodewort $a$, wird dann aus dem originalen Quellenkodewort $a^*$ und den beiden kodierten Nachrichten $k1$ und $k2$ gebildet. Bei der Wahl der Interleaver gibt es verschiedene Ansätze, wie die Bits permutiert werden, damit eine größtmögliche Distanz $d_{min}$ zwischen den Kodewörtern erreicht wird. Der einfachste Interleaver ist jener, der die Bits zufällig austauscht, wobei Sender und Empfänger den Permutationsvektor kennen müssen. Die verschiedenen Arten werden in Kapitel~\ref{cha:implementation} genauer erklärt. Am Ende der Kodierung kann noch Punktierung angewendet werden, um die Koderate zu erhöhen. Dieses Verfahren wird in Kapitel~\ref{sec:puncturing} genauer erläutert.

\begin{e_exa}
Angenommen das Quellenkodewort ist $a^*=(100)$ und die beiden kodierten Nachrichten sind $k1=(110)$ und $k2=(001)$, dann setzt sich das resultierende Kanalkodewort folgenderweise zusammen $a=(110010001)$. Dabei werden zuerst alle ersten Bits ausgegeben, dann die Zweiten und am Ende die Dritten.
\end{e_exa}

Da bei der Übertragung die Nachrichten als Spannungen interpretiert werden, ist es vonnöten die Nachrichten auf die Signalpegel -1 und +1 zu transformieren. Dabei wird die folgende Abbildungsvorschrift verwendet:

\begin{equation*}
	\begin{cases}
	-1 & \quad \text{wenn Bit}=1\\
	1 & \quad \text{wenn Bit}=0\\
	\end{cases}
\end{equation*}

Die Rückwandlung in 0er und 1er erfolgt erst nach der Dekodierung, da der Soft-In/Soft-Out Dekodierer die genauen Signalpegel bei Berechnung der Zuverlässigkeitswerte verwendet. 

\begin{figure}[th]
\centering
\includegraphics[width=\ScaleIfNeeded]{pictures/TurboDecoderSchema}
\caption{Aufbau der Dekodierschaltung,~Quelle:~\cite[262]{schoenfeld2012informations}}
\label{pic:decoderSchema}
\end{figure}

Auf der Abbildung~\ref{pic:decoderSchema} ist der Dekodierer detaillierter zu erkennen. Dabei erkennt man, dass am Anfang die Depunktierung und das Aufspalten der Nachricht in die drei Bestandteile erfolgt. Die gesamten Interleaver $I$ und Deinterleaver $I^{-1}$ werden benötigt, um die korrekte Reihenfolge der Bits herzustellen. Auf diesem Schema ist sehr gut zu erkennen, dass die extrinsischen Informationen aus den Dekodierern $L^-_e(\widehat{u})$ und $L^|_e(\widehat{u})$ in den Eingang des Nächsten geschickt werden. Bei der ersten Iteration wird noch 0 in den ersten Dekodierer geschickt, bei weiteren Iterationen wird die Rückkopplung verwendet.

Zur Berechnung der extrinsischen Informationen kann entweder der MAP-Algorithmus \cite[233-236]{schoenfeld2012informations} oder der aufwandsgünstigere Soft-Output VITERBI-Algorithmus \cite[222-233]{schoenfeld2012informations} verwendet werden. Die genaue Berechnung der extrinsischen Informationen kann im Buch von Schönefeld \cite[S.~263~f.]{schoenfeld2012informations} nachgeschlagen werden. Auch die Bachelorarbeit von Martin Nocker~\cite[7-11]{nocker} liefert Aufschluss über die Berechnung der Soft-Output-Werte beim VITERBI-Algorithmus.

Am Ende der Iterationen kann die Zuverlässigkeitsinformation berechnet werden:

\begin{equation}
L(\widehat{u}(i))=y_1(i)+L^-_e(\widehat{u}(i))+L^|_e(\widehat{u}(i))
\label{eq:resultDecode}
\end{equation}

Mit dem in der Formel~\ref{eq:resultDecode} berechneten Zuverlässigkeitswert, können die geschätzten Informationsbits abgeleitet werden:

\begin{equation*}
\begin{cases}
0 & \quad L(\widehat{u}(i)) \geq 0 \\
1 & \quad \text{sonst}
\end{cases}
\end{equation*}

Daraus wurde nun die Originalnachricht rekonstruiert und steht dem Empfänger zur Verfügung. Versuche zeigen, dass Turbo-Kodes mit kleiner Registeranzahl ($\leq 5$), sehr langen Eingangsfolgen ($2^{16}$) und mit 10-20 Iterationen, eine Leistungsfähigkeit nahe der SHANNON-Grenze erreichen.~\cite[265]{schoenfeld2012informations}

Turbo-Kodes sind aus der heutigen Zeit nicht mehr wegzudenken, da sie in fast allen Bereichen eingesetzt werden. Nur mehr bei Anwendungen, bei denen die Dekodierungsverzögerung sehr klein sein muss, werden noch klassische Kodes eingesetzt.

\begin{figure}[th]
\centering
\includegraphics[width=\ScaleIfNeeded]{pictures/ComparisonTurbo}
\caption{Vergleich von verschiedenen Turbo-Kodes,~Quelle:~\cite[603]{huffman2010fundamentals}}
\label{pic:comparisonTurbo}
\end{figure}

Die Abbildung~\ref{pic:comparisonTurbo} zeigt eine Grafik, bei der auf der Ordinate die Bitfehlerrate und auf der Abszisse das Signal/Rausch-Verhältnis  aufgetragen sind. Aufgetragen sind verschiedene Turbo-Kodes mit verschiedenen Parametern. Es ist zu erkennen, dass eine längere Nachricht zu einer Verbesserung des Ergebnisses führt. Bei steigender Iterationsanzahl sinkt die Bitfehlerrate, jedoch steigt die Verzögerungszeit für das Dekodieren. Je mehr Iterationen durchgeführt werden, desto näher kommt man der SHANNON-Grenze.

\FloatBarrier
\subsection{Punktierung}
\label{sec:puncturing}
Wenn der Übertragungskanal besser als erwartet ist und noch Leistungsreserven vorhanden sind, kann Punktierung eingesetzt werden. Dabei werden Bits vor der Übertragung nach einer genauen Vorschrift (Punktierungsmatrix) gelöscht und vor der Dekodierung wieder eingefügt. Diese $m \times \frac{p}{m}$ Punktierungsmatrix $P$  wird spaltenweise durchlaufen und an den 0 Stellen werden die Bits gelöscht und ansonsten bleiben sie erhalten. Die Matrix muss bei Turbo-Kodes immer drei Zeilen haben, um bei der Dekodierung wieder ein eindeutiges Ergebnis zu liefern. Die resultierende Koderate erhält man folgenderweise:

\begin{equation}
R_p = \frac{p}{w(P)} * \frac{1}{3}
\end{equation} 

Die generelle Koderate bei Turbo-Kodes ist gleich $\frac{1}{3}$, da die Nachrichtenlänge verdreifacht wird. Darum findet sich diese Zahl auch in der Berechnung der Koderate mit Punktierung wieder.~\cite[218]{schoenfeld2012informations}

\begin{e_exa}
Angenommen die Kanalkodefolge ist $a=(101101011)$ und zur Erhöhung der Koderate wird folgende Punktierungsmatrix $$P_{3 \times \frac{6}{3}}=\begin{pmatrix}
1 & 0 \\
0 & 1 \\
1 & 1
\end{pmatrix}$$ angewendet. Die punktierte Kanalkodefolge ist dann $a_p=(1*1*010*1)=(110101)$. Die Koderate wird auf $R_p = \frac{6}{4} * \frac{1}{3} = \frac{1}{2}$ erhöht. 
\end{e_exa}

\chapter{Verwendete Technologien}
\label{cha:technologies}
\section{R, RStudio, Pakete}
\section{C++, Rcpp}
\section{R Markdown, \LaTeX, Ti\textit{k}Z}

\chapter{Implementierung}
\label{cha:implementation}
Bei den Implementierungen der Funktionen wird am Anfang immer überprüft, ob die Parameter richtig gesetzt wurden und alle mitgegebene Daten korrekt sind. Diese Überprüfungen werden bei der Erklärung nur dann ausgeführt, wenn sie speziell für diese Funktion von Bedeutung sind. 

Wenn nicht alle Parameter verwendet wurden, werden immer bestimmte Standardwerte gesetzt. Diese können in Kapitel~\ref{cha:interface} nachgelesen werden, die Erzeugung dieser wird hier jedoch nicht vorgestellt.

Sämtliche Hilfsfunktionen oder selbsterklärende Implementationen können im Code nachgelesen werden. Zur Erstellung aller Grafiken wird das Pakete \emph{ggplot2} verwendet. Die genauen Erklärungen zu diesen Funktionen findet sich im Buch zum Pakete~\cite{ggplot2}.

In Kapitel~\ref{sec:implementation_package} werden alle grundlegende Informationen zur Erstellung eines Paketes geliefert. Danach erfolgt die genaue Erklärung der Implementierung von der Erzeugung des Permutationsvektor in Kapitel~\ref{sec:implementation_permutation} und der Punktierungsmatrix in Kapitel~\ref{sec:implementation_puncturing}. Im Anschluss folgt die Kodierung~\ref{sec:implementation_encode} und Dekodierung~\ref{sec:implementation_decode}, um danach noch das Erzeugen des Rauschens in Kapitel~\ref{sec:implementation_applyNoise} zu erläutern. Zum Abschluss wird noch die Implementierung der Simulationen in Kapitel~\ref{sec:implementation_simulation} vorgestellt, als letztes erfolgt ein kleiner Einblick in die Erzeugung der Visualisierungen~\ref{sec:implementation_visualization}.

\section{Erstellung eines Paketes in RStudio}
\label{sec:implementation_package}
Pakete sind die Basis bei der Erzeugung wiederverwendbaren R-Codes. Dabei beinhalten Pakete Funktionen, Dokumentation zu den Funktionen und Beispieldaten. Die Erstellung eines eigenen Paketes ist mit der Entwicklungsumgebung RStudio sehr einfach und benötigt nicht viel Einarbeitungszeit.

Pakete werden mittels CRAN-Servern\footnote{The Comprehensive R Archive Network: \url{https://cran.r-project.org/}} geteilt, somit gibt es eine Quelle, von denen alle Entwickler die Pakete beziehen. Zur Installation eines Paketes kann RStudio oder der erste Befehl im Listing~\ref{lst:install} verwendet werden. Zur Entwicklung eines Paketes sind einige Hilfspakete nützlich, die bereits in Kapitel~\ref{cha:technologies} eingeführt wurden. 

\begin{lstlisting}[caption=Installation eines Paketes und dessen Abhängigkeiten, label={lst:install}, float=!th]
install.package()
install_deps(dependencies = TRUE)
\end{lstlisting}

Sollte C/C++-Code verwendet werden, müssen natürlicherweise ein Compiler und weitere Tools (in Windows RTools) am Entwicklungsrechner installiert sein. Genaue Informationen finden sich im Buch von Hadley Wickham~\cite[S.~18~ff.]{wickham2015r}.

Ein Paket wird mittels Wizard von RStudio (\emph{File$\rightarrow$New Project$\rightarrow$New Directory$\rightarrow$R Package}) erstellt. Dabei wird ein Ordner für das Paket erstellt, welcher eine DESCRIPTION- und eine NAMESPACE-Datei beinhaltet. Zusätzlich wird noch ein R-Ordner erstellt, der sämtliche R-Skripte (Funktionen) enthalten wird. Sollte C/C++-Code verwendet werden, muss dieser in den src-Ordner abgelegt werden. Alle Dateien im inst-Ordner werden automatisch bei der Installation kopiert, sind also vom R-Code aus erreichbar \cite[S.~196~ff.]{wickham2015r}. Noch dazu sind viele weitere Ordner, die allerdings nicht gebraucht wurden, in der Paketstruktur möglich.~\cite[S.~28~ff.]{wickham2015r}

In der DESCRIPTION-Datei werden alle wichtigen Informationen für das Paket gespeichert. Dies sind Entwicklerdaten, eine Kurzbeschreibung, Lizenzierung und eine Versionierung. Dort können auch benötigte Pakete angegeben werden, die bei der Installation vom CRAN-Server automatisch installiert werden. Bei einer lokalen Installation müssen diese Pakete selbständig nachinstalliert werden. Dies lässt sich einfach mit dem zweiten Befehl des Listings~\ref{lst:install} erledigen. Diese Funktion ist im Paket \emph{devtools} enthalten.~\cite[67-82]{wickham2015r}

Der Code wird mittels \emph{roxygen}-Kommentare dokumentiert. Dabei werden Kommentare in den Code eingefügt, die beim Erstellen des Paketes automatisch in eine Paket-Dokumentation umgewandelt werden. Die wichtigsten Annotationen wurden bereits in Kapitel~\ref{sec:R} vorgestellt.~\cite[83-110]{wickham2015r}

Die NAMESPACE-Datei verwaltet das Exportieren und Importieren von Funktionen in den Paketnamensraum. Diese Datei muss nicht verändert werden, da diese Datei beim Erstellen des Paketes automatisch erstellt wurde. Dabei werden die \texttt{@export}-Statements ausgewertet und diese Funktionen nach außen hin zugänglich gemacht. Bei der Verwendung vom \emph{Rcpp}-Paket und C++-Code müssen die 2 Zeilen vom Listing~\ref{lst:rcppNamespace}, in ein R-Skript geschrieben werden.~\cite[144-160]{wickham2015r}

\begin{lstlisting}[caption=Nötige \emph{roxygen}-Kommentare bei der Verwendung von C++-Code, label={lst:rcppNamespace}, float=!th]
#' @useDynLib channelcoding
#' @importFrom Rcpp sourceCpp
\end{lstlisting}

Damit die \emph{roxygen}-Kommentare ausgewertet werden, muss im RStudio (\emph{Build$\rightarrow$Configure Build Tools...}) die Einstellungen des Projektes geändert werden. Dabei muss ausgewählt werden, dass \emph{devtools} und \emph{roxygen} bei der Erstellung des Paketes verwendet werden.

Danach ist alles bereit, um das Paket zu erstellen. Dabei wird einfach im R-Studio am rechten oberen Rand der Reiter \emph{Build} ausgewählt und dort auf \emph{Build\&Reload} geklickt. Nun werden alle C++-Dateien kompiliert und R-Wrapper-Funktionen erstellt, die den Zugriff auf diese erleichtern. Nachdem der Prozess abgeschlossen ist, wird das Paket lokal installiert und ist zur Verwendung bereit. Möchte man das Paket auf einem anderen Computer installieren, muss das Paket zu einem Archiv gepackt werden. Dazu muss auf \emph{More$\rightarrow$Build Binary Package} geklickt werden, dann wird ein Archiv gebildet, das auf allen PCs, mit den selben Betriebssystem, installiert werden kann. Das bedeutet, dass das Paket für jedes Betriebssystem extra kompiliert werden muss. Darum sind 3 Versionen des Paketes für Windows, Linux und Mac OS nötig. Bei einem binären Paket kann der Benutzer den Code nicht einsehen. Wird jedoch \emph{Build Source Package} ausgewählt, ist der Code für den späteren Anwender einsehbar.   

\section{Erzeugung des Permutationsvektors}
\label{sec:implementation_permutation}
Die Interleaver beim Kodieren und Dekodieren benötigen einen Permutationsvektor, der die Vertauschungsreihenfolge der Nachricht beinhaltet. Diese Reihenfolge kann ganz unterschiedlich bestimmt werden. Um die Erzeugung dieses Vektors zu erleichtern, wurde eine Hilfsfunktion geschaffen, die dem Benutzer viele verschiedene Permutationstypen anbietet. Diese wurden aus dem Buch von Morelos-Zaragoza~\cite{morelos2006art} entnommen und nachimplementiert. Die Länge des Permutationsvektors muss immer der Länge der Nachricht, plus der Anzahl der Registern des Kodierers entsprechen. Bei vielen Typen wird eine Matrix zur Erzeugung verwendet, deswegen müssen die Argumente \texttt{cols} und \texttt{rows}, die in einer Liste eingebettet sind, genau der Länge des Permutationsvektors entsprechen.

\begin{lstlisting}[caption=Implementierung von \texttt{TurboGetPermutation}, label={lst:implementation_TurboGetPermutation}, float=!th]
RANDOM = {
  interleaver <- sample(c(0:(message.length + coder.info$M - 1)))
},
PRIMITIVE = {
  init <- c(0:N)
  interleaver <- (init - args$root) %% (N + 1)
},
CYCLIC = {
  interleaver <-
     t(apply(init, 1,
       function(x) {
         temp <- Shift(x, args$distance * (i))
         i <<- i + 1
         return(temp)
         }
      ))
 },
BLOCK = {
  init <- matrix(c(0:(N - 1)), nrow = rows, byrow = TRUE)
  return(as.vector((init)))
 },
HELICAL = {
  interleaver <-
    sapply(init, function(x) {
      x <- (((i %% cols) + (i * cols)) %% N)
      i <<- i + 1
      return(x)
    })
},
DIAGONAL = {
  interleaver <-
    sapply(init, function(x) {
      x <- (i * cols) %% N + (i %/% rows + i %% rows) %% cols
      i <<- i + 1
      return(x)
    })
}
\end{lstlisting}

Die verschiedene Typen sind in Listing~\ref{lst:implementation_TurboGetPermutation} in R-Code abgebildet:

\begin{itemize}
\item \texttt{RANDOM} - Dort wird eine zufällige Reihenfolge erzeugt.
\item \texttt{PRIMITIVE} - Dabei wird ein Vektor von 0 bis N, um den mitgegebenen Parameter, nach rechts oder links verschoben.
\item \texttt{CYCLIC} - Hier wird jede Zeile der Initialisierungsmatrix (\texttt{init}), um den Index, multipliziert mit \texttt{distance} verschoben. Danach wird die Matrix von oben nach unten als Vektor ausgegeben.
\item \texttt{BLOCK} - Bei diesem Typ wird ein Vektor von 0 bis N zeilenweise in eine Matrix eingelesen und spaltenweisen wieder ausgegeben.
\item \texttt{HELICAL} - Dabei wird wieder eine Initialisierungsmatrix erzeugt, die dann diagonal von links oben nach rechts unten ausgelesen wird. Sobald die letzte Zeile erreicht ist, wird in die erste Zeile gesprungen und von der nächsten Spalte aus weiter gemacht.
\item \texttt{DIAGONAL} - Wie auch beim vorigen Typ, wird hierbei beim Erreichen der letzten Zeile, auch in die erste Zeile gesprungen. Jedoch nicht in die nächste Spalte, sondern zum ersten Bit in der ersten Spalte, welches noch nicht gelesen wurde.
\end{itemize}

Alle diese Typen lassen sich am bestem verstehen, wenn man sie anwendet und die Matrizen ausgeben lässt (\texttt{visualize = TRUE}). Beispiele finden sich in Kapitel~\ref{sec:example_createHelpers}.

\FloatBarrier
\section{Erzeugung der Punktierungsmatrix}
\label{sec:implementation_puncturing}
Die Implementierung der Funktion zur Erstellung der Punktierungsmatrix sieht sehr einfach aus. Dort wird einfach der mitgegebene Vektor in eine Matrix mit 3 Zeilen verpackt. 

\begin{lstlisting}[caption=Implementierung von \texttt{TurboGetPunctuationMatrix}, label={lst:implementation_TurboGetPunctuationMatrix}, float=!th]
if (length(punctuation.vector) %% 3 != 0) {
  stop("Wrong length of punctuation vector! Must be a multiple of 3!")
}
mat <- matrix(punctuation.vector, nrow = 3)
if (any(colSums(mat) == 0)) {
  stop("Punctuation matrix should not have a 0 column!")
}
\end{lstlisting}

In Listing~\ref{lst:implementation_TurboGetPunctuationMatrix} sieht man zuerst die Überprüfung auf die Länge des mitgegeben Vektors, da dieser ein Vielfaches von drei sein muss. Danach wird die Matrix aus dem Vektor erzeugt und im Anschluss geprüft, ob eine Nullspalte existiert. Das darf nicht vorkommen, da sonst ein eindeutiges Einfügen der Bits nicht mehr möglich wäre.

\FloatBarrier
\section{Kodierung}
\label{sec:implementation_encode}
Die Kodierung erfolgt laut der in den Grundlagen, in Kapitel \ref{sec:turboCodes}, besprochenen Schaltung. Dabei werden die Kodierungsfunktionen von der Faltungskodierung von Martin Nocker~\cite[S.~25~f.]{nocker} verwendet.

\begin{lstlisting}[caption=Implementierung von \texttt{TurboEncode}, label={lst:implementation_TurboEncode}, float=!th]
parity.1 <- ConvEncode(message, coder.info, TRUE)
temp.index <- c(rep(FALSE, 0), TRUE, rep(FALSE, coder.info$N - 1))
code.orig <- parity.1[temp.index]

code.perm <- as.numeric(code.orig[permutation.vector + 1] < 0)
parity.2 <- ConvEncode(code.perm, coder.info, FALSE)
temp.index <- c(rep(FALSE, parity.index - 1), TRUE, rep(FALSE, coder.info$N - parity.index))
parity.1 <- parity.1[temp.index]
parity.2 <- parity.2[temp.index]

code.result <- Interleave(code.orig, parity.1, parity.2)
if (!is.null(punctuation.matrix)) {
  code.punct <- PunctureCode(code.result, punctuation.matrix)
}
if (!is.null(punctuation.matrix)) {
  return(list(original = code.result, punctured = code.punct))
} else {
  return(code.result)
}
\end{lstlisting}

Das abgebildete Listing~\ref{lst:implementation_TurboEncode} spiegelt genau diese Schaltung in R-Code wieder. Dabei wird in \texttt{parity.1} und \texttt{parity.2} die Ergebnisse der beiden Kodierer abgespeichert. Vor dem zweitem Kodierer wird die Originalnachricht permutiert und in \texttt{code.perm} gesichert. Die \texttt{temp.index}-Variablen dienen zum Extrahieren der einzelnen Nachrichten eines Ausganges des Faltungskodierer. Da der Turbo-Kodierer nur einen Ausgang nutzt, ist diese Vorgehen nötig.

Nachdem alle drei Teile erzeugt wurden, können diese in die Variable \texttt{code.result} zusammengefasst werden. Wenn Punktierung verwendet wird, werden noch die nötigen Bits mit der \texttt{PunctureCode}-Funktion herausgestrichen. 

Der Rückgabewert der Funktion ist die kodierte Nachricht. Bei verwendeter Punktierung wird eine Liste mit punktierter und nicht punktierter Nachricht zurückgegeben. Somit hat der Benutzer die Möglichkeit beide Varianten für die Dekodierung zu verwenden.  

\FloatBarrier
\section{Dekodierung}
\label{sec:implementation_decode}
Bei der Dekodierung wird der Soft-Viterbi Algorithmus verwendet, um die Nachricht mit der kleinsten Abweichung zu suchen. Dieser Algorithmus ist sehr aufwändig, deswegen wurde dieser in C++ implementiert, da dort die vielen Schleifen schneller abgearbeitet werden, als im R-Code. Als Anhaltspunkt wurde eine Referenzimplementierung von Dusan Orlovic~\cite{SOVA} verwendet.

\begin{lstlisting}[caption=Implementierung von \texttt{TurboDecode}, label={lst:implementation_TurboDecode}, float=!th]
if (!is.null(punctuation.matrix)) {
  code.with.punct <- InsertPunctuationBits(code, punctuation.matrix)
} else {
  code.with.punct <- code
}

code.length <- length(code.with.punct) / 3
code.orig <- Deinterleave(code.with.punct, 1)
parity.1 <- Deinterleave(code.with.punct, 2)
parity.2 <- Deinterleave(code.with.punct, 3)

decoded <- c_turbo_decode(code.orig, parity.1, parity.2, permutation.vector, iterations, coder.info$N, coder.info$M, coder.info$prev.state, coder.info$output, parity.index)

output.soft <- head(decoded$soft.output, code.length - coder.info$M)
output.hard <- head(decoded$hard.output, code.length - coder.info$M)
message.decoded <- list(output.soft = output.soft, output.hard = output.hard)
\end{lstlisting}


Im R-Code werden, falls Punktierung verwendet wurde, zuerst die Punktierungsbits bei der erhaltenen Nachricht wieder eingefügt. Dies ist in Listing~\ref{lst:implementation_TurboDecode} ersichtlich. Danach wird das Signal wieder in die drei Teile aufgespalten und dann der C++-Funktion übergeben, die wie ein normaler Aufruf zu verwenden ist. Hier sieht man sehr gut die Kapselung des C++-Aufrufs in eine normale R-Funktion, welches das Paket \emph{Rcpp} ermöglicht. In dieser Funktion erfolgt dann die Dekodierung. Als Rückgabewert erhält man im R-Code eine Liste mit den Soft- und Hard-Werten. Bei diesen Werten werden noch die Terminierungsbits entfernt. Aus der R-Funktion wird eine Liste von den Soft-Werten und der dekodierten Nachricht
zurückgegeben. Damit kann der Anwender vergleichen, wie sich eine verschiedene Anzahl von Iterationen auf die Soft-Werte auswirkt.

\begin{lstlisting}[language=C++,caption=Implementierung der C++-Funktion zur Dekodierung, label={lst:implementation_cFunction}, float=!th]
for(int i = 0; i < N_ITERATION; i++) {
	Le1 = c_sova(x_noisy, parity_noisy1, Le2_ip, 1, N, M, previous_state, output, output_index);
    for(int k = 0; k < msg_len; k++) {
		Le1_p[k] = Le1[permutation[k]];
		x_d_p[k] = x_noisy[permutation[k]];
    }
	Le2 = c_sova(x_d_p, parity_noisy2, Le1_p, 0, N, M, previous_state, output, output_index);
    for(int k = 0; k < msg_len; k++) {
    	Le2_ip[permutation[k]] = Le2[k];
	}
}
for(int k = 0; k < msg_len; k++) {
   	soft_output[k] = Lc * x_noisy[k] + Le1[k] + Le2_ip[k];
   	hard_output[k] = (soft_output[k] >= 0.0) ? 0 : 1;
}
\end{lstlisting}

Nun ist die C++-Dekodierungsfunktion in Listing~\ref{lst:implementation_cFunction} zu sehen. Dabei wird die Dekodierungsschaltung aus dem Theoriekapitel~\ref{sec:parallelConvCodes} abgebildet. Der gesamte Code ist in eine Schleife über die Iterationen gepackt. Innerhalb der Schleife wird zuerst der erste Teil der Nachricht (\texttt{parity\_noisy1}) in die Viterbi-Funktion (\texttt{c\_sova}) geschickt. Zurückgegeben werden die Soft-Werte (\texttt{Le1}), die gleich im Anschluss permutiert (\texttt{Le1\_p}) in den nächsten Dekodierer geschickt werden. Die zurückgelieferten Soft-Werte (\texttt{Le2}) werden wieder permutiert und sind für die nächste Iteration bereit.

Nachdem alle Iterationen erledigt sind, wird die Endberechnung, die in Kapitel~\ref{sec:parallelConvCodes} erklärt wurde, durchgeführt. Danach werden noch die Hard-Werte, also die eigentlich dekodierte Nachricht, gebildet. Dazu werden einfach die Soft-Werte ausgewertet und die Bits im Vektor \texttt{hard\_output} gesetzt.

Die genaue Implementierung des Viterbi-Algorithmus (\texttt{c\_sova}) wird in der Bachelorarbeit der Faltungskodes~\cite[S.~26~ff.]{nocker} erläutert, da dies ein Teil der Faltungskodierung ist. Die Theorie zum implementierten Algorithmus ist auch im Buch von Schönefeld~\cite[222-233]{schoenfeld2012informations} erklärt.

\FloatBarrier
\section{Rauschen erzeugen}
\label{sec:implementation_applyNoise}
Das Umsetzen des bereits in Kapitel~\ref{sec:channels} erklärten Kanalmodells, orientiert sich an den Formeln der Theorie. Diese werden einfach in den R-Code verpackt und das berechnete Rauschen wird dem Signal hinzugefügt.  

\begin{lstlisting}[caption=Implementierung von \texttt{ApplyNoise}, label={lst:implementation_ApplyNoise}, float=!th]
msg.len <- length(msg);
SNR.linear <- 10^(SNR.db/10);
power <- sum(msg^2)/(msg.len); 
noise <- sqrt(power / SNR.linear) * rnorm(msg.len,0,1)
msg.out <- msg + noise;
\end{lstlisting}

Zur Berechnung des Rauschens ist die Leistung pro Bit erforderlich, das wird in den ersten drei Code-Zeilen von Listing~\ref{lst:implementation_ApplyNoise} errechnet. Danach wird eine Zufallszahl zwischen 0 und 1 erzeugt, welche mit dem Ergebnis der Wurzel multipliziert wird. Dieser Vektor wird dann mit dem Originalsignal überlagert.

\FloatBarrier
\section{Simulationen}
\label{sec:implementation_simulation}
Bei einer Simulation wird ein Kodierungsverfahren mehrmals mit einer festgelegten Nachrichtenlänge für verschiedene Signal/Rausch-Verhältnis getestet, um am Ende die Anzahl der Fehler pro Bit zu errechnen. Sämtliche Parameter der Simulation können mittels Parametern beeinflusst werden.

Die Implementierung der Turbo-Kode-Simulation wird in Kapitel~\ref{sec:implementation_turbo} erläutert. Danach wird diese Funktion bei der Simulation aller Kanalkodierungen in Kapitel~\ref{sec:implementation_channelcoding} verwendet.

\subsection{Turbo-Kode}
\label{sec:implementation_turbo}
Bei der reinen Simulation vom Turbo-Kode-Verfahren wird eine zufällige Nachricht erzeugt die dann kodiert, verrauscht und wieder dekodiert wird. Dabei wird die Anzahl der nicht korrigierten Fehlerbits gezählt, um am Ende ein Dataframe zu erzeugen, welches pro Signal/Rausch-Verhältnis eine Bitfehlerrate beinhaltet.

\begin{lstlisting}[caption=Implementierung von \texttt{TurboSimulation}, label={lst:implementation_TurboSimulation}, float=!th]
v.db <- seq(from = min.db, to = max.db, by = db.interval)
for (db in v.db) {
  for (i in 1 : iterations.per.db) {
    message <- sample(c(0,1), msg.length, replace = TRUE)
    
    coded <- TurboEncode(message, perm, coder, punctuation.matrix = punctuation.matrix)
    noisy <- ApplyNoise(coded, db)
    decoded <- TurboDecode(noisy, perm, decode.iterations, coder, punctuation.matrix = punctuation.matrix)
                             
    decode.errors <- sum(abs(decoded$output.hard - message))
    total.errors <- total.errors + decode.errors
  }
  v.ber <- c(v.ber, total.errors / (msg.length * iterations.per.db))
  total.errors <- 0
}
df <- data.frame(db = v.db, ber = v.ber)
\end{lstlisting}

Die Implementierung wird in Listing~\ref{lst:implementation_TurboSimulation} dargestellt. Dabei wird in der ersten Zeile ein Vektor (\texttt{v.db}) erzeugt, der alle Signal/Rausch-Verhältnisse beinhaltet, die getestet werden. Im Anschluss wird über diese Schleife iteriert, in welcher das gesamte Verfahren abgearbeitet wird. Nach Kodierung, Verrauschung und Dekodierung wird gezählt, wieviele Bits (\texttt{decode.errors}) nicht der Originalnachricht entsprechen. Um die Genauigkeit der Simulation zu erhöhen wird der Vorgang (\texttt{iterations.per.db}) mehrmals ausgeführt und am Ende der Durchschnitt (\texttt{v.ber}) berechnet. In den letzten Zeilen wird die Bitfehlerrate dem Dataframe hinzugefügt.

\FloatBarrier
\subsection{Kanalkodierung}
\label{sec:implementation_channelcoding}
Bei dieser Funktion werden einfach die mitgegebenen Simulationsparameter auf alle drei Kanalkodierungsverfahren ausgeführt und danach die Dataframes zusammengefasst. Somit ist diese Funktion eine Erleichterung, wenn man Block-, Faltungs- und Turbo-Kodes miteinander vergleichen möchte.

\FloatBarrier
\section{Visualisierung}
\label{sec:implementation_visualization}
Die Implementierung der Visualisierung erfolgt, wie bereits in Kapitel \ref{sec:RMarkdown} erklärt, mittels \emph{RMarkdown}. Dieses Dateiformat kann \LaTeX , Ti\textit{k}Z und R-Code beinhalten und wird mit der \texttt{render}-Funktion in ein PDF-Dokument umgewandelt. Mit dem Befehlen in Listing~\ref{lst:implementation_renderOpen} wird das PDF erzeugt und anschließend geöffnet.

\begin{lstlisting}[caption=Erzeugung und Öffnung des PDF-Berichts, label={lst:implementation_renderOpen}, float=!th]
rmarkdown::render("file.rmd", params)
rstudioapi::viewer("file.rmd")
\end{lstlisting}

Der erste Parameter ist die Pfadangabe zur Datei und mit dem zweiten kann eine Liste als Parameter an die \emph{RMarkdown}-Datei übergeben werden. Der zweite Befehl öffnet dann die erzeugte Datei. Dieser Prozess kann nur funktionieren, wenn das \emph{RMarkdown}-Paket und \LaTeX\ auf dem Rechner installiert ist.

\begin{lstlisting}[caption=Header der \emph{RMardown}-Datei, label={lst:implementation_headerRMarkdown}, float=!th]
title: "Turbokodierung ohne Punktierung"
date: "`r format(Sys.time(), '%d %B, %Y')`"
output: 
  beamer_presentation:
    keep_tex: true
params:
 output=matrix(c(0,0,1,1,3,3,2,2), ncol =2))
header-includes:
- \usepackage{tikz}
- \usepackage{pgfplots}
- \usepackage{color}
- \usetikzlibrary{arrows,positioning,calc}
\end{lstlisting}

Die \emph{RMarkdown}-Datei startet mit einem Header, welcher in Listing~\ref{lst:implementation_headerRMarkdown} gezeigt wird. Dort werden zuerst Daten eingetragen,wie Titel oder Datum, die auf der Startfolie präsentiert werden. Danach wir die Art des erzeugten Dokumentes festgelegt. Hier wird eine Präsentation gewählt, bei der die \LaTeX -Datei ebenfalls mit abgespeichert wird. Im Anschluss können mitgegebene Parameter mit Standardwerten deklariert werden. Zum Schluss werden noch Pakete für \LaTeX\ eingebunden, die zur Ausführung benötigt werden.

Danach können die einzelnen Folien für die Visualisierungsberichte verfasst werden. Auf die jeweilige Implementierung der Berichte wird nicht näher eingegangen, da es den Rahmen der Bachelorarbeit sprengen würde. 




\startlist[Funktionen]{lot}
\chapter{R-Paket Schnittstelle}
\label{cha:interface}
% R Paket Schnittstelle für Faltungskodierung
\definecolor{lightblue}{RGB}{150,180,255}

\section{Kanalkodierung}
\label{chapter:interface_kanalkodierung}
\begin{longtable}{|p{\textwidth}|}
\hline
\rowcolor{lightblue}
ApplyNoise\\
\hline
\\
\texttt{ApplyNoise(msg, SNR.db, binary)}\\
\\
Verrauscht ein Signals basierend auf das AWGN (additive white gaussian noise) Modell. Das ist das Standardmodell für die Simulation eines Übertragungskanals.\\
\\
\textbf{Argumente:}\\
\texttt{msg} - Nachricht die verrauscht wird.\\
\texttt{SNR.db} - Signal/Rausch-Verhältnis des Übertragungskanals. Standard: 3.0\\
\texttt{binary} - Blockkode-Parameter. Nicht zu verwenden! Standard: FALSE\\
\\
\textbf{Rückgabewert:}\\
Verrauschtes Signal.\\
\\
\hline
\caption{ApplyNoise - Funktionserklärung}
\end{longtable}
\stoplist[Funktionen]{lot}

\chapter{Visualisierung}
\label{cha:visualization}
Ein wichtiger Aspekt dieser Bachelorarbeit war den Studenten den Einstieg in die Kanalkodierung mit diesem Paket zu erleichtern. Um das Verständnis über die Verfahren zu erleichtern, wurden Visualisierungen geschaffen, die den Ablauf der Kodierung und Dekodierung genauestens Darstellen. Dadurch kann in zukünftigen Proseminaren, die in der Theorie gelernten Prinzipien der Kanalkodierung, mithilfe des Paketes in der Praxis ausprobiert und besser verstanden werden.

Um die Visualisierungen zu erhalten, gibt es in jeder Funktion einen Parameter (\emph{visualize}) der mit TRUE gesetzt werden muss, um die Visualisierung zu erhalten. Alle Beispiele in den nächsten Kapitel werden mit Punktierung durchgeführt, da sich die Visualisierung ohne Punktierung ein wenig vereinfacht und das nicht nochmals erklärt werden muss. Die erzeugten PDF-Dokumente werden mit \emph{RMarkdown} und dem darin enthaltenen \LaTeX - und Ti\textit{k}Z-Code erzeugt. Deswegen ist es auch notwendig, dass der Benutzer des Paketes \LaTeX und die wichtigsten Pakete dazu installiert hat!

Da eine schöne Visualisierung nur Sinn mit sehr kurzen Bitfolgen macht, ist die Darstellung auch beschränkt, da ansonsten die Datenflut für den Nutzer nicht mehr bewältigbar wäre. Nebenbei wäre eine visuell schöne Darstellung auch nicht mehr möglich. Deswegen wird ab einer Nachrichtenlänge die größer als 10 ist, eine Warnung ausgegeben. Tatsächlich wird im erzeugten Dokument ab einer Länge von \textbf{18 Bits} die Nachrichten abgeschnitten.

Die erzeugten PDF-Dokumente werden im Installationsordner des Paketes abgelegt, der im Programmverzeichnis von R liegt. Dort liegen sie im \textbf{\emph{inst\textbackslash pdf}-Ordner}. Mit der Hilfsfunktion \emph{TurboOpenPDF} lassen sich bereits erzeugte Dokumente nochmals öffnen. Möchte man diese weiterverwenden bietet sich an, die geöffneten Dokumente neu in einem gewünschten Ordner abzuspeichern.

Die Dokumente werden aus reinem \LaTeX -Code erzeugt, dieser ist ebenfalls in dem oben erwähnten Ordner zu finden. Dieser Code kann verwendet werden, um die Grafiken oder Tabellen im Visualisierungsbericht selbst in einem Dokument nachzustellen.

Zuerst wird in Kapitel \ref{sec:visualization_punctuationPermutation} die Visualisierung innerhalb vom RStudio von Permutations- und Punktierungsmatrix vorgestellt. Im Anschluss erfolgen in Kapitel \ref{sec:visualization_encode} und \ref{sec:visualization_decode} die Erklärung der erzeugten Berichte von Kodierung und Dekodierung. Am Schluss werden noch alle Darstellungen der Simulationen in Kapitel \ref{sec:visualization_simulation} an einem Beispiel gezeigt. 
\section{Permutationsmatrix und Punktierungsmatrix}
\label{sec:visualization_punctuationPermutation}
Bei der Erzeugung von den beiden Matrizen können diese im RStudio dargestellt werden. Das ist vor allem nützlich bei der Permutationsmatrix, da dort erkennbar ist, wie die Matrix von der Initialisierungsmatrix entsteht und daraus der Permutationsvektor abgeleitet wird. 

\begin{lstlisting}[caption=Visualisierung der Permutationsmatrix, label={lst:visualizePermutation}, float=!ht]
input2 <- c(1,0,1,1,0,1)
permutation.vector.cyclic <- TurboGetPermutation(length(input2), coder, "CYCLIC", list(cols=4, rows=2, distance=2), TRUE)
[1] "Initial-Matrix"
     [,1] [,2] [,3] [,4]
[1,]    0    2    4    6
[2,]    1    3    5    7
[1] "Interleaver-Matrix:  CYCLIC"
     [,1] [,2] [,3] [,4]
[1,]    0    2    4    6
[2,]    5    7    1    3
> permutation.vector.cyclic
[1] 0 5 2 7 4 1 6 3

permutation.vector.helical <- TurboGetPermutation(length(input2), coder, "HELICAL", list(cols=4, rows=2), TRUE)
[1] "Initial-Matrix"
     [,1] [,2] [,3] [,4]
[1,]    0    1    2    3
[2,]    4    5    6    7
[1] "Interleaver-Vector:  HELICAL"
[1] 0 5 2 7 0 5 2 7
\end{lstlisting}

In Listing \ref{lst:visualizePermutation} wird zuerst der selbe Eingangsvektor erzeugt, wie in Kapitel \ref{cha:examples} verwendet wurde. Danach wird ein Permutationsvektor vom Typ \emph{CYCLIC} erzeugt. Dabei wird zuerst eine Initialisierungsmatrix erstellt, bei der dann die Bits jeder Zeile, um den Index multipliziert mit der Distanz nach rechts verschoben werden. Der Index fängt bei null an zu zählen, also wird die zweite Zeile um $1*2 = 2$ nach rechts verschoben. Die erste Zeile bleibt immer unverändert. Die daraus resultierende Matrix wird dann von oben nach unten spaltenweise ausgelesen, wie im Beispiel zu sehen.

Beim zweiten Beispiel wird ein Permutationsvektor vom Typ \emph{HELICAL} kreiert. Dabei wird nicht die Initialisierungsmatrix verändert, sondern nur die Art der Auslesung des Permutationsvektors ist für diesen Typ entscheidend. Das ist ebenfalls beim Typ \emph{BLOCK} und \emph{DIAGONAL} so. Bei diesem Beispiel werden die Bits von links oben nach rechts unten ausgelesen. Wird dabei das Ende der Matrix erreicht, wir das nächste Bit in der nächsten Spalte von der ersten Zeile ausgelesen. Beim Typ \emph{DIAGONAL} wird vergleichsweise das erste Bit in der selben Spalte als nächstes verwendet.

\begin{lstlisting}[caption=Visualisierung der Punktierungssmatrix, label={lst:visualizePunctuationMatrix}, float=!ht]
punctuation.matrix <- TurboGetPunctuationMatrix(c(1,1,0,0,1,1), TRUE)
[1] "Punctuation-Matrix:"
     [,1] [,2]
[1,]    1    0
[2,]    1    1
[3,]    0    1
\end{lstlisting}

Zur Verwendung einer Punktierung beim Turbo-Kode-Verfahren benötigt es eine Punktierungsmatrix, diese lässt sich mit einer Hilfsfunktion wie in Listing \ref{lst:visualizePunctuationMatrix} erzeugen. Bei der Visualisierung handelt es sich einfach um eine Darstellung der resultierenden Matrix. Wichtig dabei ist, dass der mitgegebene Vektor immer durch 3 teilbar sein muss, also hat die Matrix immer drei Zeilen. Bei einer 0 wird das Bit gelöscht und bei einer 1 behalten.

\section{Kodierung}
\label{sec:visualization_encode}
Als erstes werden immer Informationen zum Kodierer, der verwendet wurde, dargestellt. Diese Darstellung wird allerdings bereits bei der Bachelorarbeit von Martin Nocker über Faltungskodes erklärt \cite{nocker}.

\begin{figure}[!ht]
\centering
\includegraphics[width=\ScaleIfNeeded]{pictures/TurboEncodePunctured1}
\caption{Turbo-Kodierer Informationen}
\label{pic:TurboCoderInformation}
\end{figure}  

Die genauen Informationen über die Turbo-Kodierer Parameter finden sich auf der Abbildung \ref{pic:TurboCoderInformation}. Dort wird als erstes der Permutationsvektor angezeigt, der beim Interleaver verwendet wird. Das stellt die Reihenfolge der Bits dar, wie sie aus dem Interleaver kommen. Als nächstes wird die Kode-Rate angezeigt, die bei einem normalen Turbo-Kodierer ohne Punktierung immer $\frac{1}{3}$ ist. Hier wird der Wert mit Punktierung berechnet und dargestellt. Am Ende wird noch die Punktierungsmatrix abgebildet.

\begin{figure}[!ht]
\centering
\includegraphics[width=\ScaleIfNeeded]{pictures/TurboEncodePunctured2}
\caption{Terminierung bei der Kodierung}
\label{pic:TerminationEncode}
\end{figure}  

In Abbildung \ref{pic:TerminationEncode} sieht man die Darstellung der Terminierung. Da die Faltungskodierer eine Terminierung vorsehen, muss bei der Eingangsbitfolge noch die Terminierungsbits berechnet werden, damit der Kodierer am Ende im Ausgangszustand ist. Diese Bits werden an die Ausgangsnachricht angefügt und in Orange dargestellt. Bei einem nicht rekursiven Faltungskodierer sind es immer 0er, jedoch bei einem rekursiven müssen sie berechnet werden.

\begin{figure}[!ht]
\centering
\includegraphics[width=\ScaleIfNeeded]{pictures/TurboEncodePunctured3}
\caption{Turbo-Kode Schaltung}
\label{pic:TurboEncode}
\end{figure}  

Auf der nächsten Folie, die in Abbildung \ref{pic:TurboEncode} dargestellt ist, ist die komplette Schaltung abgebildet, die für die Kodierung nötig ist. Die beiden Kodierer sind die übergebenen Faltungskodierer, die verwendet werden, um einen Teil des resultierenden Signals zu erhalten. Die Bits sind eingefärbt, damit man das Ergebnis nach dem Multiplexer aus den Farben folgern kann. Die drei Einzelbitfolgen werden so ineinander verschachtelt, damit jeweils ein Bit aus Teilfolge eins, zwei und drei hintereinander sind. Das lässt sich am Bestem mit Abbildung \ref{pic:TurboEncodeMultiplexer} zeigen. Dort sieht man sehr leicht, wie der Multiplexer die Bitfolgen ausgibt.

\begin{figure}[!ht]
\centering
\includegraphics[width=\ScaleIfNeeded]{pictures/TurboEncodePunctured4}
\caption{Multiplexer}
\label{pic:TurboEncodeMultiplexer}
\end{figure}  

\begin{figure}[!ht]
\centering
\includegraphics[width=\ScaleIfNeeded]{pictures/TurboEncodePunctured5}
\caption{Multiplexer}
\label{pic:TurboEncodePuncturing}
\end{figure}  

Auf der letzten Abbildung \ref{pic:TurboEncodePuncturing} ist die Punktierung sehr einfach dargestellt. Dabei wird jedes zu löschende Bit mit einem Stern * angezeigt. Somit lässt sich einfach nachvollziehen, dass die Bitfolge von oben nach unten spaltenweise durch die Punktierungsmatrix geführt wird und bei einer 0 dieses Bit entfernt wird. Am Ende ist das fertige Signal dargestellt, das dann auf den Übertragungskanal gelangt. 

Bei der Kodierung ohne Punktierung ist der Ablauf exakt der Selbe, nur wird eben das Signal am Ende nicht punktiert, sondern direkt ausgegeben.

\section{Dekodierung}
\label{sec:visualization_decode}

\section{Simulation}
\label{sec:visualization_simulation}

\subsection{Turbo-Kode}
\label{sec:visualization_simulations_turbo}

\subsection{Kanalkodierung}
\label{sec:visualization_simulations_channelcoding}

\chapter{Beispiele}
\label{cha:examples}
\section{Erzeugen von Kodierer, Permutationsvektor und Punktierungsmatrix}
\label{sec:example_createHelpers}
Zuerst werden Hilfsvariablen erzeugt, die für die folgende Kodierungs- und Dekodierungsverfahren benötigt werden. Diese Variablen müssen nicht unbedingt mitgegebenen werden, dann werden die Standard-Werte verwendet. Das bedeutet, dass falls keine Punktierungsmatrix übergeben wird, der Standard-Wert \emph{NULL} ist und darum nicht punktiert wird. Ebenfalls wird nicht permutiert, wenn kein Permutationsvektor als Argument übergeben wird. Es wird zwar ein Permutationsvektor benötigt, damit die Interleaver richtig arbeiten können, jedoch wird ein Vektor vom Typ \emph{PRIMITIVE} verwendet, dieser allerdings ändert die Reihenfolge der Bits nicht (root=0). Beim Kodierer verhält es sich ähnlich, jedoch wird hier ein vordefinierter Standard-Faltungskodierer verwendet (\emph{ConvGenerateRscEncoder(2,2,c(5,7))}), dieser wird in allen Turbo-Kode-Funktionen benützt, falls kein eigener Kodierer übergeben wird.

\begin{lstlisting}[caption=Erzeugung von Kodierer und Punktierungsmatrix, label={lst:createHelpersCoderPunctuation}]
input <- c(1,0,1,1,0)

coder <- ConvGenerateEncoder(2, 2, c(4,5))

punctuation.matrix <- TurboGetPunctuationMatrix(c(1,1,0,0,1,1))
\end{lstlisting}

Wie im Listing \ref{lst:createHelpersCoderPunctuation} zu sehen ist, wird als allererstes eine zu kodierende Nachricht erzeugt, die dann benötigt wird, die dann benötigt wird um eine Punktierungsmatrix zu bekommen. Anschließend wird ein nicht rekursiver Faltungskodierer, mit der Funktion vom Faltungskode-Teil des Packetes \cite{nocker}, erzeugt. Dabei wurde ein Kodierer mit 2 Registern und 2 Ausgängen gewählt. Wichtig ist, dass der erste Ausgang vom Eingang durchgeschalten wird damit es ein systematischer Kodierer ist. Das wird erreicht, indem das 1.Generatorpolynom mit $2^M = 2^2 = 4$ deklariert wird.

\begin{lstlisting}[caption=Erzeugung von verschiedenen Permutationsvektoren, label={lst:createHelpersPermutation}]
permutation.vector.random <- TurboGetPermutation(length(input), coder, "RANDOM")
permutation.vector.primitve <- TurboGetPermutation(length(input), coder, "PRIMITIVE", list(root=3))

input2 <- c(1,0,1,1,0,1)
permutation.vector.cyclic <- TurboGetPermutation(length(input2), coder, "CYCLIC", list(cols=4, rows=2, distance=2))
permutation.vector.block <- TurboGetPermutation(length(input2), coder, "BLOCK", list(cols=4, rows=2))
permutation.vector.helical <- TurboGetPermutation(length(input2), coder, "HELICAL", list(cols=4, rows=2))
permutation.vector.diagonal <- TurboGetPermutation(length(input2), coder, "DIAGONAL", list(cols=4, rows=2), TRUE)
[1] "Initial-Matrix"
     [,1] [,2] [,3] [,4]
[1,]    0    1    2    3
[2,]    4    5    6    7
[1] "Interleaver-Vector:  DIAGONAL"
[1] 0 5 1 6 2 7 3 4
\end{lstlisting}

Als nächstes benötigt der Benutzer noch einen Permutationsvektor, dieser wird im Listing \ref{lst:createHelpersPermutation} erzeugt. Dabei wird jeder Typ einmal verwendet. Bei den ersten beiden werden die Eingangsbits von Listing \ref{lst:createHelpersCoderPunctuation} benützt. Da bei den restlichen Typen eine Matrix benötigt wird, muss die Länge der Eingangsbits genau in einer Matrix Platz finden, deshalb wurde ein 2. Eingangsvektor definiert, der allerdings nur als Demonstration für die anderen Permutationstypen dienen sollte, dieser wird in den weiteren Beispielen nicht weiter verwendet. Es kann auch das Visualisierungsflag auf \emph{TRUE} gesetzt werden, dann wird die Ausgangsmatrix und der resultierende Permutationsvektor dargestellt. Dies wurde bei dem letzten Beispiel mit dem Typ \emph{DIAGONAL} demonstrativ verwendet.

\section{Kodieren und Dekodieren ohne Punktierung}
\label{sec:example_withoutPunctuation}
Nachdem alle benötigten Variablen gesetzt wurden kann nun die Kodierung und Dekodierung vorgenommen werden. Dabei wird zuerst ohne Punktierung gearbeitet, jedoch alle anderen zuvor definierten Variablen verwendet.

\begin{lstlisting}[caption=Kodierung und Dekodierung ohne Punktierung, label={lst:encodeDecodeWithoutPunctuation}]
encoded <- TurboEncode(input, permutation.vector.random, coder, 2)
encoded
[1] -1 -1 -1  1  1  1 -1  1  1 -1 -1  1  1 -1 -1  1 -1 -1  1  1  1

encoded.noisy <- ApplyNoise(encoded, 0.1)
round(encoded.noisy, 2)
[1] -2.41 -2.36 -0.26  1.06  1.37  0.22  1.13  0.93  0.33 -1.15 -1.18  0.59
[13]  1.32 -0.31 -1.37 -0.01 -1.42 -1.62  2.18  2.29  0.93

decoded <- TurboDecode(encoded.noisy, permutation.vector.random, 5, coder, 2)
decoded
$output.soft
[1]  -1.394185  1.394185  2.775904 -1.394185  2.775904

$output.hard
[1] 1 0 1 1 0
\end{lstlisting}

Der gesamte Vorgang lässt sich in wenigen Zeilen erledigen, wie in Listing \ref{lst:encodeDecodeWithoutPunctuation} zu sehen ist. Zuerst werden einfach die zuvor definierten Variablen der Kodierungsfunktion mitgegeben, als letzter Parameter wird noch der Index des zu verwendeten Ausgangs des Kodierers angegeben (2 in diesem Fall). Danach erhält man das kodierte Signal mit dem Signalpegel 1 und -1. Dieses Signal wird dann einem Signal/Rausch-Verhältnis von 0.1dB  verrauscht und anschließend ausgegeben, damit man den Unterschied zwischen Originalsignal und Verrauschtem sieht. Dabei werden nur 2 Nachkommastellen ausgegeben, um die Länge der Ausgabe zu kürzen. 

Nachdem das Signal kodiert und übertragen wurde, kann der Empfänger es nun wieder dekodieren. Dabei schickt er das empfangene Signal in die Dekodierungsfunktion, als Iterationsanzahl wird 5 gewählt. Das Ergebnis ist dann eine Liste mit den Soft- und Hard-Werten. Dabei ist zu erkennen, dass positive Soft-Werte auf eine 0 und negative auf eine 1 abgebildet werden. Bei diesem Beispiel ist schön zu sehen, dass auch ein ziemlich verrauschtes Signal vom Turbo-Dekodierer wieder hergestellt werden kann.
\section{Kodieren und Dekodieren mit Punktierung}
\label{sec:example_withPunctuation}

\section{Simulationen}
\label{sec:example_simulations}

\subsection{Turbo-Kode-Simulation}

\subsection{Kanalkodierungs-Simulation}

\subsection{Vergleich mehrerer Simulationen}

\chapter{Zusammenfassung und Ausblick}
\label{cha:result}
Das Ziel der Bachelorarbeit war die Erstellung eines R-Paketes zur Umsetzung des  Turbo-Kode-Verfahrens, welches eine Kanalkodierung darstellt. Dabei stand der didaktische Nutzen für zukünftige Studierende im Mittelpunkt. Deswegen bietet das Paket lehrreiche Visualisierungen der Kodierung und Dekodierung an. Dadurch soll das Verständnis der Verfahren erleichtert werden. Darüber hinaus werden Berichte bei den Simulationen erstellt, die eine einfache Analyse der Leistungsfähigkeit der verschiedenen Kanalkodierungsverfahren zulässt.

Die beiden alternativen Verfahren, Blockkodes und Faltungskodes, wurden von den Kollegen Nocker \cite{nocker} und Wimmer \cite{wimmer} in ihren Bachelorarbeiten umgesetzt. Alle drei Kodierungen wurden in ein gemeinsames Paket verpackt. Somit ist ein kompaktes R-Paket entstanden, das alle Kanalkodierungsverfahren beherrscht und dadurch vielseitig einsetzbar ist.  

Als Erweiterung der vorhandenen Funktionalität würde sich eine Ausweitung der Visualisierungen auf längere Nachrichten anbieten. Dadurch könnte die Darstellung auf mehr als 18 Bits erweitert werden. Das Turbo-Kode-Verfahren könnte um die Verwendung von Blockkodes ergänzt werden. Zusätzlich zur parallelen wäre die iterative Verkettung interessant, da dies der erste Ansatz der Kodeverkettung war. Als alternativer Algorithmus, bei der Dekodierung, könnte der aufwändigere BCJR-Algorithmus implementiert werden, der bei längeren Nachrichten leistungsfähiger ist, als der Viterbi-Algorithmus \cite[233-236]{schoenfeld2012informations}.

\cleardoublepage

\listofabbreviations
\clearpage

\listoffigures
\clearpage

%\printlist[tables]{lot}{}{\renewcommand\addvspace[1]{}\chapter*{Tabellenverzeichnis}}
%\clearpage

\printlist[Funktionen]{lot}{}{\renewcommand\addvspace[1]{}\chapter*{Funktionsverzeichnis}}
\clearpage

\lstlistoflistings
\clearpage

\defbibheading{myheading}[Literatur]{\chapter*{#1}}
\pagestyle{plain}
\printbibliography[heading=myheading]

\end{document}
