% !TeX spellcheck = de_DE_igerman98
\documentclass[germanthesis]{thesis-style}
% options:
% [germanthesis] - Thesis is written in German
% [plainunnumbered] - Don't print numbers on plain pages
% [earlydraft] - Settings for quick draft printouts
% [watermark] - Print current time/date at bottom of each page
% [phdthesis] - switch to PhD thesis style
% [twoside] - double sided
% [cutmargins] - text body fills complete page

% Extension to BibLaTex (useful for cites in footnotes)
%\usepackage[backend=bibtex, style=numeric-comp]{biblatex}
\bibliography{references.bib}
%\defbibheading{bibliography}[\refname]{\section*{#1}}


\author{Benedikt~Wimmer}
\title{R-Paket für Kanalkodierung mit Blockkodes}
%\title{R~package for channel~coding with block~codes}
\birthday{1. Feber 1994}
\birthplace{Sonthofen}
%\thesisstart{1. Januar 2009}
\thesistype{Bachelor's thesis}
\thesistypegerman{Bachelorarbeit}
\thesiscite{Bachelor's thesis}
\advisors{Univ.-Prof.~Dr.~Rainer~Böhme, Dr.~Pascal~Schöttle}

% additional packages
\usepackage{color, colortbl} % colors (interface table)
\usepackage{longtable}


%\usepackage[chapter]{algorithm}
%\usepackage[noend]{algpseudocode}
\makeatletter
\def\thm@space@setup{%
  \thm@preskip=\parskip \thm@postskip=0pt
}
\makeatother

% change name of caption by tables
\usepackage{caption}
\captionsetup[longtable]{name=Funktion}

%set listing language to R
\lstset{language=R, prebreak={},} 

\makeatletter
\def\ScaleIfNeeded{%
\ifdim\Gin@nat@width>\linewidth
\linewidth
\else
\Gin@nat@width
\fi
}
\makeatother

\newcommand{\plotwidth}{0.6\textwidth}
\begin{document}

% Titelseiten und Eidesstattliche Erklärung
\maketitle

% Zusammenfassung
\begin{abstract}
Diese Arbeit dokumentiert das R-Paket für Kanalkodierung mit Blockkodes und erläutert die theoretischen Grundlagen und deren Implementierung, sowie die erzeugten Visualisierungen.
\end{abstract}
% Abstract
%\begin{otherlanguage}{american}%
%\begin{abstract}
%
%\end{abstract}
%\end{otherlanguage}

\tableofcontents
\cleardoublepage
\pagenumbering{arabic}

\chapter{Einleitung}
\label{chapter:introduction}
% einleitung.tex
\section{Motivation}
Lorem Ipsum
\\
\\
\section{Ziel}




\startlist[tables]{lot}
\chapter{Grundlagen und verwandte Arbeiten}
\label{chapter:theory}
%theory.tex

\section{Verwandte Arbeiten}
\label{chapter:related}

Die Basis für theoretischen Grundlagen in dieser Arbeit basieren hauptsächlich auf 2 Werken. Die umfassendste Quelle bildet das Buch "Fundamentals of error-correcting codes" von Cary W. Huffman und Vera Pless.\cite{huffman2010fundamentals} Dort sind sehr umfassend und mathematisch sauber die Eigenschaften der meisten gängigen Kodierungen beschrieben. Im Hinblick auf die Implementierung ist aber vor allem das Buch "The art of error correcting coding" von Robert Morelos-Zaragoza\cite{morelos2006art} die Hauptquelle. Da der Autor begleitend zu diesem Buch einige Implementierungen auf der ECC-Page\cite{eccpage} veröffentlicht hat, sind auch die Grundlagen in diesem Buch etwas praxisbezogener dargestellt. Desweiteren ist Ranjan Bose's "Information Theory, Coding and Cryptography"\cite{bose2008infotheory} eine lohnende Ergänzung um das Verständnis der BCH-Kodes zu verbessern. Gerade die algebraischen Grundlagen sind dort gut erklärt.

\section{Allgemeines}
\label{sec:general}
Um die Funktionsweise der Blockkodes auf mathematischer Ebene zu erklären müssen zunächst einige Definitionen gemacht werden. Diese Eigenschaften haben alle im Folgenden genauer erklärten Kodes gemeinsam.
\newtheorem{t_def}{Definition}[chapter]

\begin{t_def}
Ein {\em Datenwort dw} wird durch die Anwendung eines {\em Kodes K} zum {\em Kodewort kw}.
\end{t_def}

\begin{t_def}
\label{def:f2}
Der binäre Körper $\mathbb{F}_{2}$ ist ein endlicher Körper $\mathbb{F}_{q}$ bestehend aus den Elementen \{0,1\}.
\end{t_def}

Definition~\ref{def:f2} entspricht auch $\mathbb{Z}_2$, dem Restklassenring $\mathbb{Z}\mod 2$. Eine andere Bezeichnung für einen endlichen Körper ist der Galoiskörper, geschrieben als GF(q) bzw. GF(2) für den binären Körper. Daraus ergibt sich für die Addition und Multiplikation:\newline

\begin{table}[!h]
\begin{center}
\begin{tabular}{c|cc}
+ & 0 & 1 \\
\hline
0 & 0 & 1 \\
%\hline
1 & 1 & 0 \\
\end{tabular}
\hspace{2cm}
\begin{tabular}{c|cc}
$\cdot$ & 0 & 1 \\
\hline
0 & 0 & 0 \\
%\hline
1 & 0 & 1 \\
\end{tabular}
\caption{Addition und Multiplikation in $\mathbb{F}_2$}
\label{table:addmul}
\end{center}
\end{table}

Sofern nicht anders angegeben beziehen sich Rechenoperationen auf Kode- und Datenwörtern immer auf den Körper $\mathbb{F}_2$. Deswegen gibt es semantisch gesehen auch keinen Unterschied zwischen Addition und Subtraktion da $\{-x = x \mid x \in \mathbb{F}_2\}$.\cite[S. 2ff]{huffman2010fundamentals}

\begin{t_def}
$\mathbb{F}_{2}^{n}$ ist der Vektorraum aller {\em n-Tupel} über dem Körper $\mathbb{F}_2$.
\end{t_def}

\begin{t_def}
\label{def:words}
Sei {\em n} die Länge der Kodewörter und {\em k} die Länge der Datenwörter. Dann sind alle Kodewörter $kw \in \mathbb{F}_{2}^{n}$ und alle Datenwörter $dw \in \mathbb{F}_{2}^{k}$.
\end{t_def}

Aus Definition~\ref{def:words} ergibt sich, dass alle Kode- und Datenwörter eine Konkatenation aus 0en und 1en sind.

\begin{t_def}[Lineare Kodes]
Ein {\em Kode K} mit Kodewörtern der Länge n ist genau dann {\em linear}, wenn er ein Untervektorraum von $\mathbb{F}_{2}^{n}$ ist.
\end{t_def}

Alle in dieser Arbeit behandelten Kodes gehören zur Klasse der \textit{linearen Kodes}.\cite[S. 3ff]{huffman2010fundamentals} Um nun deren Fehlerkorrektureigenschaften zu betrachten muss noch der Begriff der Hamming-Distanz definiert werden.

\begin{t_def}
Für 2 Kodewörtern $x,y \in \mathbb{F}_{2}^{n}$ ist die Hamming Distanz $d(x,y)$ definiert als die Anzahl der Stellen i für die gilt: $x_i \neq y_j$.
\end{t_def}

Dies bedeutet, wenn alle Kodewörter in einem Kode K eine Hamming-Distanz von $d=1$ zueinander haben, dann unterscheiden sich die Kodewörter nur in einer Stelle voneinander. Das heißt, sobald 1 Fehler passiert wird direkt ein anderes, gültiges Kodewort getroffen und der Fehler wird nicht erkannt. Auch $d=2$ reicht noch nicht aus um einen Fehler zu korrigieren. Es existieren zwar ungültige Kodewörter, diese haben aber die gleiche Distanz $d=1$ zu den benachbarten gültigen Kodewörtern. Um einen 1-Bit-Fehler zu korrigieren muss gelten: für alle Kodewörter $x,y \in C,x \neq y \colon d(x,y) \geq 3$ .\cite[S. 7ff]{huffman2010fundamentals} Daraus resultiert folgende Definition:

\begin{t_def}
\label{def:t}
Sei {\em t} die Anzahl der Fehler die ein Kode korrigieren kann. Dann muss für alle Kodewörter $x,y \in C,x \neq y$ gelten: $d(x,y) \geq 2t + 1$.
\end{t_def}


\section{Hamming-Kodes}
\label{section:hamming}

Die wohl simpelsten Blockkodes sind die Klasse der Hamming-Kodes. Ein Hamming-(n,k)-Kode erzeugt Kodewörter kw der Länge n für die gilt: $kw \in \mathbb{F}_{2}^{n}$ aus Datenwörtern dw der Länge k für die gilt: $dw \in \mathbb{F}_{2}^{k}$. Ein Hamming-Kode hat eine Hamming-Distanz von $d=3$ und kann somit mit $t=1$ einen Fehler pro Kodewort korrigieren. Gültige (n,k)-Tupel erfüllen die Gleichungen $r = n - k$ und $n = 2^r - 1$.\cite[S. 29]{huffman2010fundamentals} Die Kodierung und Dekodierung erfolgt mittels jeweils einer Generator- und Kontrollmatrix.

\begin{t_def}
\label{def:genmatrix}
Für eine Generatormatrix G eines Kodes K mit Kodewörtern kw und Datenwörtern dw gilt: $ (dw)G = kw$
\end{t_def}

\begin{t_def}
\label{def:checkmatrix}
Für eine Kontrollmatrix H eines Kodes K gilt: $ Hx^T = 0$ genau dann, wenn $x \in K$.
\end{t_def}

Definitionen \ref{def:genmatrix} und \ref{def:checkmatrix} gelten für alle lineare Kodes\cite[S. 3ff]{huffman2010fundamentals}, aber anders als bei den BCH-Kodes werden bei den Hamming-Kodes die Matrizen direkt zur Kodierung und Dekodierung verwendet.
\newblock
Alle in dieser Arbeit verwendeten Kodes sind systematisch, da dies die Nachvollziehbarkeit im Rahmen des Lehrzweckes verbessert. 

\begin{t_def}
Ein {\em systematischer} Kode enthält das Datenwort als Teil des Kodewortes.
\end{t_def}

Um einen systematischen Hamming-Kode zu erzeugen, müssen die Generatormatrix G und die Kontrollmatrix H folgende Standardform\cite[S. 8ff]{morelos2006art} haben: 

$$G = \left(I_k \mid A\right)$$
$$H = \left(A^T \mid I_r\right)$$

Um die Matrizen zu konstruieren muss man H genauer betrachten. Der Spaltenraum von H entspricht den r-Tupeln der Zahlen 1,2,...,n in Binärdarstellung. H kann nun durch Spaltenumformungen in Standardform gebracht werden. 
Anschließend kann G durch transponieren von A nach obiger Gleichung erzeugt werden.\cite[S. 29]{huffman2010fundamentals} Hier ein Beispiel für einen Hamming-(7,4)-Kode mit $r = n-k = 3$ und $I_j$ als Einheitsmatrix der Dimension $j x j$:

$$H' = \begin{pmatrix}
0 & 0 & 0 & 1 & 1 & 1 & 1 \\
0 & 1 & 1 & 0 & 0 & 1 & 1 \\
1 & 0 & 1 & 0 & 1 & 0 & 1 \\
\end{pmatrix}\text{\qquad{1,2,...,7}}$$

$$H = \begin{pmatrix}
0 & 1 & 1 & 1 & 1 & 0 & 0 \\
1 & 0 & 1 & 1 & 0 & 1 & 0 \\
1 & 1 & 0 & 1 & 0 & 0 & 1 \\
\end{pmatrix}\qquad{(A^T \mid I_r)}$$

$$G = \begin{pmatrix}
1 & 0 & 0 & 0 & 0 & 1 & 1 \\
0 & 1 & 0 & 0 & 1 & 0 & 1 \\
0 & 0 & 1 & 0 & 1 & 1 & 0 \\
0 & 0 & 0 & 1 & 1 & 1 & 1 \\
\end{pmatrix}\qquad{(I_k \mid A)}$$

Semantisch betrachtet beschreibt die Generatormatrix ein Gleichungssystem für die Paritätsbits. Die Spalten der Einheitsmatrix beschreiben die trivialen Gleichungen $dw_i = dw_i$, die Spalten von A beschreiben jeweils die Gleichung eines Paritätsbits. Für ein Datenwort $dw=(dw_a,dw_b,dw_c,dw_d)$ ergeben sich für die Paritätsbits $p=(p_e,p_f,p_g)$ des resultierenden Kodeworts $kw=(dw_a,dw_b,dw_c,dw_d,p_e,p_f,p_g)$ nach der Kodierung durch $dG = c$ folgende Gleichungen:

$$ p_e = dw_b+dw_c+dw_d$$
$$ p_f = dw_a+dw_c+dw_d$$
$$ p_g = dw_a+dw_b+dw_d$$

Stimmen alle Gleichungen, dann liegt ein gültiges Kodewort vor. Ist nur eine Gleichung falsch, liegt der Fehler im dazugehörigen Paritätsbit. Sind alle falsch, muss es $dw_d$ sein. Sind 2 falsch, so muss es das Bit sein, welches in beiden Gleichungen vorkommt. Genau diese Informationen sind in der Kontrollmatrix H. Ein Vektor des Spaltenraums von H[,i] beschreibt ein "Fehlermuster" in den Gleichungen, seine Position i das verursachende Fehlerbit im Kodewort. Eine allgemeiner Ansatz zum Dekodieren:

\begin{itemize}
\item Wenn $H(kw)^T = 0$, dann ist $kw \in K$ und damit ein gültiges Kodewort.
\item Anonsten ist an Position i für $H(kw)^T = H[ ,i]$ der Fehler passiert.
\end{itemize}

Daraus resultiert auch, dass ein 2. Fehler nicht erkannt wird, da alle möglichen r-Tupel, die bei einer Multiplikation mit H entstehen können, bereits definiert sind. 


\section{BCH-Kodes}
\label{sec:bch}

Die \textbf{B}ose, Ray-\textbf{C}haudhuri, \textbf{H}ocquenghem-Kodes bilden eine wichtige Klasse der linearen Kodes.\cite[S. 168]{huffman2010fundamentals}
Ein (n,k,d)-BCH-Kode hat n-Bit lange Kodewörter, k-Bit lange Datenwörter und eine minimale Hamming-Distanz von $d = 2t+1$.
Grundsätzlich werden hier die Daten- und Kodewörter als Polynome interpretiert. Da wir uns weiterhin auf binäre Kodes beschränken, sind die Koeffizienten der Polynome aus 
$\mathbb{F}_{2}$. Es gelten die in \ref{sec:general} definierten Eigenschaften. Daraus ergibt sich zum Beispiel $x^3 -x -1 \equiv x^3 + x + 1$. Das Datenwort 1011 wird interpretiert als $x^3 + x^2 + 1$. Der Index des Bits im Vektor entspricht dem Exponenten im Polynom, das heißt die Leserichtung ist von links nach rechts. 
Es folgen einige nötige Definitionen:

\begin{t_def}
\label{def:m}
Sei $n \in \mathbb{N}$ die Länge der Kodewörter und $m \in \mathbb{N}$ die kleinste Zahl für die gilt: $n \leq 2^m -1$, dann ist $\mathbb{F}_{2^m}$ eine Körpererweiterung des endlichen Körpers $\mathbb{F}_{2}$
\end{t_def}

Diese Körpererweiterung wird auch bezeichnet als $GF(2^m)$. 
Es existiert ein \textit{primitives Element} $\alpha \in GF(2^m)$ für das gilt: Für jedes $\beta \in GF(2^m)\setminus\{0\}$ gilt: $\beta = \alpha^j$ für $0 \leq j \leq 2^m -2$ und $n = 2^m-1$ ist die kleinste positive Zahl für die gilt: $\alpha^n = 1$. Oder anders ausgedrückt: $\alpha$ ist die {\em primitive n-te Einheitswurzel}.\cite[S. 48ff]{morelos2006art}
Nun existiert ein irreduzibles Polynom p(x) von Grad m für das gilt: $p(\alpha) = 0$, somit ist es ein {\em primitives Polynom}.

\begin{t_def}
Seien $F(x)$ alle Polynome mit Koeffizienten aus $\mathbb{F}_2$ und $p(x)$ ein primitives Polynom zu $GF(2^m)$, dann sind die Elemente von $GF(2^m)$ die Restklassen von $F(x) \mod p(x)$.  \cite[S. 161]{bose2008infotheory}
\end{t_def}

Eine weitere Darstellung der Elemente in $GF(2^m)$ ist die als binärer Vektor der Länge m. Es folgt ein Beispiel zur Veranschaulichung der 3 Darstellungen von $GF(2^m)$, den Potenzen von $\alpha$, den polynomiellen Restklassen und den korrespondierenden Binärdarstellungen. Für $m=3$ und $p(x) = x^3 + x + 1$ erhält man:


\begin{table}[!h]
\begin{center}
\begin{tabular}{c|c|c}
$\alpha^i$ & Polynom & Vektor \\
\hline
- & 0 & 000 \\
$\alpha^0$ & $1$ & 100 \\
$\alpha^1$ & $x$ & 010 \\
$\alpha^2$ & $x^2$ & 001 \\
$\alpha^3$ & $x+1$ & 110 \\
$\alpha^4$ & $x^2+x$ & 011 \\
$\alpha^5$ & $x^2+x+1$ & 111 \\
$\alpha^6$ & $x^2+1$ & 101 \\
\end{tabular}
\caption{$GF(2^3)$}
\label{table:gf}
\end{center}
\end{table}




Die Vektordarstellung ist hauptsächlich für die Implementierung relevant, da dies das Rechnen mit den Polynomen vereinfacht. Eine Addition von $a+b \mid a,b \in GF(2^m)$ entspricht $v_a \oplus v_b$ für $v_a,v_b$ in Vektordarstellung. Multiplikation entspricht einer Addition der Potenzen von $\alpha \mod 2^m-1$, also zum Beispiel $\alpha^3\alpha^5 = \alpha^8 = \alpha^1$.\cite[S. 49]{morelos2006art}

\begin{t_def}
Das Minimalpolynom $\phi_i(x)$ eines Elements $\alpha^i$ ist das Polynom kleinsten Grades mit $\phi_i(\alpha^i) = 0$.\cite[S. 50]{morelos2006art}
\end{t_def}

Um die Minimalpolynome für jedes $\alpha^i$ zu bestimmen muss man zunächst die Menge aller Zykel $C_s$ $\mod 2^m-1$ ermitteln. Diese sind für das obere Beispiel folgende:

\begin{center}
$$C_0 = \{0\}, C_1 = \{1,2,4\}, C_3 = \{3,6,5\}$$
\end{center}

\begin{t_def}
\label{def:minimal}
Für jeden Zykel s in der Menge aller Zykel ist das Minimalpolynom definiert als: $$\phi_s(x) = \prod_{i_s \in C_s}(x+\alpha^{i_s})$$\cite[S. 51]{morelos2006art}
\end{t_def}

Aus Definition~\ref{def:minimal} ergibt sich, dass jeweils die Potenzen von $\alpha$, die Elemente des selben Zykels sind, dasselbe Minimalpolynom besitzen. Dadurch können wir die Tabelle~\ref{table:gf} erweitern:

\begin{table}[!h]
\begin{center}
\begin{tabular}{c|c|c|c}
$\alpha^i$ & Polynom & Vektor & Minimalpolynom\\
\hline
- & 0 & 000 & - \\
$\alpha^0$ & $1$ & 100 & - \\
$\alpha^1$ & $x$ & 010 & $x^3+x+1$ \\
$\alpha^2$ & $x^2$ & 001  & $x^3+x+1$ \\
$\alpha^3$ & $x+1$ & 110  & $x^3+x^2+1$ \\
$\alpha^4$ & $x^2+x$ & 011  & $x^3+x+1$ \\
$\alpha^5$ & $x^2+x+1$ & 111  & $x^3+x^2+1$ \\
$\alpha^6$ & $x^2+1$ & 101  & $x^3+x^2+1$  \\
\end{tabular}
\caption{$GF(2^3)$ mit Minimalpolynomen}
\label{table:minimals}
\end{center}
\end{table}

Eine ausführliche Herleitung für Berechnung der Minimalpolynome findet sich in \cite[S. 50ff]{morelos2006art}.

\begin{t_def}
\label{def:genpoly}
Sei t die Anzahl der zu korrigierenden Fehler und LCM das kleinste gemeinsame Vielfache. Dann ist das Generatorpolynom definiert als:
$$g(x) = LCM(\phi_1(x),\phi_2(x),...,\phi_{2t}(x))$$
\end{t_def}

So ergibt sich aus Tabelle~\ref{table:minimals} und $t=1$ das Generatorpolynom:
$$g(x) = LCM(\phi_1(x),\phi_2(x)) = LCM((x^3+x+1)(x^3+x+1)) = x^3+x+1$$

Dies ist ein BCH-(7,4,3)-Kode, dieser entpsricht der zyklischen Version eines Hamming-(7,4)-Kodes. Ein Generatorpolynom für jeden binären BCH-Kode kann auf diese Weise ermittelt werden.

\begin{t_def}
\label{def:encode}
Sei dw(x) ein Datenpolynom der Länge k und g(x) das Generatorpolynom eines BCH-(n,k,d)-Kodes. Dann ist das resultierende Kodewort definiert als:
$$kw(x) = dw(x)x^{n-k} - \left(dw(x)x^{n-k} \mod g(x)\right)$$
\end{t_def}

Für den (7,4,3)-Kode aus Tabelle~\ref{table:minimals} mit Generatorpolynom $x^3+x+1$ ergibt sich folgende Kodierung für ein Datenwort $dw(x)$:

$$dw(x) = 1011$$
$$dw(x) = x^3 + x^2 + 1$$
$$dw'(x) = dw(x)x^3 = x^6 + x^5 + x^3$$
$$kw(x) = dw'(x) - \left(dw'(x) \mod g(x)\right)$$
$$kw(x) = dw'(x) + 1 = x^6 + x^5 + x^3 + 1$$
$$kw(x) = 1001011$$

Um die Schritte nachzurechnen sollte man stets die Eigenschaften der Koeffizienten aus $\mathbb{F}_2$, definiert in Tabelle~\ref{table:addmul}, berücksichtigen.
Wie man sieht handelt es sich auch hier um ein systematisches Kodewort, das Datenwort ist darin enthalten.
\newblock
\newline
Das Dekodieren der BCH-Kodes kann auf verschiedene Arten erfolgen und ist sehr detailliert in Robert Morelos-Zaragoza's "The Art of Error Correcting Coding"(Kapitel 3.5) beschrieben.\cite{morelos2006art}
An dieser Stelle folgt eine kurze Zusammenfassung der relevanten Schritte.
Die Dekodierung der BCH-Kodes erfolgt im allgemeinen in 4 Stufen:

\begin{enumerate}
\item Berechnen der Syndrome $S_i = kw(\alpha^i) \mid 1 \leq i \leq 2t$.
\item Bestimmem des Fehlerstellenpolynoms $\sigma(x)$.
\item Ermitteln der Inversen der Nullstellen von $\sigma(x)$, diese entsprechen den Fehlerindizes $\alpha^i$.
\item Korrigieren der Fehler im Kodewort an den Fehlerindizes.
\end{enumerate}

Die Syndrome sind die Auswertung des erhaltenen Kodeworts $kw(x)$ an den Nullstellen $\alpha^{1,...,2t}$. 
Für das Kodewort $kw(x) = 1001011 = x^6 + x^5 + x^3 + 1$, $t=1$ und $GF(2^3)$ aus Tabelle~\ref{table:gf} ergeben sich folgende Syndrome:

$$S_1 = kw(\alpha^1) = \alpha^6 + \alpha^5 + \alpha^3 + 1 = (x^2 +1) + (x^2+x+1)(x+1)+1 = 0$$
$$S_2 = kw(\alpha^2) = \alpha^{12} + \alpha^{10} + \alpha^6 + 1 = \alpha^5 + \alpha^3 + \alpha^6 + 1 = 0 $$

Da das Kodewort keine Fehler enthielt sind die Syndrome $S_i = 0$. Wenn es nur darum geht Fehler zu erkennen wäre dieser Schritt bereits ausreichend. Für ein Fehlerhaftes Kodewort $kw'(x) = 1101011 = x^6 + x^5 + x^3 + x+1$ ergeben sich die Syndrome: 
$$S_1 = 0 + \alpha^1 = x$$
$$S_2 = 0 + \alpha^2 = x^2$$

Um nun die Fehlerindizes zu bestimmen muss zunächst das Fehlerstellenpolynom $\sigma(x)$ mit Hilfe des Berlekamp-Massey-Algorithmus ermittelt werden. Dieser ist in \cite[S. 59ff]{morelos2006art} beschrieben. Für das Fehlerhafte Kodewort $kw'(x)$ liefert dieser $\sigma(x) = \alpha x +1$. Ist der Grad von $\sigma(x) > t$ sind mehr als t Fehler im Kodewort und es kann nicht vollständig dekodiert werden. In diesem Fall ist $deg(\sigma(x)) = t$, also müssen nun die Inversen der Nullstellen von $\sigma(x)$ bestimmt werden. Das verwendete Verfahren nennt sich Chien's Suche, dabei wird für alle $\beta \in GF(2^m)\setminus\{0\}$ die Bedingung $\sigma(\beta^{-1}) = 0$ überprüft.\cite[S. 63]{morelos2006art} Für $\sigma(x) = \alpha x +1$ ergibt sich:

$$\sigma(\alpha^6) = \alpha^1\alpha^6 +1 = \alpha^7 +1 = \alpha^0 + 1 = 1 + 1 = 0$$

Da $\alpha^{-6} = \alpha^1$ ist der Fehler in $kw'(x) = 1101011 = x^6 + x^5 + x^3 + x+1$ an Index 1 identifiziert.

$$kw'(x) - x = x^6 + x^5 + x^3 + x+1 -x = x^6+x^5+x^3+1=kw(x)$$



\subsection{Reed-Solomon-Kodes}

Die Klasse der Reed-Solomon-Kodes, kurz RS-Kodes, findet viele Anwendungen im Bereich der digitalen Datenspeicherung auf CDs und DVDs, sowie in Kommunikationssystemen.\cite[S. 73]{morelos2006art} Die hierfür verwendeten Kodes sind allerdings nicht mehr binär und daher für die Simulation in dieser Arbeit nicht verwendbar. Die binären RS-Kodes bilden eine Unterklasse der binären BCH-Kodes mit einer Einschränkung: Die Länge der Kodewörter ist $n = 2^m-1$.\cite[S. 173]{huffman2010fundamentals} Dadurch ist der einzige Unterschied im binären, dass BCH-Kodes flexibler in der Kodelänge sind. Gerade für nicht-binäre RS-Kodes ergeben sich aber Vereinfachungen im Bestimmen des Generatorpolynoms und dem Dekodieren, dies ist in \cite[S. 73ff]{morelos2006art} und \cite[S. 173ff]{huffman2010fundamentals} gut beschrieben.

\stoplist[tables]{lot}

\startlist[Funktionen]{lot}
\chapter{Implementierung}
\label{chapter:implementation}
% implementierung.tex
Dieses Kapitel gibt einen Einblick in die Konzepte der Implementierung. Als Einstiegspunkt stand eine Referenzimplementierung\footnote{\url{http://vashe.org/turbo/turbo_example.c} (01.06.2016)} zur Verfügung, die den Dekodier-Algorithmus für Turbo-Kodes beinhaltet, jedoch für ein konkretes Beispiel. Dieser musste angepasst werden um für allgemeine Faltungskodes verwendbar zu sein.
\\
\\
Kapitel \ref{kapitel:implementierung_faltungskodierer} beinhaltet den Entwurf der Faltungskodierer-Datenstruktur. [TODO: Fertigstellung]
%Die Implementierung der Kodierung wird in Kapitel \ref{kapitel:implementierung_kodierung} beschrieben, die der Dekodierung in Kapitel \ref{kapitel:implementierung_dekodierung}.
%\\
%\\
%Aus Performancegründen Kodierung, Dekodierung in C++\\
%Weiters: Kodierer erzeugen, Depunktierung, Katastrophale Kodierer Prüfung (Polynom GGT mod 2)
%\\
%Parameterprüfung, Punktierung, ApplyNoise, Aufruf Visualisierung in R\\
%Referenzimplementierung
\section{Faltungskodierer}
\label{kapitel:implementierung_faltungskodierer}
Ein Faltungskodierer ist gegeben durch 
\begin{itemize}
\item $N$: Anzahl an Ausgangsbits je Eingangsbit,
\item $M$: Länge des Schieberegisters,
\item $G$: Vektor von Generatorpolynomen.
\end{itemize}
Die Angabe von $M$ ist hier redundant, jedoch Teil der Benutzereingabe zur Generierung eines Faltungskodierers, welche durch \cite{morelos2006art} inspiriert wurde.
\\
\\
Zur leichteren Implementierung der Kodierung und Dekodierung wird die Kodierer-Datenstruktur um folgende Elemente erweitert:
\begin{itemize}
\item eine \emph{Zustandsübergangsmatrix}, die angibt, in welchen Zustand der Kodierer bei einem Eingangsbit wechselt,
\item eine \emph{inverse Zustandsübergangsmatrix}, die angibt, aus welchem Zustand der Kodierer bei einem Eingangsbit kommt,
\item eine \emph{Outputmatrix}, die angibt, welche Kodebits der Kodierer bei einem Eingangsbit in einem bestimmten Zustand ausgibt,
\item ein Flag zur Markierung rekursiver systematischer Kodierer (RSC),
\item ein \emph{Terminierungsvektor} die für rekursiver systematische Kodierer angibt, ob ein Eingangsbit 0 oder 1 in einem bestimmten Zustand für die Terminierung zu verwenden ist.
\end{itemize}
Die Implementierung der Matrizen wurde aus der Referenzimplementierung übernommen, musste jedoch erweitert werden, um für allgemeine Faltungskodes verwendbar zu sein. Für alle gilt, die Anzahl an Zeilen entspricht der Anzahl an Zuständen. Der Zeilenindex entspricht dem Zustand. Die Zustandsübergangsmatrix sowie die Outputmatrix besitzen jeweils zwei Spalten. Je eine Spalte steht für ein Eingangsbit (0 oder 1), wobei der Spaltenindex dem Eingangsbit entspricht. Die inverse Zustandsübergangsmatrix benötigt eine dritte Spalte. Für viele Kodierer (v.a. nicht-rekursive) tritt der Fall ein, dass nur durch \emph{ein bestimmtes} Eingangsbit in einen bestimmten Zustand gewechselt werden kann. Sei ein Zustand bspw. nur durch das Eingangsbit 0 erreichbar, so bedeutet das, dass es für diesen Zustand mit dem Bit 0 \emph{zwei} Vorgängerzustände gibt, für ein Eingangsbit 1 jedoch keinen Vorgänger. Diese zweite Möglichkeit wird in der dritten Spalte gespeichert.
\\
\\
Der Terminierungsvektor ist für nicht-rekursive Kodierer nicht notwendig, da ein Kode eines solchen Kodierers immer mit $M$ 0-Bits terminiert wird. Bei einem rekursiven Kodierer ist es nicht trivial zu sagen mit welchem Eingangsbit in einem bestimmten Zustand terminiert wird, um den Kodierer in den Nullzustand zu bringen. Dies hängt von der Definition des Rekursionpolynoms ab. Der Terminierungsvektor wird bei der Erzeugung rekursiver Kodierer berechnet.
\\
\\
Bei der Erzeugung von Faltungskodierern ist zu prüfen ob es sich um einen katastrophalen Kodierer handelt. RSC-Kodierer sind, wie in Kapitel \ref{kapitel:grundlagen_systematische_kodierer} beschrieben, nicht zu prüfen. Zur Prüfung wird nach Theorem \ref{thm:massey} der größte gemeinsame Teiler der Generatorpolynome berechnet. Die Berechnung des größten gemeinsamen Teilers wurde mithilfe des des euklidschen Algorithmus implementiert. Sowohl der euklidsche Algorithmus als auch die dafür notwendige binäre Polynomdivision wird an eine C++ Funktion delegiert.

\section{Kodierung}
\label{kapitel:implementierung_kodierung}
Bei Faltungskodes stellt die Kodierung den bei Weitem einfacheren Teil dar. Es muss lediglich jedes Bit der zu kodierenden Nachricht zusammen mit dem aktuellen Zustand, der nach jedem Bit mithilfe der Zustandsübergangsmatrix aktualisiert wird, auf die Outputmatrix angewendet werden. Die Terminierung funktioniert analog, einzig das zu kodierende Bit muss ermittelt werden. Für RSC-Kodierer muss im Terminierungsvektor nachgeschaut werden, andernfalls ist das Bit immer 0. Abgeschlossen wird die Kodierung mit dem Abbilden der Kodebits 0 bzw. 1 auf die Signalwerte +1 bzw. -1. Algorithmus \ref{algorithmus:kodierung} zeigt den Kodierungsalgorithmus.

\begin{algorithm}[H]
\renewcommand{\algorithmicforall}{\textbf{for each}}
\caption{Faltungskodierung}
\label{algorithmus:kodierung}
\begin{algorithmic}[1]
\STATE state $=0$, code $=$ result $=$ " "
\FORALL {bit \textbf{in} message}
   \STATE output $=$ output.matrix[state][bit]
	\STATE code $=$ $concat($code, output$)$
	\STATE state $=$ state.transition.matrix$[$state$][$bit$]$
\ENDFOR
\IF{terminate code}
   \FOR {$i=0$ \TO $M-1$}
      \STATE termination.bit $=$ rsc-coder $?$ termination.vector$[$state$]$ : $0$
      \STATE output $=$ output.matrix$[$state$][$termination.bit$]$
	   \STATE code $=$ $concat($code, output$)$
	   \STATE state $=$ state.transition.matrix$[$state$][$termination.bit$]$
   \ENDFOR
\ENDIF
\FORALL {bit \textbf{in} code}
   \STATE signal $=1-2$bit
   \STATE result $=$ $concat($result,signal$)$
\ENDFOR
\RETURN result
\end{algorithmic}
\end{algorithm}

\section{Dekodierung}
\label{kapitel:implementierung_dekodierung}
Die Dekodierung stellt den wesentlich komplexeren Teil der Faltungskodes dar. 

\begin{algorithm}[H]
\renewcommand{\algorithmicforall}{\textbf{for each}}
\caption{Faltungsdekodierung}
\label{algorithmus:dekodierung}
\begin{algorithmic}[1]
\STATE $NUM_STATES=2^{M}$
\FOR {$t=1$ \TO $length($message$)$}
   \FOR {$s=0$ \TO $NUM_STATES-1$}
   	\STATE $m_{1}=$ metric[t-1][prev.state1] + $\delta_{1}$
   	\STATE $m_{2}=$ metric[t-1][prev.state2] + $\delta_{2}$
      \STATE metric[t][s] = $min(m_{1},m_{2})$
      \STATE survivor.bit $=$ $xyz(0,1,min(min(m_{1},m_{2}))$
	\ENDFOR
\ENDFOR
\RETURN result
\end{algorithmic}
\end{algorithm}

\section{Rauschen}
\label{kapitel:implementierung_noise}
Um auch zeigen zu können, dass die Dekodierung auch tatsächlich für verrauschte Signale funktioniert, benötigt es eine Funktion, die die Übertragung einer Nachricht über einen verrauschten Kanal simuliert, d.h. das Signal mit Rauschen überlagert. Zum Signal soll ein additives weißes gaußsches Rauschen (AWGR oder AWGN\footnote{additive white Gaussian noise}) addiert werden um dieses zu verfälschen. [apply noise quelle] stellt eine alternative Implementierung zur eingebauten AWGN-Funktion in Matlab vor. Die Implementierung wurde übernommen bzw. nach R übersetzt. Durch die Möglichkeit das Signal-Rausch-Verhältnis über einen Parameter zu steuern, können verschiedene Übertragungskanäle simuliert und Nachrichten somit verschieden stark verrauscht werden. Der Benutzer kann dadurch herausfinden, ab wann eine Nachricht zu viel Rauschen enthält, um sie korrekt dekodieren zu können. Weiters kann nach mehrfacher Ausführung auf Fehlermuster geschlossen werden, mit denen die Dekodierung gut bzw. schlecht umgehen kann. Durch Versuche mit anderen Kanalkodierungs-Methoden können Vergleiche mit diesen angestellt werden. Die genannten Punkte helfen dem Benutzer sein Verständnis für Faltungskodes noch besser zu stärken.

\section{Punktierung}
\label{kapitel:implementierung_punktierung}
Punktierung leicht, jedoch Depunktierung vor der Dekodierung nicht trivial.

%\section{Visualisierung}
%\label{kapitel:implementierung_visualisierung}
\stoplist[Funktionen]{lot}

\chapter{Visualisierung}
\label{chapter:visualization}
% visualisierung.tex
LIMITS!\\\\
%Da das R-Paket für Studenten zu Lernzwecken verwendet werden soll
Um das Verständnis für Faltungskodes beim Benutzer dieses R-Pakets zu stärken, stehen Visualisierungen der Kodierung und Dekodierung zur Verfügung.
\\
Wird der \texttt{visualize} Parameter bei der Ausführung einer Funktion zur Kodierung oder Dekodierung auf TRUE~\texttt{TRUE} gesetzt, wird ein R Markdown Skript ausgeführt. Dieses generiert eine Beamer Präsentation mit Informationen und Visualisierungen zur Kodierung bzw. Dekodierung.
\\\\
\textcolor{lightgray}{\ref{chapter:visualisierung}.1 Kodierung}\\
% allg. Informationen
Bei der Kodierung befinden sich auf den ersten Folien allgemeine Informationen zum verwendeten Faltungskodierer wie die Kode-Rate, Generatorpolynome, Zustandsübergangstabelle etc. 
% Kodierung
Daraufhin folgt die Kodierungsvisualisierung. Diese zeigt zunächst die zu kodierende Nachricht (Input), das Zustandsübergangsdiagramm sowie eine noch nicht befüllte Kodierungstabelle. Um für einen noch besseren Lerneffekt zu sorgen wird Schritt für Schritt mittels Overlays ein Bit des Inputs, der aktuelle Zustand, Folgezustand sowie der resultierende Output in eine neue Zeile der Kodierungstabelle geschrieben. Der aktuelle Zustand sowie der entsprechende Übergang werden im Diagramm farblich hervorgehoben. Die kodierte Nachricht wächst mit jedem Schritt bis schlussendlich die gesamte Nachricht kodiert wurde.
% Kode zu Signal
Da die Kodierungsfunktion nicht die Bitwerte des Kodeworts zurückliefert sondern die Signalwerte (für eine Übertragung über einen Kanal) wird auf einer weiteren Folie dargestellt, wie die Kodebits in Signalwerte überführt werden.\\
% Punktierung
Wird eine Punktierungsmatrix bei der Kodierung mitgegeben, wird eine zusätzliche Folie am Ende hinzugefügt. Auf dieser wird die Punktierung des Signals, d.h. das Entfernen von Signalwerten (definiert durch die Punktierungsmatrix) dargestellt. Dabei wird neben dem originalen Signal und der Punktierungsmatrix das punktierte Signal dargestellt, wobei zunächst die punktierten Signalwerte, d.h. die entfernten Werte, durch Asterisk-Symbole (\textasteriskcentered) ersetzt werden. Diese Darstellung dient als visueller Zwischenschritt für das danach folgende tatsächlich punktierte Signal, bei dem die punktierten Werte fehlen, was auch dem Rückgabewert der Funktion entspricht.
\\\\
\textcolor{lightgray}{\ref{chapter:visualisierung}.2 Dekodierung}\\
% allg. Informationen
Bei der Dekodierung befinden sich ebenfalls, wie bei der Kodierung, allgemeine Informationen des Faltungskodierers auf den ersten Folien.
% Signal zu Kode
Als Input erhält die Dekodierung das Kodewort als Signalwerte, die möglicherweise durch Anwendung der \texttt{ApplyNoise} Funktion verfälscht worden sind. Die soft decision Dekodierung verwendet zur Dekodierung zwar kontinuierliche Signalwerte, da aber sowohl die hard decision Dekodierung Bitwerde zur Dekodierung verwendet und Trellis-Diagramme mit Bitwerten beschriftet werden, wird auf einer Folie die Überführung der Signalwerte zu Bits dargestellt. Dieser transformierte Input wird auch als Input für die Visualisierung des Viterbi-Algorithmus verwendet.
% Viterbi-Algorithmus
Anschließend folgt die Visualisierung des Viterbi-Algorithmus mithilfe des Trellis-Diagramms. Zunächst werden, zur besseren Übersicht bei großen Diagrammen, jene Pfade entfernt, für die es eine bessere Alternative gibt, d.h. die eine größere Metrik bei hard decision Dekodierung bzw. eine kleinere Metrik bei soft decision Dekodierung als ihre Alternative haben. Danach erfolgt Schritt für Schritt mittels Backtracking die Rekonstruktion der Nachricht. Der gewählte Pfad beim Backtracking wird farblich hervorgehoben. Die übrigen Pfade werden ausgegraut. Am Ende befindet sich unter dem Trellis-Diagramm die farblich hervorgehobene dekodierte Nachricht.
% Punktierung
Wird eine Punktierungsmatrix bei der Dekodierung mitgegeben, wird eine zusätzliche Folie nach den Kodiererinformationen hinzugefügt. Auf dieser wird die Depunktierung des Signals, d.h. das Einfügen des Signalwerts 0 (definiert durch die Punktierungsmatrix), dargestellt. Die eingefügten 0-Werte sind zur leichteren visuellen Erkennung farblich hervorgehoben.
\\\\
\textcolor{lightgray}{\ref{chapter:visualisierung}.3 Simulation}\\
Weiters können Berichte der Simulation generiert werden, die die resultierenden Daten u.a. in einem Diagramm darstellen.

\chapter{Beispiele für die Verwendung}
\label{chapter:examples}
\section{Erzeugen von Kodierer, Permutationsvektor und Punktierungsmatrix}
\label{sec:example_createHelpers}
Zuerst werden Hilfsvariablen erzeugt, die für die folgende Kodierungs- und Dekodierungsverfahren benötigt werden. Diese Variablen müssen nicht unbedingt mitgegebenen werden, dann werden die Standard-Werte verwendet. Das bedeutet, dass falls keine Punktierungsmatrix übergeben wird, der Standard-Wert \emph{NULL} ist und darum nicht punktiert wird. Ebenfalls wird nicht permutiert, wenn kein Permutationsvektor als Argument übergeben wird. Es wird zwar ein Permutationsvektor benötigt, damit die Interleaver richtig arbeiten können, jedoch wird ein Vektor vom Typ \emph{PRIMITIVE} verwendet, dieser allerdings ändert die Reihenfolge der Bits nicht (root=0). Beim Kodierer verhält es sich ähnlich, jedoch wird hier ein vordefinierter Standard-Faltungskodierer verwendet (\emph{ConvGenerateRscEncoder(2,2,c(5,7))}), dieser wird in allen Turbo-Kode-Funktionen benützt, falls kein eigener Kodierer übergeben wird.

\begin{lstlisting}[caption=Erzeugung von Kodierer und Punktierungsmatrix, label={lst:createHelpersCoderPunctuation}]
input <- c(1,0,1,1,0)

coder <- ConvGenerateEncoder(2, 2, c(4,5))

punctuation.matrix <- TurboGetPunctuationMatrix(c(1,1,0,0,1,1))
\end{lstlisting}

Wie im Listing \ref{lst:createHelpersCoderPunctuation} zu sehen ist, wird als allererstes eine zu kodierende Nachricht erzeugt, die dann benötigt wird, die dann benötigt wird um eine Punktierungsmatrix zu bekommen. Anschließend wird ein nicht rekursiver Faltungskodierer, mit der Funktion vom Faltungskode-Teil des Packetes \cite{nocker}, erzeugt. Dabei wurde ein Kodierer mit 2 Registern und 2 Ausgängen gewählt. Wichtig ist, dass der erste Ausgang vom Eingang durchgeschalten wird damit es ein systematischer Kodierer ist. Das wird erreicht, indem das 1.Generatorpolynom mit $2^M = 2^2 = 4$ deklariert wird.

\begin{lstlisting}[caption=Erzeugung von verschiedenen Permutationsvektoren, label={lst:createHelpersPermutation}]
permutation.vector.random <- TurboGetPermutation(length(input), coder, "RANDOM")
permutation.vector.primitve <- TurboGetPermutation(length(input), coder, "PRIMITIVE", list(root=3))

input2 <- c(1,0,1,1,0,1)
permutation.vector.cyclic <- TurboGetPermutation(length(input2), coder, "CYCLIC", list(cols=4, rows=2, distance=2))
permutation.vector.block <- TurboGetPermutation(length(input2), coder, "BLOCK", list(cols=4, rows=2))
permutation.vector.helical <- TurboGetPermutation(length(input2), coder, "HELICAL", list(cols=4, rows=2))
permutation.vector.diagonal <- TurboGetPermutation(length(input2), coder, "DIAGONAL", list(cols=4, rows=2), TRUE)
[1] "Initial-Matrix"
     [,1] [,2] [,3] [,4]
[1,]    0    1    2    3
[2,]    4    5    6    7
[1] "Interleaver-Vector:  DIAGONAL"
[1] 0 5 1 6 2 7 3 4
\end{lstlisting}

Als nächstes benötigt der Benutzer noch einen Permutationsvektor, dieser wird im Listing \ref{lst:createHelpersPermutation} erzeugt. Dabei wird jeder Typ einmal verwendet. Bei den ersten beiden werden die Eingangsbits von Listing \ref{lst:createHelpersCoderPunctuation} benützt. Da bei den restlichen Typen eine Matrix benötigt wird, muss die Länge der Eingangsbits genau in einer Matrix Platz finden, deshalb wurde ein 2. Eingangsvektor definiert, der allerdings nur als Demonstration für die anderen Permutationstypen dienen sollte, dieser wird in den weiteren Beispielen nicht weiter verwendet. Es kann auch das Visualisierungsflag auf \emph{TRUE} gesetzt werden, dann wird die Ausgangsmatrix und der resultierende Permutationsvektor dargestellt. Dies wurde bei dem letzten Beispiel mit dem Typ \emph{DIAGONAL} demonstrativ verwendet.

\section{Kodieren und Dekodieren ohne Punktierung}
\label{sec:example_withoutPunctuation}
Nachdem alle benötigten Variablen gesetzt wurden kann nun die Kodierung und Dekodierung vorgenommen werden. Dabei wird zuerst ohne Punktierung gearbeitet, jedoch alle anderen zuvor definierten Variablen verwendet.

\begin{lstlisting}[caption=Kodierung und Dekodierung ohne Punktierung, label={lst:encodeDecodeWithoutPunctuation}]
encoded <- TurboEncode(input, permutation.vector.random, coder, 2)
encoded
[1] -1 -1 -1  1  1  1 -1  1  1 -1 -1  1  1 -1 -1  1 -1 -1  1  1  1

encoded.noisy <- ApplyNoise(encoded, 0.1)
round(encoded.noisy, 2)
[1] -2.41 -2.36 -0.26  1.06  1.37  0.22  1.13  0.93  0.33 -1.15 -1.18  0.59
[13]  1.32 -0.31 -1.37 -0.01 -1.42 -1.62  2.18  2.29  0.93

decoded <- TurboDecode(encoded.noisy, permutation.vector.random, 5, coder, 2)
decoded
$output.soft
[1]  -1.394185  1.394185  2.775904 -1.394185  2.775904

$output.hard
[1] 1 0 1 1 0
\end{lstlisting}

Der gesamte Vorgang lässt sich in wenigen Zeilen erledigen, wie in Listing \ref{lst:encodeDecodeWithoutPunctuation} zu sehen ist. Zuerst werden einfach die zuvor definierten Variablen der Kodierungsfunktion mitgegeben, als letzter Parameter wird noch der Index des zu verwendeten Ausgangs des Kodierers angegeben (2 in diesem Fall). Danach erhält man das kodierte Signal mit dem Signalpegel 1 und -1. Dieses Signal wird dann einem Signal/Rausch-Verhältnis von 0.1dB  verrauscht und anschließend ausgegeben, damit man den Unterschied zwischen Originalsignal und Verrauschtem sieht. Dabei werden nur 2 Nachkommastellen ausgegeben, um die Länge der Ausgabe zu kürzen. 

Nachdem das Signal kodiert und übertragen wurde, kann der Empfänger es nun wieder dekodieren. Dabei schickt er das empfangene Signal in die Dekodierungsfunktion, als Iterationsanzahl wird 5 gewählt. Das Ergebnis ist dann eine Liste mit den Soft- und Hard-Werten. Dabei ist zu erkennen, dass positive Soft-Werte auf eine 0 und negative auf eine 1 abgebildet werden. Bei diesem Beispiel ist schön zu sehen, dass auch ein ziemlich verrauschtes Signal vom Turbo-Dekodierer wieder hergestellt werden kann.
\section{Kodieren und Dekodieren mit Punktierung}
\label{sec:example_withPunctuation}

\section{Simulationen}
\label{sec:example_simulations}

\subsection{Turbo-Kode-Simulation}

\subsection{Kanalkodierungs-Simulation}

\subsection{Vergleich mehrerer Simulationen}

\chapter{Fazit und Ausblick}
\label{chapter:conclusion}
\input{chapters/conclusion}

\cleardoublepage
\phantomsection
\addcontentsline{toc}{chapter}{\listfigurename}
\listoffigures
\cleardoublepage

\phantomsection
\addcontentsline{toc}{chapter}{Tabellenverzeichnis}
\printlist[tables]{lot}{}{\renewcommand\addvspace[1]{}\chapter*{Tabellenverzeichnis}}
\cleardoublepage

\phantomsection
\addcontentsline{toc}{chapter}{Funktionsverzeichnis}
\printlist[Funktionen]{lot}{}{\renewcommand\addvspace[1]{}\chapter*{Funktionsverzeichnis}}
\cleardoublepage

\phantomsection
\addcontentsline{toc}{chapter}{Listingverzeichnis}
\lstlistoflistings
\cleardoublepage
\phantomsection

\defbibheading{myheading}[Literatur]{%
  \chapter*{#1}}
\addcontentsline{toc}{chapter}{Literatur}
\pagestyle{plain}
\printbibliography[heading=myheading]

\end{document}
