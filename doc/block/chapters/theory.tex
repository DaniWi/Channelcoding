%theory.tex

\section{Verwandte Arbeiten}
\label{chapter:related}

Die Basis für theoretischen Grundlagen in dieser Arbeit basieren hauptsächlich auf 2 Werken. Die umfassendste Quelle bildet das Buch "Fundamentals of error-correcting codes" von Cary W. Huffman und Vera Pless.\cite{huffman2010fundamentals} Dort sind sehr umfassend und mathematisch sauber die Eigenschaften der meisten gängigen Kodierungen beschrieben. Im Hinblick auf die Implementierung ist aber vor allem das Buch "The art of error correcting coding" von Robert Morelos-Zaragoza\cite{morelos2006art} die Hauptquelle. Da der Autor begleitend zu diesem Buch einige Implementierungen auf der ECC-Page\cite{eccpage} veröffentlicht hat, sind auch die Grundlagen in diesem Buch etwas praxisbezogener dargestellt. Desweiteren ist Ranjan Bose's "Information Theory, Coding and Cryptography"\cite{bose2008infotheory} eine lohnende Ergänzung um das Verständnis der BCH-Kodes zu verbessern. Gerade die algebraischen Grundlagen sind dort gut erklärt.

\section{Allgemeines}
\label{sec:general}
Um die Funktionsweise der Blockkodes auf mathematischer Ebene zu erklären müssen zunächst einige Definitionen gemacht werden. Diese Eigenschaften haben alle im Folgenden genauer erklärten Kodes gemeinsam.
\newtheorem{t_def}{Definition}[chapter]

\begin{t_def}
Ein {\em Datenwort dw} wird durch die Anwendung eines {\em Kodes K} zum {\em Kodewort kw}.
\end{t_def}

\begin{t_def}
\label{def:f2}
Der binäre Körper $\mathbb{F}_{2}$ ist ein endlicher Körper $\mathbb{F}_{q}$ bestehend aus den Elementen \{0,1\}.
\end{t_def}

Definition~\ref{def:f2} entspricht auch $\mathbb{Z}_2$, dem Restklassenring $\mathbb{Z}\mod 2$. Eine andere Bezeichnung für einen endlichen Körper ist der Galoiskörper, geschrieben als GF(q) bzw. GF(2) für den binären Körper. Daraus ergibt sich für die Addition und Multiplikation:\newline

\begin{table}[!h]
\begin{center}
\begin{tabular}{c|cc}
+ & 0 & 1 \\
\hline
0 & 0 & 1 \\
%\hline
1 & 1 & 0 \\
\end{tabular}
\hspace{2cm}
\begin{tabular}{c|cc}
$\cdot$ & 0 & 1 \\
\hline
0 & 0 & 0 \\
%\hline
1 & 0 & 1 \\
\end{tabular}
\caption{Addition und Multiplikation in $\mathbb{F}_2$}
\label{table:addmul}
\end{center}
\end{table}

Sofern nicht anders angegeben beziehen sich Rechenoperationen auf Kode- und Datenwörtern immer auf den Körper $\mathbb{F}_2$. Deswegen gibt es semantisch gesehen auch keinen Unterschied zwischen Addition und Subtraktion da $\{-x = x \mid x \in \mathbb{F}_2\}$.\cite[S. 2ff]{huffman2010fundamentals}

\begin{t_def}
$\mathbb{F}_{2}^{n}$ ist der Vektorraum aller {\em n-Tupel} über dem Körper $\mathbb{F}_2$.
\end{t_def}

\begin{t_def}
\label{def:words}
Sei {\em n} die Länge der Kodewörter und {\em k} die Länge der Datenwörter. Dann sind alle Kodewörter $kw \in \mathbb{F}_{2}^{n}$ und alle Datenwörter $dw \in \mathbb{F}_{2}^{k}$.
\end{t_def}

Aus Definition~\ref{def:words} ergibt sich, dass alle Kode- und Datenwörter eine Konkatenation aus 0en und 1en sind.


\begin{t_def}[Lineare Kodes]
Ein {\em Kode K} mit Kodewörtern der Länge n ist genau dann {\em linear}, wenn er ein Untervektorraum von $\mathbb{F}_{2}^{n}$ ist.
\end{t_def}

Alle in dieser Arbeit behandelten Kodes gehören zur Klasse der \textit{linearen Kodes}.\cite[S. 3ff]{huffman2010fundamentals} Um nun deren Fehlerkorrektureigenschaften zu betrachten muss noch der Begriff der Hamming-Distanz definiert werden.

\begin{t_def}
Für 2 Kodewörtern $x,y \in \mathbb{F}_{2}^{n}$ ist die Hamming Distanz $d(x,y)$ definiert als die Anzahl der Stellen i für die gilt: $x_i \neq y_j$.
\end{t_def}

Dies bedeutet, wenn alle Kodewörter in einem Kode K eine Hamming-Distanz von $d=1$ zueinander haben, dann unterscheiden sich die Kodewörter nur in einer Stelle voneinander. Das heißt, sobald 1 Fehler passiert wird direkt ein anderes, gültiges Kodewort getroffen und der Fehler wird nicht erkannt. Auch $d=2$ reicht noch nicht aus um einen Fehler zu korrigieren. Es existieren zwar ungültige Kodewörter, diese haben aber die gleiche Distanz $d=1$ zu den benachbarten gültigen Kodewörtern. Um einen 1-Bit-Fehler zu korrigieren muss gelten: für alle Kodewörter $x,y \in C,x \neq y \colon d(x,y) \geq 3$ .\cite[S. 7ff]{huffman2010fundamentals} Daraus resultiert folgende Definition:

\begin{t_def}
Sei {\em t} die Anzahl der Fehler die ein Kode korrigieren kann. Dann muss für alle Kodewörter $x,y \in C,x \neq y$ gelten: $d(x,y) \geq 2t + 1$.
\end{t_def}


\section{Hamming-Kodes}
\label{section:hamming}

Die wohl simpelsten Blockkodes sind die Klasse der Hamming-Kodes. Ein Hamming-(n,k)-Kode erzeugt Kodewörter kw der Länge n für die gilt: $kw \in \mathbb{F}_{2}^{n}$ aus Datenwörtern dw der Länge k für die gilt: $dw \in \mathbb{F}_{2}^{k}$. Ein Hamming-Kode hat eine Hamming-Distanz von $d=3$ und kann somit mit $t=1$ einen Fehler pro Kodewort korrigieren. Gültige (n,k)-Tupel erfüllen die Gleichungen $r = n - k$ und $n = 2^r - 1$.\cite[S. 29]{huffman2010fundamentals} Die Kodierung und Dekodierung erfolgt mittels jeweils einer Generator- und Kontrollmatrix.

\begin{t_def}
\label{def:genmatrix}
Für eine Generatormatrix G eines Kodes K mit Kodewörtern kw und Datenwörtern dw gilt: $ (dw)G = kw$
\end{t_def}

\begin{t_def}
\label{def:checkmatrix}
Für eine Kontrollmatrix H eines Kodes K gilt: $ Hx^T = 0$ genau dann, wenn $x \in K$.
\end{t_def}

Definitionen \ref{def:genmatrix} und \ref{def:checkmatrix} gelten für alle lineare Kodes\cite[S. 3ff]{huffman2010fundamentals}, aber anders als bei den BCH-Kodes werden bei den Hamming-Kodes die Matrizen direkt zur Kodierung und Dekodierung verwendet.
\newblock
Alle in dieser Arbeit verwendeten Kodes sind systematisch, da dies die Nachvollziehbarkeit im Rahmen des Lehrzweckes verbessert. 

\begin{t_def}
Ein {\em systematischer} Kode enthält das Datenwort als Teil des Kodewortes.
\end{t_def}

Um einen systematischen Hamming-Kode zu erzeugen, müssen die Generatormatrix G und die Kontrollmatrix H folgende Standardform\cite[S. 8ff]{morelos2006art} haben: 

$$G = \left(I_k \mid A\right)$$
$$H = \left(A^T \mid I_r\right)$$

Um die Matrizen zu konstruieren muss man H genauer betrachten. Der Spaltenraum von H entspricht den r-Tupeln der Zahlen 1,2,...,n in Binärdarstellung. H kann nun durch Spaltenumformungen in Standardform gebracht werden. 
Anschließend kann G durch transponieren von A nach obiger Gleichung erzeugt werden.\cite[S. 29]{huffman2010fundamentals} Hier ein Beispiel für einen Hamming-(7,4)-Kode mit $r = n-k = 3$ und $I_j$ als Einheitsmatrix der Dimension $j x j$:

$$H' = \begin{pmatrix}
0 & 0 & 0 & 1 & 1 & 1 & 1 \\
0 & 1 & 1 & 0 & 0 & 1 & 1 \\
1 & 0 & 1 & 0 & 1 & 0 & 1 \\
\end{pmatrix}\text{\qquad{1,2,...,7}}$$

$$H = \begin{pmatrix}
0 & 1 & 1 & 1 & 1 & 0 & 0 \\
1 & 0 & 1 & 1 & 0 & 1 & 0 \\
1 & 1 & 0 & 1 & 0 & 0 & 1 \\
\end{pmatrix}\qquad{(A^T \mid I_r)}$$

$$G = \begin{pmatrix}
1 & 0 & 0 & 0 & 0 & 1 & 1 \\
0 & 1 & 0 & 0 & 1 & 0 & 1 \\
0 & 0 & 1 & 0 & 1 & 1 & 0 \\
0 & 0 & 0 & 1 & 1 & 1 & 1 \\
\end{pmatrix}\qquad{(I_k \mid A)}$$

Semantisch betrachtet beschreibt die Generatormatrix ein Gleichungssystem für die Paritätsbits. Die Spalten der Einheitsmatrix beschreiben die trivialen Gleichungen $dw_i = dw_i$, die Spalten von A beschreiben jeweils die Gleichung eines Paritätsbits. Für ein Datenwort $dw=(dw_a,dw_b,dw_c,dw_d)$ ergeben sich für die Paritätsbits $p=(p_e,p_f,p_g)$ des resultierenden Kodeworts $kw=(dw_a,dw_b,dw_c,dw_d,p_e,p_f,p_g)$ nach der Kodierung durch $dG = c$ folgende Gleichungen:

$$ p_e = dw_b+dw_c+dw_d$$
$$ p_f = dw_a+dw_c+dw_d$$
$$ p_g = dw_a+dw_b+dw_d$$

Stimmen alle Gleichungen, dann liegt ein gültiges Kodewort vor. Ist nur eine Gleichung falsch, muss es das dazugehörige Paritätsbit sein. Sind alle falsch, muss es d sein. Sind 2 falsch, dann muss das Bit sein welches in beiden Gleichungen vorkommt. Genau diese Informationen sind in der Kontrollmatrix H. Ein Vektor des Spaltenraums von H[,i] beschreibt ein "Fehlermuster" in den Gleichungen, seine Position i das verursachende Fehlerbit im Kodewort. Eine allgemeiner Ansatz zum Dekodieren:

\begin{itemize}
\item Wenn $H(kw)^T = 0$, dann ist $kw \in K$ und damit ein gültiges Kodewort.
\item Anonsten ist an Position i für $H(kw)^T = H[ ,i]$ der Fehler passiert.
\end{itemize}

Daraus resultiert auch, dass ein 2. Fehler nicht erkannt wird, da alle möglichen r-Tupel, die bei einer Multiplikation mit H entstehen können, bereits definiert sind. 


\section{BCH-Kodes}
\label{sec:bch}

Die \textbf{B}ose, Ray-\textbf{C}haudhuri, \textbf{H}ocquenghem-Kodes bilden eine wichtige Klasse der linearen Kodes.\cite[S. 168]{huffman2010fundamentals}
Ein (n,k,d)-BCH-Kode hat n-Bit lange Kodewörter, k-Bit lange Datenwörter und eine minimale Hamming-Distanz von $d = 2t+1$.
Grundsätzlich werden hier die Daten- und Kodewörter als Polynome interpretiert. Da wir uns weiterhin auf binäre Kodes beschränken sind die Koeffizienten der Polynome aus 
$\mathbb{F}_{2}$, es gelten die in \ref{sec:general} definierten Eigenschaften. Daraus ergibt sich zum Beispiel $x^3 -x -1 \equiv x^3 + x + 1$. Das Datenwort 1011 wird interpretiert als $x^3 + x^2 + 1$. Grundsätzlich entspricht der Index des Bits im Vektor dem Exponenten im Polynom, das heißt die Leserichtung ist von links nach rechts. 
Es folgen einige nötige Definitionen:

\begin{t_def}
Sei $n \in \mathbb{N}$ die Länge der Kodewörter und $m \in \mathbb{N}$ die kleinste Zahl für die gilt: $n \leq 2^m -1$, dann ist $\mathbb{F}_{2^m}$ eine Körpererweiterung des endlichen Körpers $\mathbb{F}_{2}$
\end{t_def}

Diese Körpererweiterung wird auch bezeichnet als $GF(2^m)$. 
Es existiert ein \textit{primitives Element} $\alpha \in GF(2^m)$ für das gilt: Für jedes $\beta \in GF(2^m)\setminus\{0\}$ gilt: $\beta = \alpha^j$ für $0 \leq j \leq 2^m -2$ und $n = 2^m-1$ ist die kleinste positive Zahl für die gilt: $\alpha^n = 1$. Oder anders ausgedrückt: $\alpha$ ist die {\em primitive n-te Einheitswurzel}.\cite[S. 48ff]{morelos2006art}
Nun existiert ein irreduzibles Polynom p(x) von Grad m für das gilt: $p(\alpha) = 0$, somit ist es ein {\em primitives Polynom}.

\begin{t_def}
Seien $F(x)$ alle Polynome mit Koeffizienten aus $\mathbb{F}_2$ und $p(x)$ ein primitives Polynom zu $GF(2^m)$, dann sind die Elemente von $GF(2^m)$ die Restklassen von $F(x) \mod p(x)$.  \cite[S. 161]{bose2008infotheory}
\end{t_def}

Eine weitere Darstellung der Elemente in $GF(2^m)$ ist die als binärer Vektor der Länge m. Es folgt ein Beispiel zur Veranschaulichung der 3 Darstellungen von $GF(2^m)$, den Potenzen von $\alpha$, den polynomiellen Restklassen und den korrespondierenden Binärdarstellungen. Für $m=3$ und $p(x) = x^3 + x + 1$ erhält man:


\begin{table}[!h]
\begin{center}
\begin{tabular}{c|c|c}
$\alpha^i$ & Polynom & Vektor \\
\hline
- & 0 & 000 \\
$\alpha^0$ & $1$ & 100 \\
$\alpha^1$ & $x$ & 010 \\
$\alpha^2$ & $x^2$ & 001 \\
$\alpha^3$ & $x+1$ & 110 \\
$\alpha^4$ & $x^2+x$ & 011 \\
$\alpha^5$ & $x^2+x+1$ & 111 \\
$\alpha^6$ & $x^2+1$ & 101 \\
\end{tabular}
\caption{$GF(2^3)$}
\label{table:gf}
\end{center}
\end{table}




Die Vektordarstellung ist hauptsächlich für die Implementierung relevant, da dies das Rechnen mit den Polynomen vereinfacht. Eine Addition von $a+b \mid a,b \in GF(2^m)$ entspricht $v_a \oplus v_b$ für $v_a,v_b$ in Vektordarstellung. Multiplikation entspricht einer Addition der Potenzen von $\alpha \mod 2^m-1$, also zum Beispiel $\alpha^3\alpha^5 = \alpha^8 = \alpha^1$.\cite[S. 49]{morelos2006art}

\begin{t_def}
Das Minimalpolynom $\phi_i(x)$ eines Elements $\alpha^i$ ist das Polynom kleinsten Grades mit $\phi_i(\alpha^i) = 0$.\cite[S. 50]{morelos2006art}
\end{t_def}

Um die Minimalpolynome für jedes $\alpha^i$ zu bestimmen muss man zunächst die Menge aller Zykel $C_s$ $\mod 2^m-1$ ermitteln. Diese sind für das obere Beispiel folgende:

\begin{center}
$$C_0 = \{0\}, C_1 = \{1,2,4\}, C_3 = \{3,6,5\}$$
\end{center}

\begin{t_def}
\label{def:minimal}
Für jeden Zykel s in der Menge aller Zykel ist das Minimalpolynom definiert als: $$\phi_s(x) = \prod_{i_s \in C_s}(x+\alpha^{i_s})$$\cite[S. 51]{morelos2006art}
\end{t_def}

Aus Definition~\ref{def:minimal} ergibt sich, dass jeweils die Potenzen von $\alpha$, die Elemente des selben Zykels sind, dasselbe Minimalpolynom besitzen. Dadurch können wir die Tabelle~\ref{table:gf} erweitern:

\begin{table}[!h]
\begin{center}
\begin{tabular}{c|c|c|c}
$\alpha^i$ & Polynom & Vektor & Minimalpolynom\\
\hline
- & 0 & 000 & - \\
$\alpha^0$ & $1$ & 100 & - \\
$\alpha^1$ & $x$ & 010 & $x^3+x+1$ \\
$\alpha^2$ & $x^2$ & 001  & $x^3+x+1$ \\
$\alpha^3$ & $x+1$ & 110  & $x^3+x^2+1$ \\
$\alpha^4$ & $x^2+x$ & 011  & $x^3+x+1$ \\
$\alpha^5$ & $x^2+x+1$ & 111  & $x^3+x^2+1$ \\
$\alpha^6$ & $x^2+1$ & 101  & $x^3+x^2+1$  \\
\end{tabular}
\caption{$GF(2^3)$ mit Minimalpolynomen}
\label{table:minimals}
\end{center}
\end{table}

Eine ausführliche Herleitung für Berechnung der Minimalpolynome findet sich in \cite[S. 50ff]{morelos2006art}.

\begin{t_def}
\label{def:genpoly}
Sei t die Anzahl der zu korrigierenden Fehler und LCM das kleinste gemeinsame Vielfache. Dann ist das Generatorpolynom definiert als:
$$g(x) = LCM(\phi_1(x),\phi_2(x),...,\phi_{2t}(x))$$
\end{t_def}

So ergibt sich aus Tabelle~\ref{table:minimals} und $t=1$ das Generatorpolynom:
$$g(x) = LCM(\phi_1(x),\phi_2(x)) = LCM((x^3+x+1)(x^3+x+1)) = x^3+x+1$$

Dies ist ein BCH-(7,4,3)-Kode, es entpsricht der zyklischen Version eines Hamming-(7,4)-Kodes. Ein Generatorpolynom für jeden binären BCH-Kode kann auf diese Weise ermittelt werden.

\begin{t_def}
\label{def:encode}
Sei dw(x) ein Datenpolynom der Länge k und g(x) das Generatorpolynom eines BCH-(n,k,d)-Kodes. Dann ist das resultierende Kodewort definiert als:
$$kw(x) = dw(x)x^{n-k} - \left(dw(x)x^{n-k} \mod g(x)\right)$$
\end{t_def}

Für den (7,4,3)-Kode aus Tabelle~\ref{table:minimals} mit Generatorpolynom $x^3+x+1$ ergibt sich folgende Kodierung für ein Datenwort $dw(x)$:

$$dw(x) = 1011$$
$$dw(x) = x^3 + x^2 + 1$$
$$dw'(x) = dw(x)x^3 = x^6 + x^5 + x^3$$
$$kw(x) = dw'(x) - \left(dw'(x) \mod g(x)\right)$$
$$kw(x) = dw'(x) + 1 = x^6 + x^5 + x^3 + 1$$
$$kw(x) = 1001011$$

Um die Schritte nachzurechnen sollte man stets die Eigenschaften der Koeffizienten aus $\mathbb{F}_2$, definiert in Tabelle~\ref{table:addmul}, berücksichtigen.
Wie man sieht handelt es sich auch hier um ein systematisches Kodewort, das Datenwort ist darin enthalten.
\newblock
\newline
Das Dekodieren der BCH-Kodes kann auf verschiedene Arten erfolgen und ist sehr detailliert in Robert Morelos-Zaragoza's "The Art of Error Correcting Coding"(Kapitel 3.5) beschrieben.\cite{morelos2006art}
An dieser Stelle folgt eine kurze Zusammenfassung der relevanten Schritte.
Die Dekodierung der BCH-Kodes erfolgt im allgemeinen in 4 Stufen:

\begin{enumerate}
\item Berechnen der Syndrome $S_i = kw(\alpha^i) \mid 1 \leq i \leq 2t$
\item Bestimmem des Fehlerstellenpolynoms $\sigma(x)$.
\item Ermitteln der Inversen der Nullstellen von $\sigma(x)$, diese entsprechen den Fehlerindizes $\alpha^i$
\item Korrigieren der Fehler im Kodewort und den Fehlerindizes.
\end{enumerate}

Die Syndrome sind die Auswertung des erhaltenen Kodeworts $kw(x)$ an den Nullstellen $\alpha^{1,...,2t}$. 
Für das Kodewort $kw(x) = 1001011 = x^6 + x^5 + x^3 + 1$, $t=1$ und $GF(2^3)$ aus Tabelle~\ref{table:gf} ergeben sich folgende Syndrome:

$$S_1 = kw(\alpha^1) = \alpha^6 + \alpha^5 + \alpha^3 + 1 = (x^2 +1) + (x^2+x+1)(x+1)+1 = 0$$
$$S_2 = kw(\alpha^2) = \alpha^{12} + \alpha^{10} + \alpha^6 + 1 = \alpha^5 + \alpha^3 + \alpha^6 + 1 = 0 $$

Da das Kodewort keine Fehler enthielt sind die Syndrome $S_i = 0$. Wenn es nur darum geht Fehler zu erkennen wäre dieser Schritt bereits ausreichend. Für ein Fehlerhaftes Kodewort $kw'(x) = 1101011 = x^6 + x^5 + x^3 + x+1$ ergeben sich die Syndrome: 
$$S_1 = 0 + \alpha^1 = x$$
$$S_2 = 0 + \alpha^2 = x^2$$

Um nun die Fehlerindizes zu bestimmen muss zunächst das Fehlerstellenpolynom $\sigma(x)$ mit Hilfe des Berlekamp-Massey-Algorithmus ermittelt werden. Dieser ist in \cite[S. 59ff]{morelos2006art} beschrieben. Für das Fehlerhafte Kodewort $kw'(x)$ liefert dieser $\sigma(x) = \alpha x +1$. Ist der Grad von $\sigma(x) > t$ sind mehr als t Fehler im Kodewort und es kann nicht vollständig dekodiert werden. In diesem Fall ist $deg(\sigma(x)) = t$, also müssen nun die Inversen der Nullstellen von $\sigma(x)$ bestimmt werden. Das verwendete Verfahren nennt sich Chien's Suche, dabei wird für alle $\beta \in GF(2^m)\setminus\{0\}$ die Bedingung $\sigma(\beta^{-1}) = 0$ überprüft.\cite[S. 63]{morelos2006art} Für $\sigma(x) = \alpha x +1$ ergibt sich:

$$\sigma(\alpha^6) = \alpha^1\alpha^6 +1 = \alpha^7 +1 = \alpha^0 + 1 = 1 + 1 = 0$$

Da $\alpha^{-6} = \alpha^1$ ist der Fehler in $kw'(x) = 1101011 = x^6 + x^5 + x^3 + x+1$ an Index 1 identifiziert.

$$kw'(x) - x = x^6 + x^5 + x^3 + x+1 -x = x^6+x^5+x^3+1=kw(x)$$



\subsection{Reed-Solomon-Kodes}

Die Klasse der Reed-Solomon-Kodes, kurz RS-Kodes, findet viele Anwendungen im Bereich der digitalen Datenspeicherung auf CDs und DVDs, sowie in Kommunikationssystemen.\cite[S. 73]{morelos2006art} Die hierfür verwendeten Kodes sind allerdings nicht mehr binär und daher für die Simulation in dieser Arbeit nicht verwendbar. Die binären RS-Kodes bilden eine Unterklasse der binären BCH-Kodes mit einer Einschränkung: Die Länge der Kodewörter ist $n = 2^m-1$.\cite[S. 173]{huffman2010fundamentals} Dadurch ist der einzige Unterschied im binären, dass BCH-Kodes flexibler in der Kodelänge sind. Gerade für nicht-binäre RS-Kodes ergeben sich aber Vereinfachungen im Bestimmen des Generatorpolynoms und dem Dekodieren, dies ist in \cite[S. 73ff]{morelos2006art} und \cite[S. 173ff]{huffman2010fundamentals} gut beschrieben.
