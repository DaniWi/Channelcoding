% ConvSimulation.tex
\begin{longtable}{|p{\textwidth}|}
\hline
\rowcolor{lightblue}ConvSimulation\\
\hline
\\
\texttt{ConvSimulation(conv.coder, msg.length, min.db, max.db, db.interval, iterations.per.db, punctuation.matrix, visualize)}\\
\\
Simulation of a convolutional encode and decode process over a noisy channel at several signal-noise-ratios (SNR).\\
\\
\textbf{Arguments:}\\
\texttt{conv.coder} - Convolutional coder used for the simulation. Can be created via ConvGenerateEncoder or ConvGenerateRscEncoder.\\
\texttt{msg.length} - Message length of the randomly created messages to be encoded. Default: 100\\
\texttt{min.db} - Minimum SNR to be tested. Default: 0.1\\
\texttt{max.db} - Maximum SNR to be tested. Default: 2.0\\
\texttt{db.interval} - Step between two SNRs tested. Default: 0.1\\
\texttt{iterations.per.db} - Number of encode and decode processes per SNR. Default: 100\\
\texttt{punctuation.matrix} - If not null the process involves the punctuation. Can be created via ConvGetPunctuationMatrix. Default: NULL\\
\texttt{visualize} - If true a PDF report is generated. Default: FALSE\\
\\
\textbf{Returns:}\\
Dataframe containing the bit-error-rates for each SNR tested.\\
\\
\hline
\end{longtable}