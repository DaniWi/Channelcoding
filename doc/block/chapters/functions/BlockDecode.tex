\begin{longtable}{|p{\textwidth}|}
\hline
\rowcolor{lightblue}BlockDecode\\
\hline
\\
\texttt{BlockDecode(code, block.encoder, visualize)}\\
\\
Dekodiert die Nachricht mithilfe des in \texttt{block.encoder} angegebenen Blockkodierers. Die zu dekodierende Nachricht wird dazu in Blöcke der Kodewortlänge des Kodierers aufgeteilt und ggf. 0en angehängt um alle Blöcke gleich lang zu machen. \\
\\
\textbf{Argumente:}\\
\texttt{code} - Der zu dekodierende Blockkode, als binärer Vektor.\\
\texttt{block.encoder} - Blockkodierer der mit den Funktionen \emph{BlockGenerateEncoderHamming, BlockGenerateEncoderBCH} erzeugt werden kann. Standard: \emph{BlockGenerateEncoderBCH(15,3)}\\
\texttt{visualize} - Wenn TRUE wird ein Visualisierungs-PDF erstellt. Standard: FALSE\\
\\
\textbf{Rückgabewert:}\\
Die dekodierte Nachricht als binärer Vektor, Fehler sind soweit möglich korrigiert.\\
\\
\hline
\caption{BlockDecode}
\end{longtable}