% BlockOpenPDF.tex
\begin{longtable}{|p{\textwidth}|}
\hline
\rowcolor{lightblue}
BlockOpenPDF
\\
\hline
\\
\texttt{BlockOpenPDF(encode, hamming, simulation)}\\
\\
Öffnet die mit \texttt{BlockEncode}, \texttt{BlockDecode} und \texttt{BlockSimulation} erzeugten PDF-Berichte.\\
\\
\textbf{Argumente:}\\
\texttt{encode} - Markiert ob Kodierungsbericht (\texttt{TRUE}) oder Dekodierungsbericht (\texttt{FALSE}) geöffnet wird. Standard: \texttt{TRUE}\\
\texttt{hamming} - Markiert ob Hamming-Bericht (\texttt{TRUE}) oder BCH-Bericht (\texttt{FALSE}) geöffnet werden. Standard: \texttt{FALSE}\\
\texttt{simulation} - Markiert ob Simulationsbericht geöffnet wird. Standard: \texttt{FALSE}\\
\\
\hline
\caption{BlockOpenPDF}
\end{longtable}