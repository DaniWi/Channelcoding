% einleitung.tex
In jeder Sekunde werden Daten auf verschiedenste Arten und Weisen gesendet, gelesen und geschrieben. Dabei bleibt leider nicht jedes Bit fehlerfrei, denn wäre dem so gäbe es keinen Grund für Kanalkodierung. Gründe hierfür können beim Senden von Daten zum Beispiel Störfrequenzen, kosmische Strahlung, schlechte Übertragungskanäle und vieles mehr sein. Auch greifbarere Ursachen, wie in etwa Kratzer auf einer CD, haben eine zwangsläufige Korruption von Daten zur Folge. Als Reaktion hierauf wurden schon seit Beginn der modernen Kommunikationstechnik Kodierungen zur Fehlerkorrektur entwickelt. Heute gibt es eine große Zahl unterschiedlicher Kodierungen, die auf verschiedene Anwendungsbereiche angepasst sind. In der Praxis erfolgt die Kodierung und Dekodierung für einen bestimmten Kanal jedoch meist direkt in der Hardware, da dies natürlich deutlich schneller als in der Software ist und sich die Parameter einer Kodierung innerhalb eines Anwendungsfalles kaum ändern. Dies hat allerdings zur Folge, dass die Lehre im Bereich der Kanalkodierung sich hauptsächlich auf die theoretischen Grundlagen beschränkt. Deshalb ist das Ziel dieser Arbeit ein R-Paket für die Kanalkodierung mit Blockkodes zu erstellen. So entsteht gemeinsam mit den Arbeiten von Martin Nocker, "R-Paket für Kanalkodierung mit Faltungskodes"~\cite{nocker}, und Daniel Witsch, "R-Paket für Kanalkodierung mit Turbo-Kodes"~\cite{witsch}, ein R-Paket für die Simulation der Kanalkodierung. Dieses soll vor allem für Lehrzwecke eingesetzt werden können. Das beinhaltet einerseits die Möglichkeit für Studenten verschiedene Kodierungen anzuwenden und deren Fehlerkorrektureigenschaften zu vergleichen und zu visualisieren. Andererseits gibt es auch die Möglichkeit für die Lehrenden das Paket im Rahmen einer Lehrveranstaltung einzubinden. Zum Beispiel kann der Output der Visualisierungen und Simulationen verwendet werden, oder es können Übungsaufgaben mit dem Paket gestellt werden. Die zur Benutzung des Pakets notwendigen Informationen sind bereits in der Dokumentation enthalten. Diese Arbeit soll ein weiterführendes Verständnis über die Funktionsweise der verwendeten Kodierungen, deren Umsetzung in der Implementierung und die Entstehung der Visualisierungen vermitteln. 