% einleitung.tex
Kanalkodierung stellt einen wichtigen Teil der Nachrichtentechnik dar. Kanalkodierung stellt Methoden zur Verfügung, um Fehler, die während der Übertragung über einen verrauschten Kanal auftreten, zu korrigieren. Die Leistungsfähikeit und Zuverlässigkeit vieler digitaler Systeme basiert auf der Verwendung von Kanalkodierung. Eine Art der Kanalkodierung stellen Faltungskodes dar, auf welche sich diese Arbeit konzentriert. Verwendung finden Faltungskodes in der Mobil- und Satellitenkommunikation aber vor allem bilden sie die Basis für Turbokodes, welche die Faltungskodes mittlerweile aufgrund ihrer noch höheren Leistungsfähigkeit abgelöst haben.
\\
\\
Ziel dieser Arbeit ist die Implementierung von Faltungskodes mithilfe der Programmiersprache R. Das entwickelte R-Paket dient zukünftigen Studenten zu Lernzwecken und soll sie beim Verstehen von Faltungskodes unterstützen.
\\
\\
Studierenden soll dadurch, neben den theoretischen Grundlagen im Zuge einer Lehrveranstaltung, eine praktische Anwendungsmöglichkeit für Faltungskodes geboten werden. 