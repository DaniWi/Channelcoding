% einleitung.tex
Kanalkodierung stellt, aufgrund der stetigen Ausbreitung digitaler Kommunikationssysteme, ein wichtiges Teilgebiet der Kodierungstheorie dar. Informationen, die über einen Kanal zwischen Quelle und Empfänger übertragen werden, können aufgrund von Rauschen verändert werden. Die Kanalkodierung stellt Methoden zur Verfügung, um Fehler, die während der Übertragung über einen verrauschten Kanal auftreten, zu korrigieren. Verwendung findet die Kanalkodierung in der Mobil- und Satellitenkommunikation, da beispielsweise das erneute Senden von Satellitendaten, bei Auftreten von Rauschen, aufgrund der Laufzeiten unpraktisch wäre. Weiters wird die Kanalkodierung zur Speicherung von Daten, etwa auf Compact-Disks, verwendet. Rauschen kann durch thermische Störungen bzw. Kratzer oder Fingerabdrücke auf Compact-Disks hervorgerufen werden.

Eine Art der Kanalkodierung stellen Faltungskodes dar, auf welche sich diese Arbeit konzentriert. Die wichtigste Anwendung von Faltungskodes ist die Konstruktion von Turbo-Kodes, welche eine Erweiterung der Faltungskodes darstellen~\cite{huffman2010fundamentals}. Darüber hinaus existiert mit den Blockkodes ein weiteres wichtiges Verfahren der Kanalkodierung.
\\
\\
Ziel dieser Arbeit ist die Implementierung von Faltungskodes mithilfe der Programmiersprache R. Das entwickelte R-Paket dient zukünftigen Studierenden zu Lehrzwecken und soll sie beim Studieren von Faltungskodes unterstützen. Studierenden soll dadurch, neben den theoretischen Grundlagen im Zuge einer Vorlesung, eine praktische Möglichkeit zur Anwendung von Faltungskodes geboten werden. Mithilfe dynamisch generierter Visualisierungen sollen die Prinzipien von Faltungskodes vermittelt werden.

Die Arbeiten \enquote{R-Paket für Kanalkodierung mit Blockkodes}~\cite{wimmer} von Benedikt Wimmer und \enquote{R-Paket für Kanalkodierung mit Turbo-Kodes}~\cite{witsch} von Daniel Witsch stellen Implementierungen der Blockkodes bzw. Turbo-Kodes zur Verfügung, die ebenfalls im Paket enthalten sind. Dadurch ergibt sich ein sowohl umfangreiches, als auch kompaktes Paket zur Kanalkodierung.

Diese Bachelorarbeit gliedert sich in folgende Kapitel: Die theoretischen Grundlagen der Kanalkodierung und Faltungskodierung werden in Kapitel~\ref{kapitel:grundlagen} diskutiert. Kapitel~\ref{kapitel:technologien} erläutert die verwendeten Technologien bei der Umsetzung des praktischen Teils dieser Arbeit. In Kapitel~\ref{kapitel:implementierung} wird die Implementierung der Faltungskodierung in R beschrieben, gefolgt von der Erklärung der Paket-Schnittstelle in Kapitel~\ref{kapitel:interface}. Die dynamisch generierten Visualisierungen werden in Kapitel~\ref{kapitel:visualisierung} veranschaulicht. Eine Einführung in die Verwendung des Pakets wird mithilfe der Beispiele in Kapitel~\ref{kapitel:beispiele} gegeben. Kapitel~\ref{kapitel:fazit} schließt diese Arbeit mit Vorschlägen zur Erweiterung des Pakets ab.