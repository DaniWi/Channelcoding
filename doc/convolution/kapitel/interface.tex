% R Paket Schnittstelle für Faltungskodierung
\definecolor{lightblue}{RGB}{150,180,255}

In diesem Kapitel wird die Schnittstelle für den Benutzer erläutert. Kapitel~\ref{kapitel:interface_faltungskodierung} listet Funktionen zur Erzeugung von Faltungskodierern, der Kodierung, Dekodierung und Simulation von Faltungskodes. Kapitel~\ref{kapitel:interface_hilfsfunktionen} beinhaltet Hilfsfunktionen für Faltungskodes. Schließlich beschreibt Kapitel~\ref{kapitel:interface_kanalkodierung} weitere nützliche Funktionen der Kanalkodierung.

% Faltungskodierung Funktionen
\section{Faltungskodierung}
\label{kapitel:interface_faltungskodierung}

\subsection{ConvGenerateEncoder}
% ConvGenerateEncoder
\begin{longtable}{|p{\textwidth}|}
\hline
\rowcolor{lightblue}
ConvGenerateEncoder
\\
\hline
\\
\texttt{ConvGenerateEncoder(N, M, generators)}\\
\\
Erzeugt einen Faltungskodierer für nichtrekursive Faltungskodes.\\
\\
\textbf{Argumente:}\\
\texttt{N} - Anzahl an Ausgangssymbole je Eingangssymbol.\\
\texttt{M} - Länge des Schieberegisters des Kodierers.\\
\texttt{generators} - Vektor der N oktale Generatorpolynome enthält (ein Polynom je Ausgangssymbol, siehe Kapitel~\ref{kapitel:grundlagen_darstellung}).\\
\\
\textbf{Rückgabewert:}\\
Faltungskodierer, abgebildet als Liste mit folgenden Feldern:
\vspace{-4mm}
\begin{itemize}
\renewcommand\labelitemi{--}
\itemsep-.5em % spacing between items
\item \emph{N}: Anzahl an Ausgangssymbole je Eingangssymbol
\item \emph{M}: Länge des Schieberegisters des Kodierers
\item \emph{generators}: Generatorpolynomvektor
\item \emph{next.state}: Zustandsübergangsmatrix
\item \emph{prev.state}: inverse Zustandsübergangsmatrix
\item \emph{output}: Ausgabematrix
\item \emph{rsc}: RSC-Flag (\texttt{FALSE})
\item \emph{termination}: Terminierungsvektor (\texttt{logical(0)})
\end{itemize}
\\
\hline
\caption{ConvGenerateEncoder}
\end{longtable}

\subsection{ConvGenerateRscEncoder}
% ConvGenerateRscEncoder.tex
\begin{longtable}{|p{\textwidth}|}
\hline
\rowcolor{lightblue}
ConvGenerateRscEncoder
\\
\hline
\\
\texttt{ConvGenerateRscEncoder(N, M, generators)}\\
\\
Erzeugt einen Faltungskodierer für rekursiv systematische Faltungskodes (RSC-Kodierer).\\
\\
\textbf{Argumente:}\\
\texttt{N} - Anzahl an Ausgangssymbole je Eingangssymbol.\\
\texttt{M} - Länge des Schieberegisters des Kodierers.\\
\texttt{generators} - Vektor der oktale Generatorpolynome enthält (ein Polynom je nicht-systematischen Ausgang und ein Polynom für die Rekursion, siehe Kapitel~\ref{kapitel:grundlagen_rsc}).\\
\\
\textbf{Rückgabewert:}\\
Faltungskodierer, abgebildet als Liste mit folgenden Feldern:
\vspace{-4mm}
\begin{itemize}
\renewcommand\labelitemi{--}
\itemsep-.5em % spacing between items
\item \emph{N}: Anzahl an Ausgangssymbole je Eingangssymbol
\item \emph{M}: Länge des Schieberegisters des Kodierers
\item \emph{generators}: Generatorpolynomvektor
\item \emph{next.state}: Zustandsübergangsmatrix
\item \emph{prev.state}: inverse Zustandsübergangsmatrix
\item \emph{output}: Ausgabematrix
\item \emph{rsc}: RSC-Flag (\texttt{TRUE})
\item \emph{termination}: Terminierungsvektor
\end{itemize}
\\
\hline
\caption{ConvGenerateRscEncoder}
\label{funktion:ConvGenerateRscEncoder}
\end{longtable}

\subsection{ConvEncode}
% ConvEncode.tex
\begin{longtable}{|p{\textwidth}|}
\hline
\rowcolor{lightblue}
ConvEncode
\\
\hline
\\
\texttt{ConvEncode(message, conv.encoder, terminate, punctuation.matrix, visualize)}\\
\\
Erzeugt einen Faltungskode aus einer unkodierten Nachricht.\\
\\
\textbf{Argumente:}\\
\texttt{message} - Nachricht die kodiert wird.\\
\texttt{conv.encoder} - Faltungskodierer der für die Kodierung verwendet wird.\\
\texttt{terminate} - Markiert ob der Kode terminiert werden soll. Standard: \texttt{TRUE}\\
\texttt{punctuation.matrix} - Wenn ungleich \texttt{NULL} wird die kodierte Nachricht mit der Punktierungsmatrix punktiert. Standard: \texttt{NULL}\\
\texttt{visualize} - Wenn \texttt{TRUE} wird ein PDF-Bericht der Kodierung erstellt. Standard: \texttt{FALSE}\\
\\
\textbf{Rückgabewert:}\\
Die kodierte Nachricht mit den Signalwerten +1 und -1 welche die Bits 0 und 1 darstellen. Falls punktiert wurde eine Liste mit dem Originalkode (nicht punktiert) und dem punktiertem Kode.\\
\\
\hline
\caption{ConvEncode}
\label{funktion:ConvEncode}
\end{longtable}

\subsection{ConvDecodeSoft}
% ConvDecodeSoft.tex
\begin{longtable}{|p{\textwidth}|}
\hline
\rowcolor{lightblue}ConvDecodeSoft\\
\hline
\\
\texttt{ConvDecodeSoft(code, conv.encoder, terminate, punctuation.matrix, visualize)}\\
\\
Decodes a convolutional codeword using soft decision decoding.\\
\\
\textbf{Arguments:}\\
\texttt{code} - The code to be decoded (soft input).\\
\texttt{conv.encoder} - Convolutional encoder used for encoding.\\
\texttt{terminate} - Flag if the code is terminated. Default: TRUE\\
\texttt{punctuation.matrix} - If not NULL the code is depunctured prior to the decode algorithm. Default: NULL\\
\texttt{visualize} - If TRUE a beamer PDF file is generated showing the decode process. Default: FALSE\\
\\
\textbf{Returns:}\\
The decoded message as a list containing the soft output-values and hard output-values.\\
\\
\hline
\end{longtable}

\subsection{ConvDecodeHard}
% ConvDecodeHard.tex
\begin{longtable}{|p{\textwidth}|}
\hline
\rowcolor{lightblue}
ConvDecodeHard
\\
\hline
\\
\texttt{ConvDecodeHard(code, conv.encoder, terminate, punctuation.matrix, visualize)}\\
\\
Dekodiert einen Faltungskode mittels hard decision Dekodierung.\\
\\
\textbf{Argumente:}\\
\texttt{code} - Faltungskode der dekodiert wird.\\
\texttt{conv.encoder} - Faltungskodierer der für die Kodierung verwendet wurde.\\
\texttt{terminate} - Markiert ob der Kode terminiert ist. Standard: \texttt{TRUE}\\
\texttt{punctuation.matrix} - Wenn ungleich \texttt{NULL} wird der Kode vor der Dekodierung depunktiert. Standard: \texttt{NULL}\\
\texttt{visualize} - Wenn \texttt{TRUE} wird ein PDF-Bericht der Dekodierung erstellt. Standard: \texttt{FALSE}\\
\\
\textbf{Rückgabewert:}\\
Vektor der Dekodierten Nachricht.\\
\\
\hline
\caption{ConvDecodeHard}
\end{longtable}

\subsection{ConvSimulation}
% ConvSimulation.tex
\begin{longtable}{|p{\textwidth}|}
\hline
\rowcolor{lightblue}
ConvSimulation
\\
\hline
\\
\texttt{ConvSimulation(conv.coder, msg.length, min.db, max.db, db.interval, iterations.per.db, punctuation.matrix, visualize)}\\
\\
Simulation einer Faltungskodierung und -dekodierung nach einer Übertragung über einen verrauschten Kanal mit verschiedenen Signal-Rausch-Verhältnissen (SNR).\\
\\
\textbf{Argumente:}\\
\texttt{conv.coder} - Faltungskodierer der für die Simulation verwendet wird. Kann mittels \texttt{ConvGenerateEncoder} oder \texttt{ConvGenerateRscEncoder} erzeugt werden.\\
\texttt{msg.length} - Nachrichtenlänge der zufällig generierten Nachrichten. Standard: 100\\
\texttt{min.db} - Untergrenze der getesteten SNR. Standard: 0.1\\
\texttt{max.db} - Obergrenze der getesteten SNR. Standard: 2.0\\
\texttt{db.interval} - Schrittweite zwischen zwei getesteten SNR. Standard: 0.1\\
\texttt{iterations.per.db} - Anzahl der Iterationen (Kodieren und Dekodieren) je SNR. Standard: 100\\
\texttt{punctuation.matrix} - Wenn ungleich \texttt{NULL} wird die kodierte Nachricht punktiert. Kann mittels \texttt{ConvGetPunctuationMatrix} erzeugt werden. Standard: \texttt{NULL}\\
\texttt{visualize} - Markiert ob ein Simulationsbericht erzeugt wird. Standard: \texttt{FALSE}\\
\\
\textbf{Rückgabewert:}\\
Dataframe das die Bitfehlerrate für die getesteten Signal-Rausch-Verhältnisse beinhaltet.
\\
\hline
\caption{ConvSimulation}
\label{funktion:ConvSimulation}
\end{longtable}

% Hilfsfunktionen
\section{Hilfsfunktionen}
\label{kapitel:interface_hilfsfunktionen}

\subsection{ConvGetPunctuationMatrix}
% ConvGetPunctuationMatrix.tex
\begin{longtable}{|p{\textwidth}|}
\hline
\rowcolor{lightblue}
ConvGetPunctuationMatrix
\\
\hline
\\
\texttt{ConvGetPunctuationMatrix(punctuation.vector, conv.coder)}\\
\\
Erzeugt aus dem gegebenem Punktierungsvektor und Faltungskodierer eine Punktierungsmatrix.\\
\\
\textbf{Argumente:}\\
\texttt{punctuation.vector} - Vektor der die Punktierungsinformation enthält welche in eine Punktierungsmatrix transformiert wird.\\
\texttt{conv.coder} - Faltungskodierer der für die Matrixdimension verwendet wird.\\
\\
\textbf{Rückgabewert:}\\
Punktierungsmatrix die für \texttt{ConvEncode}, \texttt{ConvDecodeSoft}, \texttt{ConvDecodeHard} und \texttt{ConvSimulation} verwendet werden kann.\\
\\
\hline
\caption{ConvGetPunctuationMatrix}
\label{funktion:ConvGetPunctuationMatrix}
\end{longtable}

\subsection{ConvOpenPDF}
\label{kapitel:interface_ConvOpenPDF}
% ConvOpenPDF.tex
\begin{longtable}{|p{\textwidth}|}
\hline
\rowcolor{lightblue}
ConvOpenPDF
\\
\hline
\\
\texttt{ConvOpenPDF(encode, punctured, simulation)}\\
\\
Öffnet die mit \texttt{ConvEncode}, \texttt{ConvDecodeSoft}, \texttt{ConvDecodeHard} und \texttt{ConvSimulation} erzeugten PDF-Berichte.\\
\\
\textbf{Argumente:}\\
\texttt{encode} - Markiert ob Kodierungsbericht (\texttt{TRUE}) oder Dekodierungsbericht (\texttt{FALSE}) geöffnet wird. Standard: \texttt{TRUE}\\
\texttt{punctured} - Markiert ob Berichte mit Punktierung geöffnet werden. Standard: \texttt{FALSE}\\
\texttt{simulation} - Markiert ob Simulationsbericht geöffnet wird. Standard: \texttt{FALSE}\\
\\
\hline
\caption{ConvOpenPDF Funktion}
\end{longtable}

% Kanalkodierung Funktionen
\section{Kanalkodierung}
\label{kapitel:interface_kanalkodierung}

\subsection{ApplyNoise}
\begin{longtable}{|p{\textwidth}|}
\hline
\rowcolor{lightblue}
ApplyNoise\\
\hline
\\
\texttt{ApplyNoise(msg, SNR.db, binary)}\\
\\
Verrauscht ein Signal basierend auf das AWGN (additive white gaussian noise) Modell. Das ist das Standardmodell für die Simulation eines Übertragungskanals.\\
\\
\textbf{Argumente:}\\
\texttt{msg} - Nachricht die verrauscht wird.\\
\texttt{SNR.db} - Signal/Rausch-Verhältnis des Übertragungskanals. Standard: 3.0\\
\texttt{binary} - Blockkode-Parameter. Nicht zu verwenden! Standard: FALSE\\
\\
\textbf{Rückgabewert:}\\
Verrauschtes Signal.\\
\\
\hline
\caption{ApplyNoise - Funktionserklärung}
\label{func:applynoise}
\end{longtable}

\subsection{ChannelcodingSimulation}
% ChannelcodingSimulation.tex
\begin{longtable}{|p{\textwidth}|}
\hline
\rowcolor{lightblue}ChannelcodingSimulation\\
\hline
\\
\texttt{ChannelcodingSimulation(msg.length, min.db, max.db, db.interval, iterations.per.db, turbo.decode.iterations, visualize)}\\
\\
Simulation of channelcoding techniques (blockcodes, convolutional codes and turbo codes) and comparison of their bit-error-rates.\\
\\
\textbf{Arguments:}\\
\texttt{msg.length} - Message length of the randomly created messages to be encoded. Default: 100\\
\texttt{min.db} - Minimum SNR to be tested. Default: 0.1\\
\texttt{max.db} - Maximum SNR to be tested. Default: 2.0\\
\texttt{db.interval} - Step between two SNRs tested. Default: 0.1\\
\texttt{iterations.per.db} - Number of encode and decode processes per SNR. Default: 100\\
\texttt{turbo.decode.iterations} - Number of decoding iterations inside the turbo decoder. Default: 5\\
\texttt{visualize} - If true a PDF report is generated. Default: FALSE\\
\\
\textbf{Returns:}\\
Distorted message containing noise.\\
\\
\hline
\end{longtable}

\subsection{PlotSimulationData}
% PlotSimulationData.tex
\begin{longtable}{|p{\textwidth}|}
\hline
\rowcolor{lightblue}PlotSimulationData\\
\hline
\\
\texttt{PlotSimulationData(\dots)}\\
\\
Stellt mehrere mitgegebene Dataframes in einem Diagramm dar. Damit kann man verschiedene Kanalkodierungsverfahren miteinander vergleichen.\\
\\
\textbf{Argumente:}\\
\texttt{\dots} - Dataframes die mit den Simulationsfunktionen erzeugt wurden.\\	
\\
\hline
\caption[PlotSimulationData]{PlotSimulationData - Funktionserklärung}
\end{longtable}