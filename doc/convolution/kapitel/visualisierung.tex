% visualisierung.tex
LIMITS!\\\\
%Da das R-Paket für Studenten zu Lernzwecken verwendet werden soll
Um das Verständnis für Faltungskodes beim Benutzer dieses R-Pakets zu stärken, stehen Visualisierungen der Kodierung und Dekodierung zur Verfügung.
\\
Wird der \texttt{visualize} Parameter bei der Ausführung einer Funktion zur Kodierung oder Dekodierung auf TRUE~\texttt{TRUE} gesetzt, wird ein R Markdown Skript ausgeführt. Dieses generiert eine Beamer Präsentation mit Informationen und Visualisierungen zur Kodierung bzw. Dekodierung.
\\\\
\textcolor{lightgray}{\ref{kapitel:visualisierung}.1 Kodierung}\\
% allg. Informationen
Bei der Kodierung befinden sich auf den ersten Folien allgemeine Informationen zum verwendeten Faltungskodierer wie die Kode-Rate, Generatorpolynome, Zustandsübergangstabelle etc. 
% Kodierung
Daraufhin folgt die Kodierungsvisualisierung. Diese zeigt zunächst die zu kodierende Nachricht (Input), das Zustandsübergangsdiagramm sowie eine noch nicht befüllte Kodierungstabelle. Um für einen noch besseren Lerneffekt zu sorgen wird Schritt für Schritt mittels Overlays ein Bit des Inputs, der aktuelle Zustand, Folgezustand sowie der resultierende Output in eine neue Zeile der Kodierungstabelle geschrieben. Der aktuelle Zustand sowie der entsprechende Übergang werden im Diagramm farblich hervorgehoben. Die kodierte Nachricht wächst mit jedem Schritt bis schlussendlich die gesamte Nachricht kodiert wurde.
% Kode zu Signal
Da die Kodierungsfunktion nicht die Bitwerte des Kodeworts zurückliefert sondern die Signalwerte (für eine Übertragung über einen Kanal) wird auf einer weiteren Folie dargestellt, wie die Kodebits in Signalwerte überführt werden.\\
% Punktierung
Wird eine Punktierungsmatrix bei der Kodierung mitgegeben, wird eine zusätzliche Folie am Ende hinzugefügt. Auf dieser wird die Punktierung des Signals, d.h. das Entfernen von Signalwerten (definiert durch die Punktierungsmatrix) dargestellt. Dabei wird neben dem originalen Signal und der Punktierungsmatrix das punktierte Signal dargestellt, wobei zunächst die punktierten Signalwerte, d.h. die entfernten Werte, durch Asterisk-Symbole ($\ast$) ersetzt werden. Diese Darstellung dient als visueller Zwischenschritt für das danach folgende tatsächlich punktierte Signal, bei dem die punktierten Werte fehlen, was auch dem Rückgabewert der Funktion entspricht.
\\\\
\textcolor{lightgray}{\ref{kapitel:visualisierung}.2 Dekodierung}\\
% allg. Informationen
Bei der Dekodierung befinden sich ebenfalls, wie bei der Kodierung, allgemeine Informationen des Faltungskodierers auf den ersten Folien.
% Signal zu Kode
Als Input erhält die Dekodierung das Kodewort als Signalwerte, die möglicherweise durch Anwendung der \texttt{ApplyNoise} Funktion verfälscht worden sind. Die soft decision Dekodierung verwendet zur Dekodierung zwar kontinuierliche Signalwerte, da aber sowohl die hard decision Dekodierung Bitwerde zur Dekodierung verwendet und Trellis-Diagramme mit Bitwerten beschriftet werden, wird auf einer Folie die Überführung der Signalwerte zu Bits dargestellt. Dieser transformierte Input wird auch als Input für die Visualisierung des Viterbi-Algorithmus verwendet.
% Viterbi-Algorithmus
Anschließend folgt die Visualisierung des Viterbi-Algorithmus mithilfe des Trellis-Diagramms. Zunächst werden, zur besseren Übersicht bei großen Diagrammen, jene Pfade entfernt, für die es eine bessere Alternative gibt, d.h. die eine größere Metrik bei hard decision Dekodierung bzw. eine kleinere Metrik bei soft decision Dekodierung als ihre Alternative haben. Danach erfolgt Schritt für Schritt mittels Backtracking die Rekonstruktion der Nachricht. Der gewählte Pfad beim Backtracking wird farblich hervorgehoben. Die übrigen Pfade werden ausgegraut. Am Ende befindet sich unter dem Trellis-Diagramm die farblich hervorgehobene dekodierte Nachricht.
% Punktierung
Wird eine Punktierungsmatrix bei der Dekodierung mitgegeben, wird eine zusätzliche Folie nach den Kodiererinformationen hinzugefügt. Auf dieser wird die Depunktierung des Signals, d.h. das Einfügen des Signalwerts 0 (definiert durch die Punktierungsmatrix), dargestellt. Die eingefügten 0-Werte sind zur leichteren visuellen Erkennung farblich hervorgehoben.
\\\\
\textcolor{lightgray}{\ref{kapitel:visualisierung}.3 Simulation}\\
Weiters können Berichte der Simulation generiert werden, die die resultierenden Daten u.a. in einem Diagramm darstellen.