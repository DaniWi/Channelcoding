% ConvGenerateEncoder
\begin{longtable}{|p{\textwidth}|}
\hline
\rowcolor{lightblue}
ConvGenerateEncoder
\\
\hline
\\
\texttt{ConvGenerateEncoder(N, M, generators)}\\
\\
Erzeugt einen Faltungskodierer für nichtrekursive Faltungskodes.\\
\\
\textbf{Argumente:}\\
\texttt{N} - Anzahl an Ausgangssymbole je Eingangssymbol.\\
\texttt{M} - Länge des Schieberegisters des Kodierers.\\
\texttt{generators} - Vektor der N oktale Generatorpolynome enthält (ein Polynom je Ausgangssymbol, siehe Kapitel~\ref{kapitel:grundlagen_darstellung}).\\
\\
\textbf{Rückgabewert:}\\
Faltungskodierer, abgebildet als Liste mit folgenden Feldern:
\vspace{-4mm}
\begin{itemize}
\renewcommand\labelitemi{--}
\itemsep-.5em % spacing between items
\item \emph{N}: Anzahl an Ausgangssymbole je Eingangssymbol
\item \emph{M}: Länge des Schieberegisters des Kodierers
\item \emph{generators}: Generatorpolynomvektor
\item \emph{next.state}: Zustandsübergangsmatrix
\item \emph{prev.state}: inverse Zustandsübergangsmatrix
\item \emph{output}: Ausgabematrix
\item \emph{rsc}: RSC-Flag (\texttt{FALSE})
\item \emph{termination}: Terminierungsvektor (\texttt{logical(0)})
\end{itemize}
\\
\hline
\caption{ConvGenerateEncoder}
\label{funktion:ConvGenerateEncoder}
\end{longtable}