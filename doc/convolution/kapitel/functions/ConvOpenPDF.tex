% ConvOpenPDF.tex
\begin{longtable}{|p{\textwidth}|}
\hline
\rowcolor{lightblue}
ConvOpenPDF
\\
\hline
\\
\texttt{ConvOpenPDF(encode, punctured, simulation)}\\
\\
Öffnet die mit \texttt{ConvEncode}, \texttt{ConvDecodeSoft}, \texttt{ConvDecodeHard} und \texttt{ConvSimulation} erzeugten PDF-Berichte.\\
\\
\textbf{Argumente:}\\
\texttt{encode} - Markiert ob Kodierungsbericht (\texttt{TRUE}) oder Dekodierungsbericht (\texttt{FALSE}) geöffnet wird. Standard: \texttt{TRUE}\\
\texttt{punctured} - Markiert ob Berichte mit Punktierung geöffnet werden. Standard: \texttt{FALSE}\\
\texttt{simulation} - Markiert ob Simulationsbericht geöffnet wird. Standard: \texttt{FALSE}\\
\\
\hline
\caption{ConvOpenPDF Funktion}
\end{longtable}