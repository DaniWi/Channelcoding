% ConvEncode.tex
\begin{longtable}{|p{\textwidth}|}
\hline
\rowcolor{lightblue}
ConvEncode
\\
\hline
\\
\texttt{ConvEncode(message, conv.encoder, terminate, punctuation.matrix, visualize)}\\
\\
Erzeugt einen Faltungskode aus einer unkodierten Nachricht.\\
\\
\textbf{Argumente:}\\
\texttt{message} - Nachricht die kodiert wird.\\
\texttt{conv.encoder} - Faltungskodierer der für die Kodierung verwendet wird.\\
\texttt{terminate} - Markiert ob der Kode terminiert werden soll. Standard: \texttt{TRUE}\\
\texttt{punctuation.matrix} - Wenn ungleich \texttt{NULL} wird die kodierte Nachricht mit der Punktierungsmatrix punktiert. Standard: \texttt{NULL}\\
\texttt{visualize} - Wenn \texttt{TRUE} wird ein PDF-Bericht der Kodierung erstellt. Standard: \texttt{FALSE}\\
\\
\textbf{Rückgabewert:}\\
Die kodierte Nachricht mit den Signalwerten +1 und -1 welche die Bits 0 und 1 darstellen. Falls punktiert wurde Liste mit dem Originalkode (nicht punktiert) und dem punktiertem Kode.\\
\\
\hline
\caption{ConvEncode}
\end{longtable}