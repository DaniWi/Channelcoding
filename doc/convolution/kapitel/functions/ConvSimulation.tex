% ConvSimulation.tex
\begin{longtable}{|p{\textwidth}|}
\hline
\rowcolor{lightblue}
ConvSimulation
\\
\hline
\\
\texttt{ConvSimulation(conv.coder, msg.length, min.db, max.db, db.interval, iterations.per.db, punctuation.matrix, visualize)}\\
\\
Simulation einer Faltungskodierung und -dekodierung nach einer Übertragung über einen verrauschten Kanal mit verschiedenen Signal-Rausch-Verhältnissen (SNR).\\
\\
\textbf{Argumente:}\\
\texttt{conv.coder} - Faltungskodierer der für die Simulation verwendet wird. Kann mittels \texttt{ConvGenerateEncoder} oder \texttt{ConvGenerateRscEncoder} erzeugt werden.\\
\texttt{msg.length} - Nachrichtenlänge der zufällig generierten Nachrichten. Standard: 100\\
\texttt{min.db} - Untergrenze der getesteten SNR. Standard: 0.1\\
\texttt{max.db} - Obergrenze der getesteten SNR. Standard: 2.0\\
\texttt{db.interval} - Schrittweite zwischen zwei getesteten SNR. Standard: 0.1\\
\texttt{iterations.per.db} - Anzahl der Iterationen (Kodieren und Dekodieren) je SNR. Standard: 100\\
\texttt{punctuation.matrix} - Wenn ungleich \texttt{NULL} wird die kodierte Nachricht punktiert. Kann mittels \texttt{ConvGetPunctuationMatrix} erzeugt werden. Standard: \texttt{NULL}\\
\texttt{visualize} - Markiert ob ein Simulationsbericht erzeugt wird. Standard: \texttt{FALSE}\\
\\
\textbf{Rückgabewert:}\\
Dataframe das die Bitfehlerrate für die getesteten Signal-Rausch-Verhältnisse beinhaltet.
\\
\hline
\caption{ConvSimulation Funktion}
\end{longtable}