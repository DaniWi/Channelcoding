% ChannelcodingSimulation.tex
\begin{longtable}{|p{\textwidth}|}
\hline
\rowcolor{lightblue}
ChannelcodingSimulation
\\
\hline
\\
\texttt{ChannelcodingSimulation(msg.length, min.db, max.db, db.interval, iterations.per.db, turbo.decode.iterations, visualize)}\\
\\
Simulation von Block-, Faltungs- und Turbo-Kodes und Vergleich ihrer Bitfehlerraten bei unterschiedlichen SNR.\\
\\
\textbf{Argumente:}\\
\texttt{msg.length} - Nachrichtenlänge der zufällig generierten Nachrichten. Standard: 100\\
\texttt{min.db} - Untergrenze der getesteten SNR. Standard: 0.1\\
\texttt{max.db} - Obergrenze der getesteten SNR. Standard: 2.0\\
\texttt{db.interval} - Schrittweite zwischen zwei getesteten SNR. Standard: 0.1\\
\texttt{iterations.per.db} - Anzahl der Iterationen (Kodieren und Dekodieren) je SNR. Standard: 100\\
\texttt{turbo.decode.iterations} - Anzahl der Iterationen bei der Turbo-Dekodierung. Standard: 5\\
\texttt{visualize} - Wenn \texttt{TRUE} wird ein PDF-Bericht erstellt. Standard: \texttt{FALSE}\\
\\
\textbf{Rückgabewert:}\\
Dataframe das alle Simulationsergebnisse der 3 Kodierungsverfahren beinhaltet.\\
\\
\hline
\caption{ChannelcodingSimulation Funktion}
\end{longtable}