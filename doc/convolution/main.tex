\documentclass[germanthesis]{thesis-style}
% options:
% [germanthesis] - Thesis is written in German
% [plainunnumbered] - Don't print numbers on plain pages
% [earlydraft] - Settings for quick draft printouts
% [watermark] - Print current time/date at bottom of each page
% [phdthesis] - switch to PhD thesis style
% [twoside] - double sided
% [cutmargins] - text body fills complete page

% Extension to BibLaTex (useful for cites in footnotes)
%\usepackage[backend=bibtex, style=numeric-comp]{biblatex}
\bibliography{references}

\author{Martin~Nocker}
\title{R-Paket für Kanalkodierung mit Faltungskodes}
%\title{R~package for channel~coding with convolutional~codes}
\birthday{1. Mai 1993}
\birthplace{Innsbruck}
%\thesisstart{1. Januar 2009}
\thesistype{Bachelor's thesis}
\thesistypegerman{Bachelorarbeit}
\thesiscite{Bachelor's thesis}
\advisors{Dr.~Pascal~Schöttle}

\newcommand{\plotwidth}{0.6\textwidth}
\begin{document}

\maketitle
\begin{abstract}

\end{abstract}
\begin{otherlanguage}{american}%
\begin{abstract}

\end{abstract}
\end{otherlanguage}{american}
\tableofcontents
\pagenumbering{arabic}

\chapter{Einleitung}
\chapter{Motivation}
\chapter{Grundlagen}
Hier werden Grundlagen aufgelistet\cite{huffman2010fundamentals}.\enquote{Das hier steht unter Anführungszeichen.}
\section{Kanalkodierung}
\section{Faltungskodierung}
\chapter{Implementierung in R}
\section{C Interface}
\chapter{Zusammenfassung}

\cleardoublepage%

\listofabbreviations
\clearpage

\listoffigures
\clearpage

\listoftables
\clearpage

\lstlistoflistings
\clearpage

\printbibliography
\end{document}
