% R Paket Schnittstelle für Faltungskodierung
\definecolor{lightblue}{RGB}{150,180,255}

In diesem Kapitel wird die Schnittstelle für den Benutzer erläutert. Kapitel \ref{chapter:interface_kodierer_erzeugen} listet Funktionen zur Erzeugung von Faltungskodierern auf. Die Kodierungsfunktion wird in Kapitel \ref{chapter:interface_kodieren} beschrieben, gefolgt von den Dekodierungsfunktionen in Kapitel \ref{chapter:interface_dekodieren}. Eine Funktion zur Simulation von Faltungskodes wird in Kapitel \ref{chapter:interface_simulation} erläutert. Kapitel \ref{chapter:interface_hilfsfunktionen} beinhaltet Hilfsfunktionen für Faltungskodes. Schließlich beschreibt Kapitel \ref{chapter:interface_kanalkodierung} weitere nützliche Funktionen der Kanalkodierung.

\subsection{Kodierer erzeugen}
\label{chapter:interface_kodierer_erzeugen}
% ConvGenerateEncoder
\begin{longtable}{|p{\textwidth}|}
\hline
\rowcolor{lightblue}ConvGenerateEncoder\\
\hline
\\
\texttt{ConvGenerateEncoder(N, M, generators)}\\
\\
Generates a convolutional encoder for nonrecursive convolutional codes.\\
\\
\textbf{Arguments:}\\
\texttt{N} - Numer ob output symbols per input symbol.\\
\texttt{M} - Memory length of the encoder.\\
\texttt{generators} - Vector of N octal generator polynoms (one for each output symbol).\\
\\
\textbf{Returns:}\\
A convolutional encoder represented as a list containing:
\vspace{-4mm}
\begin{itemize}
\renewcommand\labelitemi{--}
\itemsep-.5em % spacing between items
\item N
\item M
\item generator polynoms vector
\item nextState matrix
\item previousState matrix
\item output matrix
\item rsc flag
\item termination vector
\end{itemize}
\\
\hline
\end{longtable}

% ConvGenerateRscEncoder.tex
\begin{longtable}{|p{\textwidth}|}
\hline
\rowcolor{lightblue}ConvGenerateRscEncoder\\
\hline
\\
\texttt{ConvGenerateRscEncoder(N, M, generators)}\\
\\
Generates a recursive systematic convolutional (rsc) encoder.\\
\\
\textbf{Arguments:}\\
\texttt{N} - Numer ob output symbols per input symbol.\\
\texttt{M} - Memory length of the encoder.\\
\texttt{generators} - Vector of octal generator polynoms (one for each non-systematic output symbol and one for the recursion).\\
\\
\textbf{Returns:}\\
A convolutional encoder represented as a list containing:
\vspace{-4mm}
\begin{itemize}
\renewcommand\labelitemi{--}
\itemsep-.5em % spacing between items
\item N
\item M
\item generator polynoms vector
\item nextState matrix
\item previousState matrix
\item output matrix
\item rsc flag
\item termination vector
\end{itemize}
\\
\hline
\end{longtable}

\subsection{Kodieren}
\label{chapter:interface_kodieren}
% ConvEncode.tex
\begin{longtable}{|p{\textwidth}|}
\hline
\rowcolor{lightblue}
ConvEncode
\\
\hline
\\
\texttt{ConvEncode(message, conv.encoder, terminate, punctuation.matrix, visualize)}\\
\\
Erzeugt einen Faltungskode aus einer unkodierten Nachricht.\\
\\
\textbf{Argumente:}\\
\texttt{message} - Nachricht die kodiert wird.\\
\texttt{conv.encoder} - Faltungskodierer der für die Kodierung verwendet wird.\\
\texttt{terminate} - Markiert ob der Kode terminiert werden soll. Standard: \texttt{TRUE}\\
\texttt{punctuation.matrix} - Wenn ungleich \texttt{NULL} wird die kodierte Nachricht mit der Punktierungsmatrix punktiert. Standard: \texttt{NULL}\\
\texttt{visualize} - Wenn \texttt{TRUE} wird ein PDF-Bericht der Kodierung erstellt. Standard: \texttt{FALSE}\\
\\
\textbf{Rückgabewert:}\\
Die kodierte Nachricht mit den Signalwerten +1 und -1 welche die Bits 0 und 1 darstellen. Falls punktiert wurde Liste mit dem Originalkode (nicht punktiert) und dem punktiertem Kode.\\
\\
\hline
\caption{ConvEncode}
\label{funktion:ConvEncode}
\end{longtable}

\subsection{Dekodieren}
\label{chapter:interface_dekodieren}
% ConvDecodeSoft.tex
\begin{longtable}{|p{\textwidth}|}
\hline
\rowcolor{lightblue}ConvDecodeSoft\\
\hline
\\
\texttt{ConvDecodeSoft(code, conv.encoder, terminate, punctuation.matrix, visualize)}\\
\\
Decodes a convolutional codeword using soft decision decoding.\\
\\
\textbf{Arguments:}\\
\texttt{code} - The code to be decoded (soft input).\\
\texttt{conv.encoder} - Convolutional encoder used for encoding.\\
\texttt{terminate} - Flag if the code is terminated. Default: TRUE\\
\texttt{punctuation.matrix} - If not NULL the code is depunctured prior to the decode algorithm. Default: NULL\\
\texttt{visualize} - If TRUE a beamer PDF file is generated showing the decode process. Default: FALSE\\
\\
\textbf{Returns:}\\
The decoded message as a list containing the soft output-values and hard output-values.\\
\\
\hline
\end{longtable}

% ConvDecodeHard.tex
\begin{longtable}{|p{\textwidth}|}
\hline
\rowcolor{lightblue}ConvDecodeHard\\
\hline
\\
\texttt{ConvDecodeHard(code, conv.encoder, terminate, punctuation.matrix, visualize)}\\
\\
Decodes a convolutional codeword using hard decision decoding.\\
\\
\textbf{Arguments:}\\
\texttt{code} - The code to be decoded.\\
\texttt{conv.encoder} - Convolutional encoder used for encoding.\\
\texttt{terminate} - Flag if the code is terminated. Default: TRUE\\
\texttt{punctuation.matrix} - If not NULL the code is depunctured prior to the decode algorithm. Default: NULL\\
\texttt{visualize} - If TRUE a beamer PDF file is generated showing the decode process. Default: FALSE\\
\\
\textbf{Returns:}\\
The hard-decoded message vector.\\
\\
\hline
\end{longtable}

\subsection{Simulation}
\label{chapter:interface_simulation}
% ConvSimulation.tex
\begin{longtable}{|p{\textwidth}|}
\hline
\rowcolor{lightblue}
ConvSimulation
\\
\hline
\\
\texttt{ConvSimulation(conv.coder, msg.length, min.db, max.db, db.interval, iterations.per.db, punctuation.matrix, visualize)}\\
\\
Simulation einer Faltungskodierung und -dekodierung nach einer Übertragung über einen verrauschten Kanal mit verschiedenen Signal-Rausch-Verhältnissen (SNR).\\
\\
\textbf{Argumente:}\\
\texttt{conv.coder} - Faltungskodierer der für die Simulation verwendet wird. Kann mittels \texttt{ConvGenerateEncoder} oder \texttt{ConvGenerateRscEncoder} erzeugt werden.\\
\texttt{msg.length} - Nachrichtenlänge der zufällig generierten Nachrichten. Standard: 100\\
\texttt{min.db} - Untergrenze der getesteten SNR. Standard: 0.1\\
\texttt{max.db} - Obergrenze der getesteten SNR. Standard: 2.0\\
\texttt{db.interval} - Schrittweite zwischen zwei getesteten SNR. Standard: 0.1\\
\texttt{iterations.per.db} - Anzahl der Iterationen (Kodieren und Dekodieren) je SNR. Standard: 100\\
\texttt{punctuation.matrix} - Wenn ungleich \texttt{NULL} wird die kodierte Nachricht punktiert. Kann mittels \texttt{ConvGetPunctuationMatrix} erzeugt werden. Standard: \texttt{NULL}\\
\texttt{visualize} - Markiert ob ein Simulationsbericht erzeugt wird. Standard: \texttt{FALSE}\\
\\
\textbf{Rückgabewert:}\\
Dataframe das die Bitfehlerrate für die getesteten Signal-Rausch-Verhältnisse beinhaltet.
\\
\hline
\caption{ConvSimulation}
\label{funktion:ConvSimulation}
\end{longtable}

\subsection{Hilfsfunktionen}
\label{chapter:interface_hilfsfunktionen}
% ConvGetPunctuationMatrix.tex
\begin{longtable}{|p{\textwidth}|}
\hline
\rowcolor{lightblue}ConvGetPunctuationMatrix\\
\hline
\\
\texttt{ConvGetPunctuationMatrix(punctuation.vector, conv.coder)}\\
\\
Creates a punctuation matrix from the passed punctuation vector and the passed coder.\\
\\
\textbf{Arguments:}\\
\texttt{punctuation.vector} - Vector containing the punctuation information which will be transformed to a punctuation matrix.\\
\texttt{conv.coder} - Convolutional coder which is used for the matrix dimensions.\\
\\
\textbf{Returns:}\\
Punctuation matrix suitable for ConvEncode, ConvDecodeSoft, ConvDecodeHard and ConvSimulation.\\
\\
\hline
\end{longtable}

% ConvOpenPDF.tex
\begin{longtable}{|p{\textwidth}|}
\hline
\rowcolor{lightblue}
ConvOpenPDF
\\
\hline
\\
\texttt{ConvOpenPDF(encode, punctured, simulation)}\\
\\
Öffnet die mit \texttt{ConvEncode}, \texttt{ConvDecodeSoft}, \texttt{ConvDecodeHard} und \texttt{ConvSimulation} erzeugten PDF-Berichte.\\
\\
\textbf{Argumente:}\\
\texttt{encode} - Markiert ob Kodierungsbericht (\texttt{TRUE}) oder Dekodierungsbericht (\texttt{FALSE}) geöffnet wird. Standard: \texttt{TRUE}\\
\texttt{punctured} - Markiert ob Berichte mit Punktierung geöffnet werden. Standard: \texttt{FALSE}\\
\texttt{simulation} - Markiert ob Simulationsbericht geöffnet wird. Standard: \texttt{FALSE}\\
\\
\hline
\caption{ConvOpenPDF Funktion}
\end{longtable}

\subsection{Kanalkodierung}
\label{chapter:interface_kanalkodierung}
\begin{longtable}{|p{\textwidth}|}
\hline
\rowcolor{lightblue}
ApplyNoise\\
\hline
\\
\texttt{ApplyNoise(msg, SNR.db, binary)}\\
\\
Verrauscht ein Signals basierend auf das AWGN (additive white gaussian noise) Modell. Das ist das Standardmodell für die Simulation eines Übertragungskanals.\\
\\
\textbf{Argumente:}\\
\texttt{msg} - Nachricht die verrauscht wird.\\
\texttt{SNR.db} - Signal/Rausch-Verhältnis des Übertragungskanals. Standard: 3.0\\
\texttt{binary} - Blockkode-Parameter. Nicht zu verwenden! Standard: FALSE\\
\\
\textbf{Rückgabewert:}\\
Verrauschtes Signal.\\
\\
\hline
\caption{ApplyNoise - Funktionserklärung}
\end{longtable}

% ChannelcodingSimulation.tex
\begin{longtable}{|p{\textwidth}|}
\hline
\rowcolor{lightblue}
ChannelcodingSimulation
\\
\hline
\\
\texttt{ChannelcodingSimulation(msg.length, min.db, max.db, db.interval, iterations.per.db, turbo.decode.iterations, visualize)}\\
\\
Simulation von Block-, Faltungs- und Turbo-Kodes und Vergleich ihrer Bitfehlerraten bei unterschiedlichen SNR.\\
\\
\textbf{Argumente:}\\
\texttt{msg.length} - Nachrichtenlänge der zufällig generierten Nachrichten. Standard: 100\\
\texttt{min.db} - Untergrenze der getesteten SNR. Standard: 0.1\\
\texttt{max.db} - Obergrenze der getesteten SNR. Standard: 2.0\\
\texttt{db.interval} - Schrittweite zwischen zwei getesteten SNR. Standard: 0.1\\
\texttt{iterations.per.db} - Anzahl der Iterationen (Kodieren und Dekodieren) je SNR. Standard: 100\\
\texttt{turbo.decode.iterations} - Anzahl der Iterationen bei der Turbo-Dekodierung. Standard: 5\\
\texttt{visualize} - Wenn \texttt{TRUE} wird ein PDF-Bericht erstellt. Standard: \texttt{FALSE}\\
\\
\textbf{Rückgabewert:}\\
Dataframe das alle Simulationsergebnisse der 3 Kodierungsverfahren beinhaltet.\\
\\
\hline
\caption{ChannelcodingSimulation Funktion}
\end{longtable}

% PlotSimulationData.tex
\begin{longtable}{|p{\textwidth}|}
\hline
\rowcolor{lightblue}PlotSimulationData\\
\hline
\\
\texttt{PlotSimulationData(\dots)}\\
\\
Stellt mehrere mitgegebene Dataframes in einem Diagramm dar. Damit kann man verschiedene Kanalkodierungsverfahren miteinander vergleichen.\\
\\
\textbf{Argumente:}\\
\texttt{\dots} - Dataframes die mit den Simulationsfunktionen erzeugt wurden.\\	
\\
\hline
\caption{PlotSimulationData - Funktionserklärung}
\end{longtable}