% kommunikationskanal.tex
\begin{tikzpicture}[>=stealth]
\def\myinnersep{3mm}
\tikzstyle{block} = [draw, rectangle, node distance=10mm, inner sep=\myinnersep, align=center, font=\small, minimum height=15mm, minimum width=width("Dekodierer")+2*\myinnersep]
\node[block] (quelle) {Informations- \\ quelle};
\node[block] (kodierer) [right=of quelle] {Kodierer};
\node[block] (kanal) [right=of kodierer] {Kanal};
\node[block] (dekodierer) [right=of kanal] {Dekodierer};
\node[block] (senke) [right=of dekodierer] {Empfänger};
\draw[->] (quelle) -- node [above] {$\mathbf{u}$} (kodierer);
\draw[->] (kodierer) -- node [above] {$\mathbf{v}$} (kanal);
\draw[->] (kanal) -- node [above] {$\mathbf{y}$} (dekodierer);
\draw[->] (dekodierer) -- node [above] {$\mathbf{u'}$} (senke);
\draw[<-] (kanal.south) -- ++(0,-10mm) node [below] {$\mathbf{e}$};
\end{tikzpicture}