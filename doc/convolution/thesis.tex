\documentclass[germanthesis]{thesis-style}
% options:
% [germanthesis] - Thesis is written in German
% [plainunnumbered] - Don't print numbers on plain pages
% [earlydraft] - Settings for quick draft printouts
% [watermark] - Print current time/date at bottom of each page
% [phdthesis] - switch to PhD thesis style
% [twoside] - double sided
% [cutmargins] - text body fills complete page

% Extension to BibLaTex (useful for cites in footnotes)
%\usepackage[backend=bibtex, style=numeric-comp]{biblatex}
\bibliography{references}

\author{Martin~Nocker}
\title{R-Paket für Kanalkodierung mit Faltungskodes}
%\title{R~package for channel~coding with convolutional~codes}
\birthday{1. Mai 1993}
\birthplace{Innsbruck}
%\thesisstart{1. Januar 2009}
\thesistype{Bachelor's thesis}
\thesistypegerman{Bachelorarbeit}
\thesiscite{Bachelor's thesis}
\advisors{Dr.~Pascal~Schöttle}

% additional packages
\usepackage{color, colortbl} % colors (interface table)
%\usepackage{tabularx} % set table width
\usepackage{longtable}

\newcommand{\plotwidth}{0.6\textwidth}
\begin{document}

% Titelseiten und Eidesstattliche Erklärung
\maketitle

% Zusammenfassung
\begin{abstract}

\end{abstract}
% Abstract
\begin{otherlanguage}{american}%
\begin{abstract}

\end{abstract}
\end{otherlanguage}

% Inhaltsverzeichnis
\tableofcontents
\pagenumbering{arabic}

\chapter{Einleitung}
Hier werden Grundlagen aufgelistet \cite{huffman2010fundamentals}.\\
\enquote{Das hier steht unter Anführungszeichen.}
\section{Motivation}
\section{Problem}
\section{Ziel}

\chapter{Grundlagen und ähnliche Arbeiten}

\chapter{Verwendete Technologien}
\section{R, RStudio, Pakete}
\section{C++, Rcpp}
\section{R Markdown}
\section{\LaTeX, Ti\textit{k}Z}

\chapter{R Paket Schnittstelle}
\label{chapter:interface}
% R Paket Schnittstelle für Faltungskodierung
\definecolor{lightblue}{RGB}{150,180,255}

\section{Kanalkodierung}
\label{chapter:interface_kanalkodierung}
\begin{longtable}{|p{\textwidth}|}
\hline
\rowcolor{lightblue}
ApplyNoise\\
\hline
\\
\texttt{ApplyNoise(msg, SNR.db, binary)}\\
\\
Verrauscht ein Signals basierend auf das AWGN (additive white gaussian noise) Modell. Das ist das Standardmodell für die Simulation eines Übertragungskanals.\\
\\
\textbf{Argumente:}\\
\texttt{msg} - Nachricht die verrauscht wird.\\
\texttt{SNR.db} - Signal/Rausch-Verhältnis des Übertragungskanals. Standard: 3.0\\
\texttt{binary} - Blockkode-Parameter. Nicht zu verwenden! Standard: FALSE\\
\\
\textbf{Rückgabewert:}\\
Verrauschtes Signal.\\
\\
\hline
\caption{ApplyNoise - Funktionserklärung}
\end{longtable}

\chapter{Visualisierung}

\chapter{Implementierung}

\chapter{Beispiele}

\chapter{Fazit, Ausblick, Erweiterungen}

\cleardoublepage%

\listofabbreviations
\clearpage

\listoffigures
\clearpage

\listoftables
\clearpage

\lstlistoflistings
\clearpage

\printbibliography
\end{document}
