\documentclass[germanthesis]{thesis-style}
% options:
% [germanthesis] - Thesis is written in German
% [plainunnumbered] - Don't print numbers on plain pages
% [earlydraft] - Settings for quick draft printouts
% [watermark] - Print current time/date at bottom of each page
% [phdthesis] - switch to PhD thesis style
% [twoside] - double sided
% [cutmargins] - text body fills complete page

% Extension to BibLaTex (useful for cites in footnotes)
%\usepackage[backend=bibtex, style=numeric-comp]{biblatex}
\bibliography{references}

\author{Martin~Nocker}
\title{R-Paket für Kanalkodierung mit Faltungskodes}
%\title{R~package for channel~coding with convolutional~codes}
\birthday{1. Mai 1993}
\birthplace{Innsbruck}
%\thesisstart{1. Januar 2009}
\thesistype{Bachelor's thesis}
\thesistypegerman{Bachelorarbeit}
\thesiscite{Bachelor's thesis}
\advisors{Univ.-Prof.~Dr.~Rainer~Böhme, Dr.~Pascal~Schöttle}

% additional packages
\usepackage{color, colortbl} % colors (interface table)
\usepackage{longtable}
\usepackage{amsmath}
\usepackage{graphicx}
\usepackage{caption}
\usepackage{tikz}
\usetikzlibrary{backgrounds, calc}

% Listings standardmaessig in R
\lstset{language=R} 

% Definiert Makro für Grafik Skalierung
\makeatletter
\def\ScaleIfNeeded{%
\ifdim\Gin@nat@width>\linewidth
\linewidth
\else
\Gin@nat@width
\fi
}
\makeatother

\newcommand{\plotwidth}{0.6\textwidth}
\begin{document}

% Titelseiten und Eidesstattliche Erklärung
\maketitle

% Zusammenfassung
\begin{abstract}

\end{abstract}

\tableofcontents
\clearpage
\pagenumbering{arabic}

\chapter{Einleitung}
\label{kapitel:einleitung}
% einleitung.tex
\section{Motivation}
Lorem Ipsum
\\
\\
\section{Ziel}

\chapter{Grundlagen und ähnliche Arbeiten}
\label{kapitel:grundlagen}

\chapter{Verwendete Technologien}
\label{kapitel:technologien}
% technologien.tex
Kapitel~\ref{kapitel:R} behandelt die Programmiersprache R und die verwendete Entwicklungsumgebung RStudio. In Kapitel~\ref{kapitel:rcpp} werden die Möglichkeiten der Einbindung von C/C++-Code, vor allem mithilfe des Pakets \texttt{Rcpp}, beschrieben. Schließlich wird in Kapitel~\ref{kapitel:rmarkdown} auf die Erstellung dynamischer Dokumente und Visualisierungen mittels \texttt{RMarkdown}, \LaTeX\ und Ti\textit{k}Z eingegangen.
\section{R, RStudio, Pakete}
\label{kapitel:R}
\begin{figure}[t]
\centering
\includegraphics[width=0.9\textwidth]{abbildungen/rstudio}
\caption{RStudio Standardansicht}
\label{abb:rstudio}
\end{figure}
% R
R ist eine, im Jahre 1992 entwickelte, schwach und dynamisch typisierte Programmiersprache, die vor allem in der Statistik für die Analyse von großen Datenmengen Anwendung findet. Ein weiteres Motiv für die Verwendung von R sind die vielseitigen Möglichkeiten, bei gleichzeitig einfacher Handhabung, große Datenmengen graphisch darzustellen. R-Code wird nicht kompiliert, sondern nur interpretiert und ist daher plattformübergreifend verwendbar. Datentypen müssen zur Übersetzungszeit nicht bekannt sein. Die Typüberprüfung findet zur Laufzeit statt. Diese Eigenschaft erschwert das Finden von Fehlern im Code erheblich.

Der Funktionsumfang der Sprache kann durch sogenannte Pakete erweitert werden. Bei der Installation von R sind die wichtigsten Pakete inkludiert. Über Repositories wie CRAN\footnote{The Comprehensive R Archive Network: \url{https://cran.r-project.org/}} oder GitHub sind über 8000 zusätzliche Pakete (Stand: Mai 2016) für die verschiedensten Anwendungsbereiche verfügbar. Diese Vielfalt an Paketen ist ein Grund für den Erfolg von R~\cite[S.~18]{wickham2015r}.
\\
Pakete werden laufend aktualisiert und verbessert. Selbst entwickelte Pakete können via CRAN für andere Entwickler veröffentlicht werden, müssen jedoch strenge Auflagen zur Aufrechterhaltung der Konsistenz bei Inhalt, Form und Dokumentation der Pakete einhalten~\cite{wickham2015r}.  

Ein wichtiges Paket, welches im Rahmen dieser Arbeit verwendet wurde, ist \texttt{roxygen2}. Mithilfe dieses Pakets wird, ähnlich zur JavaDoc für Java, durch spezielle Kommentare und Annotations überhalb der Paketfunktionen automatisch die Paketdokumentation erstellt. Die \texttt{roxygen}-Kommentare der Paketfunktionen, die für wartbaren Code ohnehin unabdingbar sind, sind für den Entwickler erheblich angenehmer, als die Paketdokumentation von Hand zu schreiben. Diese werden durch das Kommentarsymbol \mbox{\texttt{\#'}} am Zeilenbeginn eingeleitet. Zu den wichtigsten Annotations gehören jene für die Beschreibung der Parameter (\texttt{@param}) und Rückgabewerte (\texttt{@return}) sowie Beispiele zur Ausführung der Funktion (\texttt{@examples}). Weiters wird über die \texttt{@export} Annotation geregelt, welche Funktionen nach Auslieferung des Pakets von außen aufrufbar sind.~\cite{roxygen}

Ein weiteres hilfreiches Paket ist das \texttt{devtools}-Paket. Dieses Paket stellt Funktionen für die Erstellung (Build) von Paketen zur Verfügung und beschleunigt so den Build-Workflow für den Entwickler.~\cite{devtools}

% RStudio
RStudio ist eine freie open-source Entwicklungsumgebung für R. RStudio verfügt über alle notwendigen Funktionalitäten für die Softwareentwicklung mit R und bietet darüber hinaus Funktionen für eine vereinfachte Entwicklung von R-Paketen an. Abbildung~\ref{abb:rstudio} zeigt die Version 0.99.893. 
\section{C++, Rcpp}
\label{kapitel:rcpp}
Vorteile von R, wie die einfache Analyse von Datenmengen, kommen mit einem Nachteil: R ist keine schnelle Sprache. Typische Flaschenhälse sind Schleifen und rekursive Funktionen. Die Performance kann in solchen Fällen durch Auslagern von Funktionen und Algorithmen in C oder C++ erheblich verbessert werden, da der Code in diesen Sprachen kompiliert und somit optimiert werden kann, anstatt nur interpretiert zu werden.\\
R bietet drei Möglichkeiten C/C++-Code aufzurufen:
\begin{itemize}
	\itemsep-.2em % spacing between items
	\item \texttt{.C}-Schnittstelle
	\item \texttt{.Call}-Schnittstelle
	\item \texttt{Rcpp}-Paket
\end{itemize}
Die \texttt{.C}-Schnittstelle ist die einfachste Variante C-Code auszuführen, jedoch auch jene mit den größten Einschränkungen. Im C-Code sind keinerlei R-Datentypen oder R-Funktionen bekannt. Alle Argumente sowie der Rückgabewert müssen als Zeiger in der Parameterliste übergeben werden und deren Speicher muss vor dem Aufruf reserviert werden.

Bei der \texttt{.Call}-Schnittstelle handelt es sich um eine Erweiterung der \texttt{.C}-Schnittstelle. Die Implementierung ist komplexer, dafür sind R-Datentypen verfügbar und es gibt, mithilfe des \texttt{return} Statements, die Möglichkeit eines Rückgabewerts.

Sowohl bei der \texttt{.C}-Schnittstelle als auch bei der \texttt{.Call}-Schnittstelle muss der C-Code vor dem Aufruf per Hand kompiliert und in der R Session geladen werden. Das \texttt{Rcpp}-Paket ermöglicht die Verwendung von C++-Code ohne diesen Aufwand. Im C++-Code stehen R-Datentypen wie Vektoren, Matrizen oder Listen ohne komplizierte Syntax zur Verfügung. Die Funktionsaufrufe sehen, im Gegensatz zu den C-Schnittstellen, wie normale R-Funktionsaufrufe aus und machen dadurch den Code erheblich lesbarer. Weiters stehen Vektorfunktionen zur Verfügung, d.h. eine auf einen Vektor angewandte Funktion wird auf jedes Vektorelement ausgeführt und erspart somit beispielsweise eine Schleife. Bei der Entwicklung eines eigenen Pakets ist es bei der Verwendung des \texttt{Rcpp}-Pakets zusammen mit RStudio sehr einfach C++-Code zu integrieren. Durch all diese Vorteile ist das \texttt{Rcpp}-Paket die zu wählende Schnittstelle. Die genaue Verwendung des \texttt{Rcpp}-Pakets ist in \cite[S.~395~ff.]{wickham2015advanced} beschrieben.
\begin{figure}[t]
\centering
\includegraphics[width=0.9\textwidth]{abbildungen/rmarkdown}
\caption[RMarkdown Überblick]{RMarkdown Überblick, Quelle:~\cite{rmarkdown}}
\label{abb:rmarkdown}
\end{figure}
\section{RMarkdown, \LaTeX, Ti\textit{k}Z}
\label{kapitel:rmarkdown}
% R Markdown
Zur Erstellung von dynamischen Dokumenten wird das Paket \texttt{RMarkdown} verwendet. Durch die Kombination der Syntax von Markdown, R, und \LaTeX\ ergibt sich ein flexibles und einfaches Werkzeug. Die unterstützen Ausgabeformate beinhalten u.a. HTML, PDF, MS Word und Beamer (Präsentationen).

Abbildung~\ref{abb:rmarkdown} zeigt den Workflow für die Generierung eines dynamischen Dokuments mittels \texttt{RMarkdown}. Der Markdown-, R- und \LaTeX -Code wird zusammen mit dem gewünschten Ausgabeformat, wobei mehrere Angaben möglich sind, in die \texttt{RMarkdown}-Datei (Dateiendung .rmd) geschrieben. Die RMD-Datei wird dem \texttt{knitr}-Paket übergeben, welches den R-Code ausführt und eine neue Markdown-Datei (Dateiendung .md) erstellt, die den R-Code und dessen Ergebnisse beinhaltet. Die erzeugte Markdown-Datei wird von \texttt{pandoc} weiterverarbeitet, das für die Erstellung des endgültigen Dokuments im gewünschten Format zuständig ist. Bei der Verwendung von RStudio ist \texttt{pandoc} automatisch verfügbar. Den eben beschriebenen Ablauf kapselt das \texttt{RMarkdown}-Paket in einem einzigen \texttt{render}-Funktionsaufruf.

Für die Erzeugung dynamischer Grafiken wird das \LaTeX -Sprachpaket Ti\textit{k}Z verwendet. Mithilfe des Dokumenttyps Beamer in \LaTeX\ lassen sich Präsentationen erstellen. Die Grafiken und Inhalte können dadurch dynamisch ein- oder ausgeblendet, sowie farblich hervorgehoben werden. Dies ist insofern wertvoll, da Informationen, die Schritt für Schritt vervollständigt werden, es dem Benutzer leichter machen den Ablauf nachzuvollziehen. Damit Benutzer des R-Pakets dieser Arbeit die Prinzipien von Faltungskodes besser verstehen können, werden die Visualisierungen der Kodierung und Dekodierung sukzessive eingeblendet.

\chapter{Implementierung}
\label{kapitel:implementierung}
% implementierung.tex
Dieses Kapitel gibt einen Einblick in die Konzepte der Implementierung. Als Einstiegspunkt stand eine Referenzimplementierung\footnote{\url{http://vashe.org/turbo/turbo_example.c} (01.06.2016)} zur Verfügung, die den Dekodier-Algorithmus für Turbo-Kodes beinhaltet, jedoch für ein konkretes Beispiel. Dieser musste angepasst werden um für allgemeine Faltungskodes verwendbar zu sein.
\\
\\
Kapitel \ref{kapitel:implementierung_faltungskodierer} beinhaltet den Entwurf der Faltungskodierer-Datenstruktur. [TODO: Fertigstellung]
%Die Implementierung der Kodierung wird in Kapitel \ref{kapitel:implementierung_kodierung} beschrieben, die der Dekodierung in Kapitel \ref{kapitel:implementierung_dekodierung}.
%\\
%\\
%Aus Performancegründen Kodierung, Dekodierung in C++\\
%Weiters: Kodierer erzeugen, Depunktierung, Katastrophale Kodierer Prüfung (Polynom GGT mod 2)
%\\
%Parameterprüfung, Punktierung, ApplyNoise, Aufruf Visualisierung in R\\
%Referenzimplementierung
\section{Faltungskodierer}
\label{kapitel:implementierung_faltungskodierer}
Ein Faltungskodierer ist gegeben durch 
\begin{itemize}
\item $N$: Anzahl an Ausgangsbits je Eingangsbit,
\item $M$: Länge des Schieberegisters,
\item $G$: Vektor von Generatorpolynomen.
\end{itemize}
Die Angabe von $M$ ist hier redundant, jedoch Teil der Benutzereingabe zur Generierung eines Faltungskodierers, welche durch \cite{morelos2006art} inspiriert wurde.
\\
\\
Zur leichteren Implementierung der Kodierung und Dekodierung wird die Kodierer-Datenstruktur um folgende Elemente erweitert:
\begin{itemize}
\item eine \emph{Zustandsübergangsmatrix}, die angibt, in welchen Zustand der Kodierer bei einem Eingangsbit wechselt,
\item eine \emph{inverse Zustandsübergangsmatrix}, die angibt, aus welchem Zustand der Kodierer bei einem Eingangsbit kommt,
\item eine \emph{Outputmatrix}, die angibt, welche Kodebits der Kodierer bei einem Eingangsbit in einem bestimmten Zustand ausgibt,
\item ein Flag zur Markierung rekursiver systematischer Kodierer (RSC),
\item ein \emph{Terminierungsvektor} die für rekursiver systematische Kodierer angibt, ob ein Eingangsbit 0 oder 1 in einem bestimmten Zustand für die Terminierung zu verwenden ist.
\end{itemize}
Die Implementierung der Matrizen wurde aus der Referenzimplementierung übernommen, musste jedoch erweitert werden, um für allgemeine Faltungskodes verwendbar zu sein. Für alle gilt, die Anzahl an Zeilen entspricht der Anzahl an Zuständen. Der Zeilenindex entspricht dem Zustand. Die Zustandsübergangsmatrix sowie die Outputmatrix besitzen jeweils zwei Spalten. Je eine Spalte steht für ein Eingangsbit (0 oder 1), wobei der Spaltenindex dem Eingangsbit entspricht. Die inverse Zustandsübergangsmatrix benötigt eine dritte Spalte. Für viele Kodierer (v.a. nicht-rekursive) tritt der Fall ein, dass nur durch \emph{ein bestimmtes} Eingangsbit in einen bestimmten Zustand gewechselt werden kann. Sei ein Zustand bspw. nur durch das Eingangsbit 0 erreichbar, so bedeutet das, dass es für diesen Zustand mit dem Bit 0 \emph{zwei} Vorgängerzustände gibt, für ein Eingangsbit 1 jedoch keinen Vorgänger. Diese zweite Möglichkeit wird in der dritten Spalte gespeichert.
\\
\\
Der Terminierungsvektor ist für nicht-rekursive Kodierer nicht notwendig, da ein Kode eines solchen Kodierers immer mit $M$ 0-Bits terminiert wird. Bei einem rekursiven Kodierer ist es nicht trivial zu sagen mit welchem Eingangsbit in einem bestimmten Zustand terminiert wird, um den Kodierer in den Nullzustand zu bringen. Dies hängt von der Definition des Rekursionpolynoms ab. Der Terminierungsvektor wird bei der Erzeugung rekursiver Kodierer berechnet.
\\
\\
Bei der Erzeugung von Faltungskodierern ist zu prüfen ob es sich um einen katastrophalen Kodierer handelt. RSC-Kodierer sind, wie in Kapitel \ref{kapitel:grundlagen_systematische_kodierer} beschrieben, nicht zu prüfen. Zur Prüfung wird nach Theorem \ref{thm:massey} der größte gemeinsame Teiler der Generatorpolynome berechnet. Die Berechnung des größten gemeinsamen Teilers wurde mithilfe des des euklidschen Algorithmus implementiert. Sowohl der euklidsche Algorithmus als auch die dafür notwendige binäre Polynomdivision wird an eine C++ Funktion delegiert.

\section{Kodierung}
\label{kapitel:implementierung_kodierung}
Bei Faltungskodes stellt die Kodierung den bei Weitem einfacheren Teil dar. Es muss lediglich jedes Bit der zu kodierenden Nachricht zusammen mit dem aktuellen Zustand, der nach jedem Bit mithilfe der Zustandsübergangsmatrix aktualisiert wird, auf die Outputmatrix angewendet werden. Die Terminierung funktioniert analog, einzig das zu kodierende Bit muss ermittelt werden. Für RSC-Kodierer muss im Terminierungsvektor nachgeschaut werden, andernfalls ist das Bit immer 0. Abgeschlossen wird die Kodierung mit dem Abbilden der Kodebits 0 bzw. 1 auf die Signalwerte +1 bzw. -1. Algorithmus \ref{algorithmus:kodierung} zeigt den Kodierungsalgorithmus.

\begin{algorithm}[H]
\renewcommand{\algorithmicforall}{\textbf{for each}}
\caption{Faltungskodierung}
\label{algorithmus:kodierung}
\begin{algorithmic}[1]
\STATE state $=0$, code $=$ result $=$ " "
\FORALL {bit \textbf{in} message}
   \STATE output $=$ output.matrix[state][bit]
	\STATE code $=$ $concat($code, output$)$
	\STATE state $=$ state.transition.matrix$[$state$][$bit$]$
\ENDFOR
\IF{terminate code}
   \FOR {$i=0$ \TO $M-1$}
      \STATE termination.bit $=$ rsc-coder $?$ termination.vector$[$state$]$ : $0$
      \STATE output $=$ output.matrix$[$state$][$termination.bit$]$
	   \STATE code $=$ $concat($code, output$)$
	   \STATE state $=$ state.transition.matrix$[$state$][$termination.bit$]$
   \ENDFOR
\ENDIF
\FORALL {bit \textbf{in} code}
   \STATE signal $=1-2$bit
   \STATE result $=$ $concat($result,signal$)$
\ENDFOR
\RETURN result
\end{algorithmic}
\end{algorithm}

\section{Dekodierung}
\label{kapitel:implementierung_dekodierung}
Die Dekodierung stellt den wesentlich komplexeren Teil der Faltungskodes dar. 

\begin{algorithm}[H]
\renewcommand{\algorithmicforall}{\textbf{for each}}
\caption{Faltungsdekodierung}
\label{algorithmus:dekodierung}
\begin{algorithmic}[1]
\STATE $NUM_STATES=2^{M}$
\FOR {$t=1$ \TO $length($message$)$}
   \FOR {$s=0$ \TO $NUM_STATES-1$}
   	\STATE $m_{1}=$ metric[t-1][prev.state1] + $\delta_{1}$
   	\STATE $m_{2}=$ metric[t-1][prev.state2] + $\delta_{2}$
      \STATE metric[t][s] = $min(m_{1},m_{2})$
      \STATE survivor.bit $=$ $xyz(0,1,min(min(m_{1},m_{2}))$
	\ENDFOR
\ENDFOR
\RETURN result
\end{algorithmic}
\end{algorithm}

\section{Rauschen}
\label{kapitel:implementierung_noise}
Um auch zeigen zu können, dass die Dekodierung auch tatsächlich für verrauschte Signale funktioniert, benötigt es eine Funktion, die die Übertragung einer Nachricht über einen verrauschten Kanal simuliert, d.h. das Signal mit Rauschen überlagert. Zum Signal soll ein additives weißes gaußsches Rauschen (AWGR oder AWGN\footnote{additive white Gaussian noise}) addiert werden um dieses zu verfälschen. [apply noise quelle] stellt eine alternative Implementierung zur eingebauten AWGN-Funktion in Matlab vor. Die Implementierung wurde übernommen bzw. nach R übersetzt. Durch die Möglichkeit das Signal-Rausch-Verhältnis über einen Parameter zu steuern, können verschiedene Übertragungskanäle simuliert und Nachrichten somit verschieden stark verrauscht werden. Der Benutzer kann dadurch herausfinden, ab wann eine Nachricht zu viel Rauschen enthält, um sie korrekt dekodieren zu können. Weiters kann nach mehrfacher Ausführung auf Fehlermuster geschlossen werden, mit denen die Dekodierung gut bzw. schlecht umgehen kann. Durch Versuche mit anderen Kanalkodierungs-Methoden können Vergleiche mit diesen angestellt werden. Die genannten Punkte helfen dem Benutzer sein Verständnis für Faltungskodes noch besser zu stärken.

\section{Punktierung}
\label{kapitel:implementierung_punktierung}
Punktierung leicht, jedoch Depunktierung vor der Dekodierung nicht trivial.

%\section{Visualisierung}
%\label{kapitel:implementierung_visualisierung}

\chapter{R-Paket Schnittstelle}
\label{kapitel:interface}
% R Paket Schnittstelle für Faltungskodierung
\definecolor{lightblue}{RGB}{150,180,255}

\section{Kanalkodierung}
\label{chapter:interface_kanalkodierung}
\begin{longtable}{|p{\textwidth}|}
\hline
\rowcolor{lightblue}
ApplyNoise\\
\hline
\\
\texttt{ApplyNoise(msg, SNR.db, binary)}\\
\\
Verrauscht ein Signals basierend auf das AWGN (additive white gaussian noise) Modell. Das ist das Standardmodell für die Simulation eines Übertragungskanals.\\
\\
\textbf{Argumente:}\\
\texttt{msg} - Nachricht die verrauscht wird.\\
\texttt{SNR.db} - Signal/Rausch-Verhältnis des Übertragungskanals. Standard: 3.0\\
\texttt{binary} - Blockkode-Parameter. Nicht zu verwenden! Standard: FALSE\\
\\
\textbf{Rückgabewert:}\\
Verrauschtes Signal.\\
\\
\hline
\caption{ApplyNoise - Funktionserklärung}
\end{longtable}

\chapter{Visualisierung}
\label{kapitel:visualisierung}
% visualisierung.tex
LIMITS!\\\\
%Da das R-Paket für Studenten zu Lernzwecken verwendet werden soll
Um das Verständnis für Faltungskodes beim Benutzer dieses R-Pakets zu stärken, stehen Visualisierungen der Kodierung und Dekodierung zur Verfügung.
\\
Wird der \texttt{visualize} Parameter bei der Ausführung einer Funktion zur Kodierung oder Dekodierung auf TRUE~\texttt{TRUE} gesetzt, wird ein R Markdown Skript ausgeführt. Dieses generiert eine Beamer Präsentation mit Informationen und Visualisierungen zur Kodierung bzw. Dekodierung.
\\\\
\textcolor{lightgray}{\ref{chapter:visualisierung}.1 Kodierung}\\
% allg. Informationen
Bei der Kodierung befinden sich auf den ersten Folien allgemeine Informationen zum verwendeten Faltungskodierer wie die Kode-Rate, Generatorpolynome, Zustandsübergangstabelle etc. 
% Kodierung
Daraufhin folgt die Kodierungsvisualisierung. Diese zeigt zunächst die zu kodierende Nachricht (Input), das Zustandsübergangsdiagramm sowie eine noch nicht befüllte Kodierungstabelle. Um für einen noch besseren Lerneffekt zu sorgen wird Schritt für Schritt mittels Overlays ein Bit des Inputs, der aktuelle Zustand, Folgezustand sowie der resultierende Output in eine neue Zeile der Kodierungstabelle geschrieben. Der aktuelle Zustand sowie der entsprechende Übergang werden im Diagramm farblich hervorgehoben. Die kodierte Nachricht wächst mit jedem Schritt bis schlussendlich die gesamte Nachricht kodiert wurde.
% Kode zu Signal
Da die Kodierungsfunktion nicht die Bitwerte des Kodeworts zurückliefert sondern die Signalwerte (für eine Übertragung über einen Kanal) wird auf einer weiteren Folie dargestellt, wie die Kodebits in Signalwerte überführt werden.\\
% Punktierung
Wird eine Punktierungsmatrix bei der Kodierung mitgegeben, wird eine zusätzliche Folie am Ende hinzugefügt. Auf dieser wird die Punktierung des Signals, d.h. das Entfernen von Signalwerten (definiert durch die Punktierungsmatrix) dargestellt. Dabei wird neben dem originalen Signal und der Punktierungsmatrix das punktierte Signal dargestellt, wobei zunächst die punktierten Signalwerte, d.h. die entfernten Werte, durch Asterisk-Symbole (\textasteriskcentered) ersetzt werden. Diese Darstellung dient als visueller Zwischenschritt für das danach folgende tatsächlich punktierte Signal, bei dem die punktierten Werte fehlen, was auch dem Rückgabewert der Funktion entspricht.
\\\\
\textcolor{lightgray}{\ref{chapter:visualisierung}.2 Dekodierung}\\
% allg. Informationen
Bei der Dekodierung befinden sich ebenfalls, wie bei der Kodierung, allgemeine Informationen des Faltungskodierers auf den ersten Folien.
% Signal zu Kode
Als Input erhält die Dekodierung das Kodewort als Signalwerte, die möglicherweise durch Anwendung der \texttt{ApplyNoise} Funktion verfälscht worden sind. Die soft decision Dekodierung verwendet zur Dekodierung zwar kontinuierliche Signalwerte, da aber sowohl die hard decision Dekodierung Bitwerde zur Dekodierung verwendet und Trellis-Diagramme mit Bitwerten beschriftet werden, wird auf einer Folie die Überführung der Signalwerte zu Bits dargestellt. Dieser transformierte Input wird auch als Input für die Visualisierung des Viterbi-Algorithmus verwendet.
% Viterbi-Algorithmus
Anschließend folgt die Visualisierung des Viterbi-Algorithmus mithilfe des Trellis-Diagramms. Zunächst werden, zur besseren Übersicht bei großen Diagrammen, jene Pfade entfernt, für die es eine bessere Alternative gibt, d.h. die eine größere Metrik bei hard decision Dekodierung bzw. eine kleinere Metrik bei soft decision Dekodierung als ihre Alternative haben. Danach erfolgt Schritt für Schritt mittels Backtracking die Rekonstruktion der Nachricht. Der gewählte Pfad beim Backtracking wird farblich hervorgehoben. Die übrigen Pfade werden ausgegraut. Am Ende befindet sich unter dem Trellis-Diagramm die farblich hervorgehobene dekodierte Nachricht.
% Punktierung
Wird eine Punktierungsmatrix bei der Dekodierung mitgegeben, wird eine zusätzliche Folie nach den Kodiererinformationen hinzugefügt. Auf dieser wird die Depunktierung des Signals, d.h. das Einfügen des Signalwerts 0 (definiert durch die Punktierungsmatrix), dargestellt. Die eingefügten 0-Werte sind zur leichteren visuellen Erkennung farblich hervorgehoben.
\\\\
\textcolor{lightgray}{\ref{chapter:visualisierung}.3 Simulation}\\
Weiters können Berichte der Simulation generiert werden, die die resultierenden Daten u.a. in einem Diagramm darstellen.

\chapter{Beispiele}
\label{kapitel:beispiele}

\chapter{Fazit, Ausblick, Erweiterungen}

\label{kapitel:fazit}

\cleardoublepage%

\listofabbreviations
\clearpage

\listoffigures
\clearpage

\listoftables
\clearpage

\lstlistoflistings
\clearpage

\printbibliography
\end{document}
