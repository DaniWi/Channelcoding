\documentclass[8pt,ignorenonframetext,]{beamer}
\setbeamertemplate{caption}[numbered]
\setbeamertemplate{caption label separator}{:}
\setbeamercolor{caption name}{fg=normal text.fg}
\usepackage{amssymb,amsmath}
\usepackage{ifxetex,ifluatex}
\usepackage{fixltx2e} % provides \textsubscript
\usepackage{lmodern}
\ifxetex
  \usepackage{fontspec,xltxtra,xunicode}
  \defaultfontfeatures{Mapping=tex-text,Scale=MatchLowercase}
  \newcommand{\euro}{€}
\else
  \ifluatex
    \usepackage{fontspec}
    \defaultfontfeatures{Mapping=tex-text,Scale=MatchLowercase}
    \newcommand{\euro}{€}
  \else
    \usepackage[T1]{fontenc}
    \usepackage[utf8]{inputenc}
      \fi
\fi
% use upquote if available, for straight quotes in verbatim environments
\IfFileExists{upquote.sty}{\usepackage{upquote}}{}
% use microtype if available
\IfFileExists{microtype.sty}{\usepackage{microtype}}{}



\setlength{\parindent}{0pt}
\setlength{\parskip}{6pt plus 2pt minus 1pt}
\setlength{\emergencystretch}{3em}  % prevent overfull lines
\setcounter{secnumdepth}{0}
\usepackage{tikz}
\usepackage{pgfplots}
\usepackage{mathtools}
\usetikzlibrary{arrows,decorations.pathmorphing,backgrounds,positioning,fit,petri,calc}

\title{Convolution Encode}
\date{03 Mai, 2016}

\begin{document}
\frame{\titlepage}

\begin{frame}{Kodierer Information}

\begin{itemize}
\itemsep1pt\parskip0pt\parsep0pt
\item
  Nicht-Rekursiver Kodierer
\item
  Anzahl von Ausgängen : \[N=2\]
\item
  Anzahl von Registern : \[M=2\]
\item
  Generatoren : \[(7,5)_8 = \begin{pmatrix}111 \\ 101 \\ \end{pmatrix}\]
\item
  Kode-Rate: \[\frac{1}{2}\]
\end{itemize}

\end{frame}

\begin{frame}{Kodierer Matrix : Nächster Zustand}

\begin{tabular}{l|c|c}
\hline
  & Bit 0 & Bit 1\\
\hline
Zustand  0 & 0 & 2\\
\hline
Zustand  1 & 0 & 2\\
\hline
Zustand  2 & 1 & 3\\
\hline
Zustand  3 & 1 & 3\\
\hline
\end{tabular}

\end{frame}

\begin{frame}{Kodierer Matrix : Ausgangsbits}

\begin{tabular}{l|c|c}
\hline
  & Bit 0 & Bit 1\\
\hline
Zustand  0 & 00 & 11\\
\hline
Zustand  1 & 11 & 00\\
\hline
Zustand  2 & 10 & 01\\
\hline
Zustand  3 & 01 & 10\\
\hline
\end{tabular}

\end{frame}

\begin{frame}{Convolution Encode}

input: (1, 0, 1, 0, 0)

\begin{center}
\begin{minipage}{.45\textwidth}
\begin{center}
\begin{tabular}{c c c c}
  \visible<1-> {state&input&output&next state\\ \hline}
   \visible<2-> { 00 & 1 & 11 & 10 \\  } \visible<3-> { 10 & 0 & 10 & 01 \\  } \visible<4-> { 01 & 1 & 00 & 10 \\ \hline } \visible<5-> { 10 & 0 & 10 & 01 \\  } \visible<6-> { 01 & 0 & 11 & 00 \\  }
\end{tabular}
\end{center}
\visible<6> {output: (0, 0, 0, 0, 0, 0, 0, 0, 0, 0)}
\end{minipage}
\hfill
\begin{minipage}{.45\textwidth}
\begin{center}
\begin{tikzpicture}[scale=.70, >=stealth, font=\tiny]
\tikzstyle{state} = [draw, circle, inner sep=1mm, minimum size=6mm, font=\scriptsize]
 \alt<2> {\node[state, orange] (state0) at (0,2.5) {00};}{\node[state] (state0) at (0,2.5) {00};}; \alt<3,5> {\node[state, orange] (state2) at (2.5,1.53080849893419e-16) {10};}{\node[state] (state2) at (2.5,1.53080849893419e-16) {10};}; \alt<0> {\node[state, orange] (state3) at (3.06161699786838e-16,-2.5) {11};}{\node[state] (state3) at (3.06161699786838e-16,-2.5) {11};}; \alt<4,6> {\node[state, orange] (state1) at (-2.5,-4.59242549680257e-16) {01};}{\node[state] (state1) at (-2.5,-4.59242549680257e-16) {01};}; \alt<0> {\draw[->, orange] (state0) to [looseness=8,out=105,in=75] node [sloped, above] {00} (state0) ;}{\draw[->] (state0) to [looseness=8,out=105,in=75] node [sloped, above] {00} (state0) ;}; \alt<2> {\draw[->, dashed, orange] (state0) to  node [sloped, above] {11} (state2) ;}{\draw[->, dashed] (state0) to  node [sloped, above] {11} (state2) ;}; \alt<6> {\draw[->, orange] (state1) to  node [sloped, above] {11} (state0) ;}{\draw[->] (state1) to  node [sloped, above] {11} (state0) ;}; \alt<4> {\draw[->, dashed, orange] (state1) to [bend left=15] node [sloped, above,near start] {00} (state2) ;}{\draw[->, dashed] (state1) to [bend left=15] node [sloped, above,near start] {00} (state2) ;}; \alt<3,5> {\draw[->, orange] (state2) to [bend left=15] node [sloped, above,near start] {10} (state1) ;}{\draw[->] (state2) to [bend left=15] node [sloped, above,near start] {10} (state1) ;}; \alt<0> {\draw[->, dashed, orange] (state2) to  node [sloped, above] {01} (state3) ;}{\draw[->, dashed] (state2) to  node [sloped, above] {01} (state3) ;}; \alt<0> {\draw[->, orange] (state3) to  node [sloped, above] {01} (state1) ;}{\draw[->] (state3) to  node [sloped, above] {01} (state1) ;}; \alt<0> {\draw[->, dashed, orange] (state3) to [looseness=8,out=285,in=255] node [sloped, below] {10} (state3) ;}{\draw[->, dashed] (state3) to [looseness=8,out=285,in=255] node [sloped, below] {10} (state3) ;};
\end{tikzpicture}
\end{center}
\end{minipage}
\end{center}

\end{frame}

\end{document}
